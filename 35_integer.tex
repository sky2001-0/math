\lsection{整数}

\lsubsection{整数の構成}

\lem{整数の準備:前順序}{
  直積集合$\N^2$上で以下の自己関係$\preccurlyeq_Z$は前順序である。
  \eq*{
    \forall \qty(a_1, a_2), \qty(b_1, b_2) \in \N^2 \qty(\qty(a_1, a_2) \preccurlyeq_Z \qty(b_1, b_2) \defiff a_1 + b_2 \leq b_1 + a_2)
  }
}{
  反射的である。\\*

  推移的であることを示す。
  \eq*{
    a_1 + b_2 \leq b_1 + a_2 \land b_1 + c_2 \leq c_1 + b_2 \rightarrow a_1 + b_2 + b_1 + c_2 \leq b_1 + a_2 + c_1 + b_2
  }

  \lemref{自然数の順序の加法による保存}の対偶をとって、
  \eq*{
    a_1 + c_2 \leq c_1 + a_2
  }

  したがって、前順序である。
}

\lem{整数の準備:前順序の全域性}{
  直積集合$\N^2$上で\lemref{整数の準備:前順序}の前順序$\preccurlyeq_Z$は以下を満たす。
  \eq*{
    \forall a, b \in \N^2 \qty(a \preccurlyeq_Z b \lor b \preccurlyeq_Z a)
  }
}{
  \thmref{自然数の全順序}より明らか。
}

\lem{整数の準備:加法可換群}{
  直積集合$\N^2$上の以下の演算$+$は可換群をなす。
  \eq*{
    \forall \qty(a_1, a_2), \qty(b_1, b_2) \in \N^2 \qty(\qty(a_1, a_2) + \qty(b_1, b_2) \defiff \qty(a_1 + b_1, a_2 + b_2))
  }
}{
  結合法則は、\thmref{自然数は可換モノイド}より明らか。

  $\qty(0, 0)$は単位元である。

  $\qty(a_1, a_2)$に対して、$\qty(a_2, a_1)$は逆元である。

  交換法則は、\thmref{自然数は可換モノイド}より明らか。
}

\lem{整数の準備:可換環}{
  \lemref{整数の準備:加法可換群}の定める可換群$\qty(\N^2, +)$上の以下の演算$\times$は可換環をなす。
  \eq*{
    \forall \qty(a_1, a_2), \qty(b_1, b_2) \in \N^2 \qty(\qty(a_1, a_2) \times \qty(b_1, b_2) \defiff \qty(a_1 b_1 + a_2 b_2, a_1 b_2 + a_2 b_1))
  }
}{
  結合法則を示す。
  \eqa*{
    \qty(a_1, a_2) \times \qty(\qty(b_1, b_2) \times \qty(c_1, c_2))
    &= \qty(a_1, a_2) \times \qty(b_1 c_1 + b_2 c_2, b_1 c_2 + b_2 c_1) \\*
    &= \qty(a_1 \qty(b_1 c_1 + b_2 c_2) + a_2 \qty(b_1 c_2 + b_2 c_1) , a_1 \qty(b_1 c_2 + b_2 c_1) + a_2 \qty(b_1 c_1 + b_2 c_2)) \\*
    &= \qty(\qty(a_1 b_1 + a_2 b_2) c_1 + \qty(a_1 b_2 + a_2 b_1) c_2 , \qty(a_1 b_1 + a_2 b_2) c_2 + \qty(a_1 b_2 + a_2 b_1) c_1) \\*
    &= \qty(\qty(a_1, a_2) \times \qty(b_1, b_2)) \times \qty(c_1, c_2)
  }

  $\qty(1, 0)$は単位元である。\\*

  交換法則は、\thmref{自然数は可換モノイド}より明らか。\\*

  右分配法則を示す。
  \eqa*{
    \qty(\qty(a_1, a_2) + \qty(b_1, b_2)) \times \qty(c_1, c_2)
    &= \qty(a_1 + b_1, a_2 + b_2) \times \qty(c_1, c_2) \\*
    &= \qty(\qty(a_1 + b_1) c_1 + \qty(a_2 + b_2) c_2, \qty(a_1 + b_1) c_2 + \qty(a_2 + b_2) c_1) \\*
    &= \qty(a_1 c_1 + a_2 c_2 + b_1 c_1 + b_2 c_2, a_1 c_2 + a_2 c_1 + b_1 c_2 + b_2 c_1) \\*
    &= \qty(a_1, a_2) \times \qty(c_1, c_2) + \qty(b_1, b_2) \times \qty(c_1, c_2)
  }

  左分配法則は、右分配法則と交換法則より示される。
}

\lem{整数の準備:同値類}{
  直積集合$\N^2$上で以下の自己関係$\sim_Z$は同値類である。
  \eq*{
    \forall \qty(a_1, a_2), \qty(b_1, b_2) \in \N^2 \qty(\qty(a_1, a_2) \sim_Z \qty(b_1, b_2) \defiff \qty(a_1, a_2) \preccurlyeq \qty(b_1, b_2) \land \qty(b_1, b_2) \preccurlyeq \qty(a_1, a_2))
  }
}{
  \lemref{整数の準備:前順序}より前順序。明らかに対称的であるので、同値類である。
}

\lem{整数の準備:両立}{
  直積集合$\N^2$上で\lemref{整数の準備:同値類}の同値類$\sim_Z$は、\lemref{整数の準備:前順序}の前順序、\lemref{整数の準備:加法可換群}の加法、\lemref{整数の準備:可換環}の乗法のそれぞれと両立する。
}{
  前順序と両立することは、$\sim_Z$の定義と前順序の推移性より明らか。\\*

  $\qty(a_1, a_2) \sim_Z \qty(a'_1, a'_2) \land \qty(b_1, b_2) \sim_Z \qty(b'_1, b'_2)$とする。

  $\qty(a_1 + b_1) + \qty(a'_2 + b'_2) = \qty(a_1 + a'_2) + \qty(b_1 + b'_2) = \qty(a'_1 + a_2) + \qty(b'_1 + b_2) = \qty(a'_1 + b'_1) + \qty(a_2 + b_2)$より成り立つ。\\*

  $\qty(a_1, a_2) \sim_Z \qty(a'_1, a'_2) \land \qty(b_1, b_2) \sim_Z \qty(b'_1, b'_2)$とする。
  \eqa*{
    a_2 \qty(b_2 + b'_1) &= a_2 \qty(b_1 + b'_2) \\*
    a_2 \qty(b_2 + b'_1) + a'_1 b'_1 &= a_2 \qty(b_1 + b'_2) + a'_1 b'_1 \\*
    a_2 b_2 + \qty(a_2 + a'_1) b'_1 &= a_2 \qty(b_1 + b'_2) + a'_1 b'_1 \\*
    a_2 b_2 + \qty(a_1 + a'_2) b'_1 &= a_2 \qty(b_1 + b'_2) + a'_1 b'_1 \\*
    a_2 b_2 + \qty(a_1 + a'_2) b'_1 + a_1 b_2 + a'_1 b'_2 &= a_2 \qty(b_1 + b'_2) + a'_1 b'_1 + a_1 b_2 + a'_1 b'_2 \\*
    a_2 b_2 + a_1 \qty(b'_1 + b_2) + a'_2 b'_1 + a'_1 b'_2 &= a_2 b_1 + \qty(a'_1 + a_2) b'_2 + a'_1 b'_1 + a_1 b_2 \\*
    a_2 b_2 + a_1 \qty(b_1 + b'_2) + a'_2 b'_1 + a'_1 b'_2 &= a_2 b_1 + \qty(a_1 + a'_2) b'_2 + a'_1 b'_1 + a_1 b_2 \\*
    a_1 b_1 + a_2 b_2 + a_1 b'_2 + a'_1 b'_2 + a'_2 b'_1 &= a'_1 b'_1 + a'_2 b'_2 + a_1 b'_2 + a_1 b_2 + a_2 b_1
  }
  \dfnref{自然数の減法}より
  \eq*{
    a_1 b_1 + a_2 b_2 + a'_1 b'_2 + a'_2 b'_1 = a_1 b_2 + a_2 b_1 + a'_1 b'_1 + a'_2 b'_2
  }
}

\dfn{整数}{
  \lemref{整数の準備:両立}より定まる商マグマ$\N^2 / \sim_Z$を整数と呼び、$\Z$で表す。
  また、その元も整数と呼ぶ。

  定義より$\Z$は可換環である。
}


\lsubsection{整数の性質}

\lem*{
  可換環$\Z$は零環でない。
}{
  $0_\Z = \qty[\qty(0, 0)], 1_\Z = \qty[\qty(1, 0)]$より明らか。
}

\lem{整数の全順序}{
  \lemref{整数の準備:両立}より与えられる$\Z$上の前順序$\leq$は全順序である。
}{
  $\sim_Z$の定義より反対称的である。

  \lemref{整数の準備:前順序の全域性}より全順序である。
}

\thm{順序環としての整数}{
  $\qty(\Z, \leq)$は順序環である。すなわち$\Z$は零環でない可換環で、$\leq$は全順序であり、かつ以下を満たす。
  \eqg*{
    \forall a, b, c \in \Z \qty(a \leq b \rightarrow a + c \leq b + c) \\*
    \forall a, b \in \Z \qty(0_\Z < a \land 0_\Z < b \rightarrow 0_\Z < a b)
  }
}{
  \lemref{整数の全順序}より全順序。\mlemref{-2}より零環でない。\\*

  第一式について、$a = \qty[\qty(a_1, a_2)], b = \qty[\qty(b_1, b_2)], c = \qty[\qty(c_1, c_2)]$として、\lemref{自然数の順序の加法による保存}より、
  \eq*{
    a_1 + b_2 < a_2 + b_1 \rightarrow \qty(a_1 + c_1) + \qty(b_2 + c_2) < \qty(a_2 + c_2) + \qty(b_1 + c_1)
  }

  第二式について、$0_\Z < a, 0_\Z < b$より、\dfnref{自然数の減法}より$a = \qty[\qty(a', 0)], b = \qty[\qty(b', 0)]$と置ける。
  \eq*{
    a b = \qty[\qty(a', 0)] \times \qty[\qty(b', 0)] = \qty[\qty(a' b', 0)] > 0
  }
}

\lem*{
  写像$\varphi \in \Z^\N, \varphi(n) = \qty[\qty(n, 0)]$は以下を満たす。
  \begin{enumerate}
    \item 単射
    \item 加法についてモノイド準同型
    \item 乗法についてモノイド準同型
    \item 全順序と両立
  \end{enumerate}
}{
  定義より明らか。
}

\dfn{自然数の整数への埋め込み}{
  \mlemref{0}の写像$\varphi$について、像$\varphi(\N)$を誤解のない範囲で自然数$\N$と呼ぶ。
}

\cor*{
  \eq*{
    \N \subset \Z
  }
}

\thmf{整数は\textit{Euclid}整域}{整数はEuclid整域}{
  整数$\Z$は\textit{Euclid}整域である。
}{
  以下で定める写像$\sign \in \qty{1, -1}^\Z$を考える。
  \eq*{
    \sign(x) =
    \begin{cases}
      1 & \qty(x \geq 0) \\*
      -1 & \qty(x < 0)
    \end{cases}
  }

  写像$\abs{} \in \Z^\Z, \abs{x} = x \sign(x)$を考える。

  ここで、$\abs{\Z} = \N$である。

  $\forall a \in \Z \forall b \in \Z \setminus \qty{0}$について、\dfnref{自然数の除法}より以下が成り立つ。
  \eq*{
    \exists q, r \in \N \qty(\abs{a} = \abs{b} q + r \land r < \abs{b})
  }

  ゆえに、
  \eq*{
    a = b \qty(\sign(a) \sign(b) q) + \qty(\sign(a) r) \land \abs{\sign(a) r} < \abs{b}
  }

  また、$\forall a, b \in \Z \setminus \qty{0}$について、$1 \leq \abs{a} \leq \abs{a b}$
}

\lem{整数の乗法の簡約則}{
  \eq*{
    \forall a, b, c \in \Z \qty(c \neq 0 \land a c = b c \rightarrow a = b)
  }
}{
  \thmref{整数はEuclid整域}および\lemref{簡約則}より明らか。
}

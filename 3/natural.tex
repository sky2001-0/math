\lsection{自然数}

\lsubsection{自然数の公理}

\dfn{自然数}{
  自然数論では、\num{3}つの公理(
    \axiref{零は後者ではない}、\axiref{後者は単射}、\axiref{数学的帰納法}
  )が与えられる。

  項を自然数と呼ぶ。
}

\dfn{零}{
  アリティ\num{0}の函数記号$0$を考える。
}

\dfn{後者}{
  アリティ\num{1}の函数記号$s$を考える。

  自然数$n$について、$s(n)$を$n$の後者と呼ぶ。
}

\axi{零は後者ではない}{
  \eq*{
    \forall n \qty(s(n) \neq 0)
  }
}

\axi{後者は単射}{
  \eq*{
    \forall n, m \qty(s(n) = s(m) \rightarrow n = m)
  }
}

\axi{数学的帰納法}{
  アリティ\num{1}の述語記号$\psi$について、以下が成り立つ。
  \eq*{
    \psi(0) \land \forall k \qty\Big(\psi(k) \rightarrow \psi(s(k))) \rightarrow \forall n \qty(\psi(n))
  }
}


\lsubsection{集合論における自然数}

\thm{Peanoの公理}{
  \thmref{最小無限系譜の一意性}より、無限系譜$X$の取り方によらず$\omega(X)$が一意に定まる。
  これを$\N$で表す。

  $\N$は以下の全てを満たす。
  \eqg*{
    \varnothing \in \N \\*
    \forall n \in \N \qty(n^+ \in \N) \\*
    \forall n \in \N \qty(n^+ \neq \varnothing) \\*
    \forall n, m \in \N \qty(n^+ = m^+ \rightarrow n = m) \\*
    \forall E \qty(E \subset \N \land \varnothing \in E \land \forall n \qty(n \in E \rightarrow n^+ \in E) \rightarrow E = \N)
  }
}{
  第一、第二の公理は、無限系譜であることより明らか。

  \vskip\baselineskip

  第三の公理は$n \in n^+$と、空集合は要素を持たないことから示される。

  \vskip\baselineskip

  第四の公理は、\lemref{後者は単射的}より示される。

  \vskip\baselineskip

  第五の公理を示す。

  前件より、$E \subset \N$である。

  $\varnothing \in E \land \forall n \qty(n \in E \rightarrow n^+ \in E)$より、$E$は無限系譜である。

  最小無限系譜であるので、$\N = \omega(E) \subset E$である。
}

\thm{数学的帰納法}{
  アリティ\num{1}の述語記号$\psi$について、
  \eq*{
    \forall n \in \N \qty(\psi(n)) \leftrightarrow \psi(0) \land \forall n \in \N \qty\Big(\psi(n) \rightarrow \psi(n^+))
  }
}{
  右は明らか。

  \vskip\baselineskip

  左は、$E = \qty{n \in \N \mid \psi(n)}$として、\thmref{Peanoの公理}第五式から示される。
}

\dfn{自然数集合}{
  \thmref{Peanoの公理}、\thmref{数学的帰納法}より、以下と$n \in \N$で定める議論領域は自然数である。
  \eqa*{
    0 &\coloneqq \varnothing \\*
    s(n) &\coloneqq n^+
  }

  $\N$を自然数集合と呼ぶ。

  誤解のない限り、自然数集合を自然数と呼ぶ
  。
}

\lem*{
  以下が成り立つ。
  \eq*{
    \forall n \in \N \qty(\text{$n$は順序数})
  }
}{
  $n = 0$のとき、$\varnothing$は順序数より成り立つ。

  ある$n$が順序数であるとき、\corref{順序数の後者は順序数}より$s(n)$は順序数である。

  \thmref{数学的帰納法}より、任意の自然数$n$について成り立つ。
}

\dfn{自然数の順序}{
  自然数$\N$上で二項関係$\leq$を以下のように定める。
  \lemref{順序数の比較可能性}より、$\leq$は全順序である。
  \eq*{
    n \leq m \defiff n \subset m
  }
}

\lem{自然数は推移的}{
  以下が成り立つ。
  \eq*{
    \forall n \in \N \qty(m \in n \rightarrow m \in \N)
  }
}{
  $n = 0 = \varnothing$のとき明らか。

  ある$n$で成り立つとき、$s(n) = n \cup \qty{n}$より、$s(n)$で成り立つ。

  \thmref{数学的帰納法}より、任意の自然数$n$について成り立つ。
}

\lem{最小値原理}{
  自然数$\N$の空でない部分集合$A$は最小元を持つ。
}{
  持たないと仮定する。

  このとき、$\forall n \in \N \forall m \in \N \qty(m \leq n \rightarrow m \notin A)$を示す。

  $n = 0$のとき、$0 \in A$ならば仮定に反するので明らか。

  ある$n$で成り立つとする。$s(n) \in A$ならば$s(n)$は$A$の最小元となるので、$s(n) \notin A$。

  \thmref{数学的帰納法}を用いて上の命題が示されるので、$\forall n \in \N \qty(n \notin A)$が直ちに言える。

  これは$A$が空でないことに反する。背理法より、示される。
}

\thm{自然数は順序数}{
  自然数$\N$は順序数である。
}{
  \lemref{自然数は推移的}と\lemref{最小値原理}より、順序数である。
}


\lsubsection{自然数の加法}

\dfn{自然数の加法}{
  自然数$n$について、以下で定める写像$G_n \colon \bigcup \qty{\N^m \mid m \in \N} \to \N$を考える。
  \eq*{
    G_n(h) \coloneqq
    \begin{cases}
      n & \qty(\dom(h) = 0) \\*
      s(h(\bigcup \dom(h))) & \qty(\dom(h) \neq 0)
    \end{cases}
  }

  このとき、\thmref{超限再帰}が定める写像$+_n \colon \N \to \N$が存在する。

  誤解のない限り、$+_n(m)$を$n + m$と表す。
}

\lem*{
  以下が成り立つ。
  \eqg*{
    \forall n \in \N \qty(n + 0 = n) \\*
    \forall n, m \in \N \qty(n + s(m) = s(n + m))
  }
}{
  第一式について、定義より$+_n(0) = G(+_n \circ \iota_{0, \N}) = n$である。

  第二式について、定義より$+_n(s(m)) = G(+_n \circ \iota_{s(m), \N}) = s(+_n(\bigcup s(m))) = s(+_n(m)) = s(n + m)$である。
}

\lem{自然数の加法の結合法則}{
  \eq*{
    \forall n, m, l \in \N \qty(\qty(n + m) + l = n + \qty(m + l))
  }
  つまり上の等式の両辺は、$n + m + l$と表記してもよい。
}{
  $l = 0$のとき、以下より成り立つ。
  \eq*{
    \qty(n + m) + 0 = n + m = n + \qty(m + 0)
  }

  ある自然数$l$で成り立つとき、
  \eqa*{
    \qty(n + m) + s(l) &= s(\qty(n + m) + l) \\*
    &= s(n + \qty(m + l)) \\*
    &= n + s(m + l) \\*
    &= n + \qty(m + s(l))
  }

  \thmref{数学的帰納法}より、任意の自然数$l$について示される。
}

\lem{自然数の加法の単位元}{
  \eq*{
    \forall n \in \N \qty(n + 0 = 0 + n = n)
  }
}{
  $\forall n \in \N \qty(n + 0 = n)$はすでに示されている。

  $\forall n \in \N \qty(0 + n = n)$を示す。

  $n = 0$のとき、$0 + 0 = 0$より満たす。

  ある自然数$n$で成り立つとき、
  \eqa*{
    0 + s(n) = s(0 + n) = s(n)
  }

  \thmref{数学的帰納法}より、任意の自然数$n$について示される。
}

\lem*{
  \eq*{
    \forall n, m \in \N \qty(s(n) + m = s(n + m))
  }
}{
  $m = 0$のとき、以下より成り立つ。
  \eq*{
    s(n) + 0 = s(n) = s(n + 0)
  }

  ある自然数$m$について成り立つとき、
  \eq*{
    s(n) + s(m) = s(s(n) + m) = s(s(n + m)) = s(n + s(m))
  }

  \thmref{数学的帰納法}より、任意の自然数$m$について示される。
}

\lem{自然数の加法の交換法則}{
  \eq*{
    \forall n, m \in \N \qty(n + m = m + n)
  }
}{
  $m = 0$のとき、\lemref{自然数の加法の単位元}より示される。

  ある$m$で成り立つとき、\mlemref{-1}より、
  \eq*{
    n + s(m) = s(n + m) = s(m + n) = s(m) + n
  }

  \thmref{数学的帰納法}より、任意の自然数$m$について示される。
}

\lem{自然数の順序の加法による保存}{
  \eq*{
    \forall n, m, l \in \N \qty(n < m \rightarrow n + l < m + l)
  }
}{
  $l = 0$のとき、明らか。

  ある自然数$l$で成り立つとき、\lemref{順序数の後者と比較}より、
  \eq*{
    n + s(l) = s(n + l) < s(m + l) = m + s(l)
  }

  \thmref{数学的帰納法}より、任意の自然数$n$について示される。
}

\lem{自然数の順序の加法による特徴づけ}{
  \eq*{
    \forall n, m \in \N \qty(m \leq n \leftrightarrow \exists k \in \N \qty(m + k = n))
  }

  右について、この$k$は一意に定まる。
}{
  左は、全順序性、\lemref{自然数の順序の加法による保存}、$\forall k \in \N \qty(0 \leq k)$を用いて、背理法より示される。

  右について考える。

  $n = 0$のとき、$m = 0$ならば$k = 0$が存在。$m \neq 0$ならば前件否定。

  ある$n$で成り立つときを考える。$n < m \lor m \leq n$である。

  $n < m$のとき、\lemref{順序数の後者と比較}より$s(n) \leq m$である。
  $s(n) < m$のとき前件否定より自明。$s(n) = m$のとき、$k = 0$で存在。

  $m \leq n$のとき、仮定より$\exists k \in \N \qty(m + k = n)$

  ゆえに、$m \leq s(n)$かつ$s(n) = s(m + k) = m + s(k)$となる。

  \thmref{数学的帰納法}より、任意の$n$について示される。\\*

  $k < k'$とすると、\lemref{自然数の順序の加法による保存}より矛盾。$k > k'$でも同様より、示される。
}

\dfn{自然数の減法}{
  自然数$n, m$について、$m \leq n$であるとき、\lemref{自然数の順序の加法による特徴づけ}より一意に定まる$k$を$n - m$と表す。
}


\lsubsection{自然数の乗法}

\dfn{自然数の乗法}{
  自然数$n$について、以下で定める写像$G_n \colon \bigcup \qty{\N^m \mid m \in \N} \to \N$を考える。
  \eq*{
    G_n(h) \coloneqq
    \begin{cases}
      0 & \qty(\dom(h) = 0) \\*
      h(\bigcup \dom(h)) + n & \qty(\dom(h) \neq 0)
    \end{cases}
  }

  このとき、\thmref{超限再帰}が定める写像$\times_n \colon \N \to \N$が存在する。

  誤解のない限り、$\times_n(m)$を$n \times m$と表す。
}

\lem*{
  以下が成り立つ。
  \eqg*{
    \forall n \in \N \qty(n \times 0 = 0) \\*
    \forall n, m \in \N \qty(n \times s(m) = n \times m + n)
  }
}{
  第一式について、定義より$\times_n(0) = G(\times_n \circ \iota_{0, \N}) = 0$である。

  第二式について、定義より$\times_n(s(m)) = G(\times_n \circ \iota_{s(m), \N}) = \times_n(\bigcup s(m)) + n = \times_n(m) + n$である。
}

\lem{左零元}{
  \eq*{
    \forall n \in \N \qty(0 \times n = 0)
  }
}{
  $n = 0$のとき、$0 \times 0 = 0$

  ある自然数$n$で成り立つとき、
  \eq*{
    0 \times s(n) = 0 \times n + 0 = 0 + 0 = 0
  }

  \thmref{数学的帰納法}より、任意の自然数$n$について示される。
}

\dfn{1}{
  自然数において、$1 \coloneqq s(0)$で定める。
}

\lem{自然数の乗法の単位元}{
  \eq*{
    \forall n \in \N \qty(n = n \times 1 = 1 \times n)
  }
}{
  左について$1 = s(0)$より、
  \eq*{
    n \times 1 = n \times s(0) = n \times 0 + n = 0 + n = n
  }

  右について考える。

  $n = 0$のとき、$1 \times 0 = 0$

  ある自然数$n$で成り立つとき、
  \eq*{
    1 \times s(n) = 1 \times n + 1 = n + 1 = n + s(0) = s(n + 0) = s(n)
  }

  \thmref{数学的帰納法}より、任意の自然数$n$について示される。
}

\lem{自然数の右分配法則}{
  \eq*{
    \forall n, m, l \in \N \qty(\qty(n + m) \times l = n \times l + m \times l)
  }
}{
  $l = 0$のとき、以下より成り立つ。
  \eq*{
    \qty(n + m) \times 0 = 0 = 0 + 0 = n \times 0 + m \times 0
  }

  ある自然数$l$で成り立つとき、
  \eqa*{
    \qty(n + m) \times s(l) &= \qty(n + m) \times l + \qty(n + m) \\*
    &= n \times l + m \times l + n + m \\*
    &= n \times s(l) + m \times s(l)
  }

  \thmref{数学的帰納法}より、任意の自然数$l$について示される。
}

\lem{自然数の乗法の交換法則}{
  \eq*{
    \forall n, m \in \N \qty(n \times m = m \times n)
  }
}{
  \lemref{左零元}より、$m = 0$について、$n \times 0 = 0 \times n$

  ある自然数$m$で成り立つとき、
  \eqa*{
    n \times s(m) &=  n \times m + n\\*
    &= m \times n + n \\*
    &= m \times n + 1 \times n \\*
    &= \qty(m + 1) \times n \\*
    &= s(m) \times n
  }

  \thmref{数学的帰納法}より、任意の自然数$m$について示される。
}

\lem{自然数の分配法則}{
  \eqg*{
    \forall n, m, l \in \N \qty(\qty(n + m) \times l = n \times l + m \times l) \\*
    \forall n, m, l \in \N \qty(n \times \qty(m + l) = n \times m + n \times l)
  }
}{
  右分配法則と交換法則より示される。
}

\lem{自然数の乗法の結合法則}{
  \eq*{
    \forall n, m, l \in \N \qty(\qty(n \times m) \times l = n \times \qty(m \times l))
  }
  つまり上の等式の両辺は、$n \times m \times l$と表記してもよい。
}{
  $l = 0$のとき、以下より成り立つ。
  \eq*{
    \qty(n \times m) \times 0 = 0 = n \times 0 = n \times \qty(m \times 0)
  }

  ある自然数$l$で成り立つとき、
  \eqa*{
    \qty(n \times m) \times s(l) &= \qty(n \times m) \times l + n \times m \\*
    &= n \times \qty(m \times l) + n \times m \\*
    &= \qty(m \times l) \times n + m \times n \\*
    &= \qty(m \times l + m) \times n \\*
    &= \qty(m \times l + m \times 1) \times n \\*
    &= n \times \qty(m \times \qty(l + 1)) \\*
    &= n \times \qty(m \times s(l))
  }

  \thmref{数学的帰納法}より、任意の自然数$l$について示される。
}

\lem{自然数の順序の乗法による保存}{
  \eq*{
    \forall n, m, l \in \N \qty(\qty(n \neq 0 \land m < l) \rightarrow n \times m < n \times l)
  }
}{
  $n = 0$のとき、前件否定より明らか。

  $n = 1$のとき、$1$が乗法の単位元であることから明らか。

  ある自然数$n \geq 1$で成り立つとき、
  \eq*{
    s(n) \times m = n \times m + m < n \times m + l < n \times l + l = s(n) \times l
  }

  \thmref{数学的帰納法}より、任意の自然数$n$について示される。
}

\lem{自然数の乗法の簡約則}{
  \eq*{
    \forall n, m, l \in \N \qty(n \times m = n \times l \land n \neq 0 \rightarrow m = l)
  }
}{
  $m < l$のとき、\lemref{自然数の順序の乗法による保存}より$n \times m < n \times l$。ゆえに、$n \times m \neq n \times l$。

  $l < m$のとき、同様に$n \times m \neq n \times l$。

  背理法より示される。
}

\lem*{
  \eq*{
    \forall n, m \in \N \qty(n \times m = 0 \rightarrow n = 0 \lor m = 0)
  }
}{
  $n = 0$のとき自明。

  $n \neq 0$のとき、$n \times m = n \times 0$より、自然数の乗法の簡約則から$m = 0$。
}


\lsubsection{自然数の除法}

\dfn{自然数の除法}{
  $0$でない自然数$b$について、以下で定める写像$G_b \colon \bigcup \qty{\qty(\N \times \N)^a \mid a \in \N} \to \N \times \N$を考える。
  \eq*{
    G_n(h) \coloneqq
    \begin{cases}
      \qty(0, 0) & \qty(\dom(h) = 0) \\*
      \qty(q, s(r)) & \qty(\dom(h) \neq 0 \land s(r) \in b) \\*
      \qty(s(q), 0) & \qty(\dom(h) \neq 0 \land s(r) \notin b)
    \end{cases}
    \hspace{3em}
    \text{ただし、$\qty(q, r) \coloneqq h(\bigcup \dom h)$とする。}
  }

  このとき、\thmref{超限再帰}が定める写像$\divisionsymbol_b \colon \N \to \N \times \N$が存在する。

  誤解のない限り、$\divisionsymbol_b(a) = \qty(q, r)$を$a \divisionsymbol b = q \cdots r$と表す。

  特に、$q$を商、$r$を余りと呼ぶ。
}

\lem*{
  以下が成り立つ。
  \eqg*{
    \forall b \in \N \setminus \qty{0} \qty(0 \divisionsymbol b \coloneqq 0 \cdots 0) \\*
    \forall a \in \N \forall b \in \N \setminus \qty{0} \qty(s(a) \divisionsymbol b = q \cdots s(r))
  }
}{
  第一式について、定義より$\times_n(0) = G(\times_n \circ \iota_{0, \N}) = 0$である。

  第二式について、定義より$\times_n(s(m)) = G(\times_n \circ \iota_{s(m), \N}) = \times_n(\bigcup s(m)) + n = \times_n(m) + n$である。
}

\dfn{倍数}{
  自然数$n, m$について、$n \divisionsymbol m$の余りが$0$のとき、$n$は$m$の倍数、または$m$は$n$の約数と言う。
}

\cor*{
  任意の自然数$n$について、$0$は$n$の倍数であり、$1$は$n$の約数である。
  また、$n$は$n$の倍数でも約数でもある。
}

\dfn{2}{
  自然数において、$2 \coloneqq s(s(0))$で定める。
}

\dfn{偶奇}{
	自然数$n$について、$2$で割った余りが$0$であるとき、$n$を偶数と呼ぶ。

	そうでないとき、$n$を奇数と呼ぶ。
}

\dfn{素数}{
	約数をちょうど\num{2}つ持つ自然数を素数と呼ぶ。
}

\cor*{
  素数$p$の約数は、$1$と$p$である。
}


\lsubsection{点列}

\dfn{有限列}{
  自然数$n \in \N$から集合$X$へのネットを、$X$上の有限列、または組、$n$-組と呼ぶ。
}

\dfn{点列}{
  自然数$\N$から集合$X$へのネットを、$X$上の点列と呼ぶ。
}

\dfn{点列の部分列}{
  単射$\varphi \colon \N \to \N$を用いて表せる部分ネット$\qty(x_{\varphi(m)})_{m \in \N}$を点列$\qty(x_n)_{n \in \N}$の部分列と呼ぶ。
}

\dfn{非交叉列}{
  点列$\qty(x_n)_{n \in \N}$について以下が成り立つとき、$\qty(x_n)_{n \in \N}$を非交叉列と呼ぶ。
  \eq*{
    \forall n, m \in \N \qty(n \neq m \rightarrow x_n \cap x_m = \varnothing)
  }
}

\dfn{単調列}{
  半順序集合$X$上の点列$\qty(x_n)_{n \in \N}$について、以下のいずれかを満たす点列を単調列と呼ぶ。
  \eqg*{
    \forall n \in \N \qty(x_n < x_{s(n)}) \\*
    \forall n \in \N \qty(x_n \leq x_{s(n)}) \\*
    \forall n \in \N \qty(x_n > x_{s(n)}) \\*
    \forall n \in \N \qty(x_n \geq x_{s(n)})
  }

  特にそれぞれ、第一式を満たす点列を狭義単調増加列、第二式を広義単調増加列、第三式を狭義単調減少列、第四式を広義単調減少列と呼ぶ。
}

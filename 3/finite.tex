\lsection{有限と可算}

\lsubsection{有限}

\dfn{有限集合}{
  集合$X$について、ある自然数$n \in \N$が存在して全単射$f \colon n \to X$が構成できるとき、$X$を有限集合と呼ぶ。
  または、単に$X$は有限であると呼ぶ。
}

\lem{全射と要素数}{
  $\forall n, m \in \N$について、$n$から$m$への全射が存在するならば、$m \leq n$
}{
  $n = 0$のとき、$m \neq 0$ならば全射が存在しない。$m = 0$より示される。

  \vskip\baselineskip

  ある自然数$n$で成り立つとする。

  全射$f \colon s(n) \to m$が存在するとする。

  $\Hom_{\mathbf{Set}}\qty(s(n), 0) = \varnothing$より、$m \neq 0$である。

  $\exists k \in n \qty(f(k) = f(n))$のとき、全射$f \circ \iota_{n, s(n)}$を構成できて、仮定より$m \leq n \leq s(n)$

  $\forall k \in n \qty(f(k) \neq f(n))$のとき、以下のような全射$g \colon n \to \bigcup m$を構成できる。
  \eq*{
    g(x) =
    \begin{cases}
      f(x) & \qty(f(x) < f(n)) \\*
      \bigcup f(x) & \qty(f(x) > f(n))
    \end{cases}
  }

  仮定より$\bigcup m \leq n$より、$m = s(\bigcup m) \leq s(n)$

  \vskip\baselineskip

  \thmref{数学的帰納法}より、任意の自然数$n$について成り立つ。
}

\lem{単射と要素数}{
  自然数$n, m \in \N$について、$n$から$m$への単射が存在するならば、$n \leq m$
}{
  $n = 0$のとき、明らか。

  $n \neq 0$のとき、\lemref{単射}より$m$から$n$への左逆写像すなわち全射を構成できて、\lemref{全射と要素数}より$n \leq m$
}

\thm{鳩の巣原理}{
  自然数$n, m \in \N$について、$m < n$ならば$n$から$m$への単射が存在しない。
}{
  \lemref{単射と要素数}の対偶である。
}

\lem*{
  有限集合$X$について、全単射$f \colon n \to X$が構成できる自然数$n \in \N$は一意に定まる。
}{
  \lemref{全射と要素数}、\lemref{単射と要素数}より、背理法より示される。
}

\dfn{要素数}{
  有限集合$X$について、\mlemref{0}より主張される自然数を要素数$\abs{X}$と呼ぶ。

  また、集合$X$が有限集合であることを主張するときは、アリティ1の述語記号$\abs{X} < \infty$を用いる。
}

\lem{単射と有限}{
  自然数$n$と集合$A$について、以下の\num{2}つは同値である。
  \begin{enumerate}
    \item $A$は有限集合であり、$\abs{A} \leq n$
    \item 単射$f \colon A \to n$が存在する。
  \end{enumerate}
}{
  $1. \to 2.$を示す。

  全単射$f_1 \colon A \to \abs{A}$が存在して、単射$f_2 \colon \abs{A} \to n$が存在する。

  ゆえに、単射$f_2 \circ f_1 \colon A \to n$が存在する。

  \vskip\baselineskip

  $2. \to 1.$を示す。

  $n = 0$のとき、$\Hom_{\mathbf{Set}}(A, n) \neq \varnothing$より、$A = \varnothing$である。

  ある自然数$n$について成り立つとする。

  $f \colon A \to s(n)$が存在するとする。

  $\forall a \in A \qty(f(a) \neq n)$のとき、以下で定める単射$\tilde{f} \colon A \to n$が存在する。
  \eq*{
    \tilde{f}(m) \coloneqq f(m)
  }

  よって$\abs{A} \leq n \leq s(n)$である。

  $\exists a \in A \qty(f(a) = n)$のとき、単射$\tilde{f} \colon A \setminus \qty{a} \to n$が構成できる。

  仮定より$A \setminus \qty{a}$は有限集合であり、$\abs{A \setminus \qty{a}} \leq n$である。

  ゆえに全単射$g \colon A \setminus \qty{a} \to \abs{A \setminus \qty{a}}$が存在する。

  ここで、以下のような全単射$h \colon A \to s\qty(\abs{A \setminus \qty{a}})$を構成できる。
  \eq*{
    h(x) =
    \begin{cases}
      \abs{A \setminus \qty{a}} & \qty(x = a) \\*
      g(f(x)) & \qty(x \neq a)
    \end{cases}
  }

  $\abs{A} = s\qty(\abs{A \setminus \qty{a}}) \leq s(n)$を得るので成り立つ。
}


\lsubsection{有限集合}

\lem{有限集合の部分}{
  有限集合$X$について、$X$の部分$Y$は有限であり、$\abs{Y} \leq \abs{X}$である。
}{
  単射$f \colon Y \to X, f(y) = y$と、$n \in \N$で全単射$g \colon X \to n$が存在する。

  ゆえに、単射$g \circ f \colon Y \to n$が存在して、\lemref{単射と有限}より成り立つ。
}

\lem{有限集合の像}{
  有限集合$X$、集合$Y$、写像$f \colon X \to Y$について、像$f(X)$は有限であり、$\abs{f(X)} \leq \abs{X}$である。
}{
  \lemref{全射}より、$f \colon X \to f(X)$は右逆写像すなわち単射$f^{-1} \colon f(X) \to X$を持つ。

  $n \in \N$と全単射$g \colon X \to n$が存在するので、単射$g \circ f^{-1} \colon f(X) \to n$が存在して、\lemref{単射と有限}より成り立つ。
}

\lem{有限集合の商}{
  有限集合$X$と、$X$上の同値関係$\sim$について、商集合$X / \sim$は有限であり、$\abs{X / \sim} \leq \abs{X}$である。
}{
  商写像$\pi$について、$X / \sim = \pi(X)$である。
  \lemref{有限集合の像}より成り立つ。
}

\lem{有限集合の和}{
  有限集合$X, Y$について、$X \cap Y = \varnothing$ならば、$X \cup Y$は有限集合となり、以下が成り立つ。
  \eq*{
    \abs{X \cup Y} = \abs{X} + \abs{Y}
  }
}{
  $X, Y$は有限より、全単射$f \colon \abs{X} \to X, f^\prime \colon \abs{Y} \to Y$が存在する。

  ゆえに、以下のような全単射$g \colon \abs{X} + \abs{Y} \to X \cup Y$を構成できる。
  \eq*{
    g(x) =
    \begin{cases}
      f(x) & \qty(x < \abs{X}) \\*
      f^\prime(x - \abs{X}) & \qty(\abs{X} \leq x < \abs{X} + \abs{Y})
    \end{cases}
  }
}

\lem{有限集合の和と共通部分}{
  有限集合$X, Y$について、$X \cup Y, X \cap Y$は有限集合となり、以下が成り立つ。
  \eq*{
    \abs{X \cup Y} + \abs{X \cap Y} = \abs{X} + \abs{Y}
  }
}{
  単射$f \colon X \cap Y \to X$が構成できるので、\lemref{単射と有限}より$X \cap Y$は有限集合である。

  \lemref{有限集合の和}より、$\abs{X} = \abs{X \cap Y} + \abs{X \setminus Y}$

  \lemref{有限集合の和}より、$\abs{X \cup Y} = \abs{\qty(X \setminus Y) \cup Y} = \abs{X \setminus Y} + \abs{Y}$

  したがって、$\abs{X \cup Y} + \abs{X \cap Y} + \abs{X \setminus Y} = \abs{X \setminus Y} + \abs{Y} + \abs{X}$

  \dfnref{自然数の減法}より成り立つ。
}

\lem{有限集合の直積}{
  有限集合$X, Y$について、$X \times Y$は有限集合となり、以下が成り立つ。
  \eq*{
    \abs{X \times Y} = \abs{X} \times \abs{Y}
  }
}{
  全単射$x \colon \abs{X} \to X$が存在する。

	$\forall n \in \abs{X}$について、$Z_n \coloneqq \qty{\qty(x(n), y) \mid y \in Y}$を考えると、$\abs{Z_n} = \abs{Y}$である。

	$X \times Y = \bigsqcup \qty{Z_n \mid n \in \abs{X}}$が成り立つ。

	\lemref{有限集合の和}より、$\abs{X}$についての帰納法を用いて、$\abs{X \times Y} = \abs{X} \times \abs{Y}$
}

\thm{有限有向集合の上界}{
  空でない有向集合の有限部分$A$は、上界を持つ。
}{
  $\abs{A} = 1$のとき、$A$は唯一つの元を持ち、上界である。

  $\abs{A} = 2$のとき、$a_0 \neq a_1$を用いて$A = \qty{a_0, a_1}$と書ける。有向集合であることより上界が存在する。

  $\abs{A} = n$のとき成り立つとする。

  $\abs{A} = s(n)$について、$A$のある元$a$を考える。集合$A \setminus \qty{a}$は要素数が$n$であるため、仮定より上界$a_n$を持つ。

  ゆえに、今、集合$\qty{a, a_n}$は上界$a_{s(n)}$を持ち、推移律から$A$の上界となる。

  要素数$\abs{A}$について、\thmref{数学的帰納法}より示される。
}

\thm{有限全順序集合の最大元}{
  全順序集合の空でない有限部分$A$は、最大元と最小元を持つ。
}{
  $\abs{A} = 1$のとき、$A$は唯一つの元を持ち、最大元かつ最小元である。

  $\abs{A} = 2$のとき、$a_0 \neq a_1$を用いて$A = \qty{a_0, a_1}$と書ける。全順序性より$a_0 > a_1 \lor a_0 < a_1$である。

  $a_0 > a_1$のとき、最大元$a_0$、最小元$a_1$である。$a_0 < a_1$のとき、最大元$a_1$、最小元$a_0$である。

  $\abs{A} = n$のとき成り立つとする。

  $\abs{A} = s(n)$について、$A$のある元$a$を考える。集合$A \setminus \qty{a}$は要素数が$n$であるため、仮定より最大元$a_{\text{max}}$と最小元$a_{\text{min}}$を持つ。

  ゆえに、$A$は最大元$\max \qty{a, a_{\text{max}}}$を持ち、$A$は最小元$\min \qty{a, a_{\text{min}}}$を持つ。

  要素数$\abs{A}$について、\thmref{数学的帰納法}より示される。
}

\thm{自然数の上界と有限}{
  $\N$の部分$A$について、$A$が上に有界ならば$A$は有限である。
}{
  上界$n$について考える。

  $n = 0$のとき、$A \neq \varnothing$ならば全射が存在しないので、示される。

  ある$n$で成り立つとき、$A$が有限であることを考える。

  $s(n)$が上界で、$n$が上界でないとき、$s(n) \in A$である。

  仮定より、$\exists m \in \N$で、全単射$f_0 \colon m \to A \setminus \qty{s(n)}$が存在する。

  したがって、以下のような全単射$f \colon s(m) \to A$を構成できる。
  \eq*{
    f(x) =
    \begin{cases}
      f_0(x) & \qty(x \neq m) \\*
      s(n) & \qty(x = m)
    \end{cases}
  }

  \thmref{数学的帰納法}より、任意の自然数$n$について成り立つ。
}


\lsubsection{濃度}

\dfn{可算}{
  集合$X$について、$X$が空集合または全射$f \colon \N \to X$が存在するとき、$X$を可算集合と呼ぶ。
  または、単に$X$は可算であると呼ぶ。

  また、集合$X$が可算集合であることを主張するときは、アリティ1の述語記号$\abs{X} \leq \aleph_0$を用いる。
}

\cor*{
  可算集合の部分集合は可算集合である。
}

\lem*{
  集合$X$について、全単射$\sigma \colon \P(X) \to 2^X$が存在する。
}{
  $x \in X$について、以下のように$\sigma$を定める。
  \eq*{
    \sigma(A)(x) =
    \begin{cases}
      1 & \qty(x \in A) \\*
      0 & \qty(x \notin A)
    \end{cases}
  }

  以下で定める$\sigma^{-1}$は逆写像である。
  \eq*{
    \sigma^{-1}(f) = \qty{x \in X \mid f(x) = 1}
  }

  \lemref{単射}、\lemref{全射}より全単射である。
}

\lem*{
  集合$X$と$n \in \N_{\geq 1}$について、全単射$\sigma \colon X^n \times X \to X^{s(n)}$が存在する。

  また、全単射$\sigma^\prime \colon X \to X^1$が存在する。
}{
  以下で定める$\sigma$は全単射である。
  \eq*{
    \sigma(\qty(x_m)_{m \in n}, x)(l) =
    \begin{cases}
      x_l & \qty(l \in n) \\*
      x & \qty(l = n)
    \end{cases}
  }

  以下で定める$\sigma^\prime$は全単射である。
  \eq*{
    \sigma(x)(0) = x
  }
}

\lem*{
  集合$X, Y, Z$について、自然な全単射$\sigma \colon \qty(X \times Y) \times Z \to X \times \qty(Y \times Z)$が存在する。
}{
  $\sigma \qty(\qty(x, y), z) = \qty(x, \qty(y, z))$は全単射である。
}

\thm{無限集合は可算部分を含む}{
  有限でない集合$X$について、単射$a \colon \N \to X$が存在する。
}{
  集合$M \coloneqq \qty{X \setminus F \mid F \subset X \land \abs{F} < \infty}$を考える。

  $X$は有限でないので、$\varnothing \notin M$である。

  \thmref{選択関数の存在}より定まる写像$c \colon M \to X, c(A) \in A$を考える。

  \lemref{有限集合の像}より、以下の写像$G \colon \bigcup \qty{X^n \mid n \in \N} \to X$を考える。
  \eq*{
    G(h) \coloneqq c\qty(X \setminus \Im(h))
  }

  \thmref{超限再帰}より、$f \colon \N \to X, f(n) = c\qty(X \setminus \Im(f \circ \iota_{n, \N}))$が存在する。

  $n < m$とすると、$f(m) = c(X \setminus \Im(f \circ \iota_{m, \N})) \in X \setminus \Im(f \circ \iota_{m, \N})$より、$f(n) \neq f(m)$である。

  ゆえに、$f$は単射である。
}

\lem*{
  有限でない順序数$\alpha$について、単射$\alpha \times \alpha \to \alpha$が存在する。
}{
  $A \coloneqq \qty{\beta \in \alpha^+ \mid \text{単射$\alpha \to \beta$}}$を考える。

  $\alpha \in A$より、$A$は空でないので、$\min A$が存在する。
  これを$\alpha_0$とする。

  \vskip\baselineskip

  $\alpha_0 \times \alpha_0 = \bigsqcup \qty{\qty(\qty{\gamma} \times \gamma^+) \cup \qty(\gamma^+ \times \qty{\gamma}) \mid \gamma \in \alpha_0}$が成り立つ。

  順序数$\gamma$について、以下で定める組$\qty\Big(\qty(\qty{\gamma} \times \gamma^+) \cup \qty(\gamma^+ \times \qty{\gamma}), \leq)$は整列順序である。
  \eq*{
    \qty(\delta_0, \delta_1) \leq \qty(\eta_0, \eta_1) \defiff \delta_0 < \eta_0 \lor \qty(\delta_0 = \eta_0 \land \delta_1 \leq \eta_1)
  }

  \lemref{整列集合は順序型を持つ}より、整列集合$\qty(\qty(\qty{\gamma} \times \gamma^+) \cup \qty(\gamma^+ \times \qty{\gamma}), \leq)$と順序同型な順序数$\beta_\gamma$が一意に存在する。

  \vskip\baselineskip

  以下で定める$G \colon \bigcup \qty{\alpha_0^\xi \mid \xi \in \alpha_0} \to \alpha_0$を考える。
  \eq*{
    G(h) \coloneqq
    \begin{cases}
      \varnothing & \qty(\dom(h) = \varnothing) \\*
      h(\bigcup \dom(h)) + \beta_{\dom(h)} & \qty(\bigcup \dom(h) \neq \dom(h)) \\*
      \sup{h(\gamma) \mid \gamma \in \dom(h)} & \otherwise
    \end{cases}
  }

  ???
}

\thm{無限の直積}{
  有限でない集合$X$について、全単射$X \times X \to X$が存在する。
}{
  ???
}




全単射$\varphi \colon \N \times \N \to \N, \varphi \qty(n, m) = \qty(\qty(n + m) \times \qty(n + m + 1)) \divisionsymbol 2 + n$が存在する。

ただし、$a \divisionsymbol b$を除算の商を表すものとする。




\lsection{環上の加群}

\lsubsection{自由加群}

\dfn{線型包}{
  左$R$-加群$V$について、$V$の部分集合$W$を考える。このとき、以下で定める集合$\Span W$は$V$の部分加群である。
  \eq*{
    \Span W \coloneqq \qty{v \in V \mid \exists n \in \N \exists \qty(r_m, w_m)_{m \in \N} \in \qty(R \times W)^\N \qty(v = \sum_{m \in n} r_m w_m)}
  }

  これを線型包と呼ぶ。
}

\dfn{生成系}{
  左$R$-加群$V$について、$V$の部分集合$W$が以下を満たすとき、$W$を$V$の生成系と呼ぶ。
  \eq*{
    V = \Span W
  }
}

\dfn{一次独立}{
  左$R$-加群$V$と、$V$の部分集合$W$が以下を満たすとき、$W$は一次独立であると呼ぶ。
  \eq*{
    \forall n \in \N \forall \qty(r_m, w_m)_{m \in \N} \in \qty(R \times W)^\N \qty(\sum_{m \in n} r_m w_m = 0_V \rightarrow \forall m \in n \qty(r_m = 0_R))
  }
}

\lem{一次独立と一意性}{
  左$R$-加群$V$と、$V$の一次独立な部分集合$W$について、以下が成り立つ。

  $\exists n_1, n_2 \in \N \exists \qty(r_{1, m}, w_{1, m})_{m \in \N}, \qty(r_{2, m'}, w_{2, m'})_{m' \in \N} \in \qty(R \times W)^\N$について、
  \eq*{
    \sum_{m \in n_1} r_{1, m} w_{1, m} = \sum_{m' \in n_2} r_{2, m'} w_{2, m'} \rightarrow \forall m \in n_1 \qty(r_{1, m} = 0_R \lor \exists m' \in n_2 \qty(r_{1, m} = r_{2, m'} \land w_{1, m} = w_{2, m'}))
  }
}{
  $0_V = \sum_{m \in n_1} r_{1, m} w_{1, m} = \sum_{m' \in n_2} r_{2, m'} w_{2, m'}$より、一次独立の定義から成り立つ。
}

\dfn{基底}{
  左$R$-加群$V$と$V$の部分集合$B$について、$B$が一次独立かつ$V$の生成系であるとき、$B$を$V$の基底と呼ぶ。
}

\dfn{自由加群}{
  基底を持つ左$R$-加群を自由加群と呼ぶ。
}

\thm{基底の存在}{
  体$F$について、$\qty{0_V}$でない$F$-加群$V$は自由加群である。
}{
  $\mathcal{S} \coloneqq \qty{S \in \P(V) \mid \text{$S$は一次独立}}$について$\qty(\mathcal{S}, \subset)$は半順序集合である。

  $\exists v \in V \setminus \qty{0_V}$について、$\exists r \in F \qty(r v = 0_V)$と仮定すると、$F$は体であることより$v = r^{-1} 0_V = 0_V$となり反する。

  ゆえに、$\qty{v} \in \mathcal{S}$\\*

  $\mathcal{S}$の全順序部分$T$を考える。

  $\forall n \in \N \forall \qty(x_m)_{m \in n} \in \qty(\bigcup T)^n$について、$\forall m \in n \exists S_m \in T \qty(x_m \in S_m)$

  $\qty{S_m \mid m \in n}$は全順序な有限集合であるので、\thmref{有限全順序集合の最大元}より最大元を持つ。

  ゆえに、$\bigcup T$は一次独立すなわち、$\bigcup T \in \mathcal{S}$

  したがって$\mathcal{S}$は帰納的である。

  \thmref{Zornの補題}より、極大元$S_0$が存在する。\\*

  $S_0$が生成系でないと仮定すると、一次独立な新たな元がとれるので、極大性に反する。
}

\lem{次元}{
  体$F$について、$F$-加群$V$を考える。$V$の任意の基底$B_1, B_2$について、全単射$f \in B_2^{B_1}$が存在する。
}{
  $P \coloneqq \qty{\varphi \in \bigcup \qty{B_2^D \mid D \in \P(B_1) \land B_1 \cap B_2 \subset D} \mid \varphi \rvert_{B_1 \cap B_2} = \id{B_1 \cap B_2} \land C(\varphi) \coloneqq \dom(\varphi) \cup \qty(B_2 \setminus \Im(\varphi)) \text{は一次独立}}$を考える。

  ここで$p \in B_2^{B_1 \cap B_2}$について、$B_2$が基底であることから$p \in P$である。

  $P$上の半順序$\preccurlyeq$を以下のように定義する。
  \eq*{
    \varphi \preccurlyeq \psi \defiff \dom(\varphi) \subset \dom(\psi) \land \psi \rvert_{\dom(\varphi)} = \varphi
  }

  $P$の全順序部分$Q$を考える。

  自明な単射$q \in B_2^{\bigcup \qty{\dom(\varphi) \mid \varphi \in Q}}$が存在する。

  $C(q)$が一次独立でないとすると、$\exists C' \in \P(C(q)) \qty(\abs{C'} < \infty \land C' \text{は一次独立でない})$

  $\forall c \in C' \exists \varphi_c \in Q \qty(c \in \dom(\varphi_c) \cup B_2)$であり、\thmref{有限全順序集合の最大元}より、$\qty{\varphi_c \mid c \in C'} \subset Q$は最大元$\varphi_q$を持つ。

  ここで、$\Im(\varphi_q) \subset \Im(q)$であるので、$C' \subset C(\varphi_q)$であり、$C(\varphi_q)$が一次独立であることに反する。

  ゆえに$q$は上界であり、すなわち$P$は帰納的である。

  \thmref{Zornの補題}より、極大元$\sigma$が存在する。\\*

  $\exists b' \in B_2 \setminus \Im(\sigma)$であるとする。このとき$b' \notin B_1$より、$b' \notin \dom(\sigma)$である。

  $H \coloneqq \Span(C(\sigma) \setminus \qty{b'})$を考える。一次独立性から$b' \notin H$

  ここで$B_1$は基底より、$\exists n \in \N \exists \qty(r_m, b_m)_{m \in \N} \in \qty(R \times B_1)^\N \qty(b' = \sum_{m \in n} r_m b_m)$と書ける。

  $b' \notin H$より、$\exists m \in n \qty(b_m \notin H)$であり、$b_m \notin \dom(\sigma)$である。

  今、以下を満たす写像$\tau \in B_2^{\dom(\sigma) \cup \qty{b_m}}$を考える。
  \eq*{
    \tau \rvert_{\dom(\sigma)} = \sigma \land \tau(b_m) = b'
  }

  $b' \notin \Im(\sigma)$より単射。$F$は体より、$C(\tau) = \qty(C(\sigma) \setminus \qty{b'}) \cup \qty{b_m}$は一次独立である。

  ゆえに$\tau \in P$であり$\sigma \prec \tau$より、$\sigma$の極大性に反する。

  背理法より$\sigma$は全射である。
  \thmref{全射と右逆写像}より、単射な右逆写像$s \in B_1^{B_2}$が存在する。\\*

  同様に、単射$t \in B_2^{B_1}$が存在する。

  \thmref{Bernsteinの定理}より、全単射$f \in B_2^{B_1}$が存在する。
}

\dfn{次元}{
  体$F$について、$F$-加群$V$について考える。

  \thmref{基底の存在}より、$V$は基底$B$を持つ。$B$が有限であるとき、\lemref{次元}より、$V$の基底の要素数は一意に定まる。
  この要素数を、$V$の次元と呼び、$\dim V$と表す。
}


\lsubsection{加群の双対}

\lem*{
  可換環$R$と、$R$-加群$\qty(V, \rho)$について、以下で定める順序対$\qty(\Hom_R(V, R), \rho')$は$R$-加群である。
  \eq*{
    \forall f \in \Hom_R(V, R) \forall r \in R \forall v \in V \qty(\rho'(r)(f) (v) \coloneqq f \circ \rho(r)(v))
  }
}{
  \thmref{可換群上の準同型全体}より、$\Hom_R(V, R)$は可換群である。

  $\rho$の環準同型性より、以下が成り立つ。
  \eqa*{
    \rho'(r + s)(f) (v)
    &= f \circ \rho(r + s)(v) \\*
    &= f \qty(\rho(r)(v) + \rho(s)(v)) \\*
    &= f \circ \rho(r) (v) + f \circ \rho(s) (v) \\*
    &= \qty(\rho'(r)(f) + \rho'(r)(g)) (v)
  }

  \eqa*{
    \rho'(s r)(f) (v)
    &= \rho'(r s)(f) (v) \\*
    &= f \circ \rho(r s)(v) \\*
    &= f \circ \rho(r) \circ \rho(s) (v) \\*
    &= \rho'(r) (f) \circ \rho(s) (v) \\*
    &= \rho'(s) \circ \rho'(r) (f) (v)
  }
}

\dfn{双対加群}{
  \mlemref{0}より定まる加群を$V$の双対と呼び、$V^*$で表す。
}

\dfn{標準双線型式}{
  $R$-加群$V$について、以下で定まる写像$\braket{} \in R^{V^* \times V}$を標準双線形式と呼ぶ。
  \eq*{
    \forall \qty(f, x) \in V^* \times V \qty(\braket{f}{x} \coloneqq f(x))
  }
}

\rem{双対加群の補足}{
  $R$-加群$V$は以下を満たす。$\forall x, y \in V \forall f, g \in V^* \forall r \in R$とする。
  \begin{enumerate}
    \item $f$の線型性:$\braket{f}{x + y} = \braket{f}{x} + \braket{f}{y}$
    \item $f$の線型性:$\braket{f}{r x} = r \braket{f}{x}$
    \item \thmref{可換群上の準同型全体}より:$\braket{f + g}{x} = \braket{f}{x} + \braket{g}{x}$
    \item 双対加群の定義:$\braket{r f}{x} = r \braket{f}{x}$
  \end{enumerate}
}

\dfn{双対写像}{
  $R$-加群$V, W$を考える。線型写像$u \in W^V$について、以下で定まる写像$u^* \in \qty(V^*)^{W^*}$を双対写像と呼ぶ。
  \eq*{
    \forall f \in W^* \qty(u^*(f) \coloneqq f \circ u)
  }

  ゆえに、$\braket{f}{u(v)} = \braket{u^*(f)}{v}$が直ちに求まる。
}


\lsubsection{加群の積}

\dfn{直積加群}{
  環$R$と、写像$V \in Z^\Lambda$を考える。$\forall \lambda \in \Lambda$について、$V(\lambda)$は左$R$-加群であるとする。

  以下で定める順序対$\qty(\qty(\prod V, +), \rho)$は左加群である。
  \eqg*{
    \forall v_1, v_2 \in \prod V \forall \lambda \in \Lambda \qty(\qty(v_1 + v_2)(\lambda) \coloneqq v_1(\lambda) + v_2(\lambda)) \\*
    \forall v \in \prod V \forall r \in R \forall \lambda \in \Lambda \qty(\qty(r v)(\lambda) \coloneqq r v(\lambda))
  }

  この左$R$-加群を$V$の直積加群または直積と呼ぶ。
}

\dfn{直和加群}{
  環$R$と、写像$V \in Z^\Lambda$を考える。$\forall \lambda \in \Lambda$について、$V(\lambda)$は左$R$-加群であるとする。

  以下で定める順序対$\qty(\qty(\bigoplus V, +), \rho)$は左加群である。
  \eq*{
    \bigoplus V \coloneqq \qty{v \in \prod V \mid \abs{\qty{\lambda \in \Lambda \mid v(\lambda) \neq 0_{V(\lambda)}}} < \infty}
  }

  この左$R$-加群を$V$の直積加群または直和と呼ぶ。
}

\cor*{
  直和$\bigoplus V$は、直積$\prod V$の部分加群である。
}

\cor*{
  $V \in Z^\Lambda$を考える。$\forall \lambda \in \Lambda$について、$V(\lambda)$は左$R$-加群であるとする。

  $\abs{\Lambda} < \infty$ならば$\bigoplus V = \prod V$である。
}

\lem*{
  可換環$R$と、$R$-加群$V_1, V_2$、写像$A \in \qty{R}^{V_1 \times V_2}$を考える。

  以下で定める単射$\varphi \in \qty(\bigoplus A)^{V_1 \times V_2}$を考える。
  \eq*{
    \varphi \qty(x, y) \qty(x', y') =
    \begin{cases}
      1_R & \qty(x = x' \land y = y') \\*
      0_R & \qty(\otherwise)
    \end{cases}
  }

  さらに以下で定める$B$は、$\bigoplus A$の部分加群である。
  \eqg*{
    B_1 \coloneqq \qty{\varphi \qty(x_1 + x_2, y) - \varphi \qty(x_1, y) - \varphi \qty(x_2, y) \mid x_1, x_2 \in V_1, y \in V_2} \\*
    B_2 \coloneqq \qty{\varphi \qty(x, y_1 + y_2) - \varphi \qty(x, y_1) - \varphi \qty(x, y_2) \mid x \in V_1, y_1, y_2 \in V_2} \\*
    B_3 \coloneqq \qty{r \varphi \qty(x, y) - \varphi \qty(r x, y) \mid r \in R, x \in V_1, y \in V_2} \\*
    B_4 \coloneqq \qty{r \varphi \qty(x, y) - \varphi \qty(x, r y) \mid r \in R, x \in V_1, y \in V_2} \\*
    B \coloneqq \Span(\bigcup \qty{B_1, B_2, B_3, B_4})
  }
}{
  $\bigcup \qty{B_1, B_2, B_3, B_4} \subset \bigoplus A$より明らか。
}

\dfn{テンソル積}{
  可換環$R$と、$R$-加群$V_1, V_2$、写像$A \in \qty{R}^{V_1 \times V_2}$を考える。

  \mlemref{0}の主張する$R$-加群$B$について、剰余加群$\qty(\bigoplus A) / B$をテンソル積と呼び、$V_1 \bigotimes V_2$と表す。

  さらに\mlemref{0}の定める$\varphi$について、$\qty[\varphi \qty(x, y)]$を$x \otimes y$と表す。
}

\rem{テンソル積の補足}{
  可換環$R$と、$R$-加群$V_1, V_2$は以下を満たす。$\forall x, x_1, x_2 \in V_1 \forall y, y_1, y_2 \in V_2 \forall r \in R$とする。
  \begin{enumerate}
    \item $\qty(x_1 + x_2) \otimes y = x_1 \otimes y + x_2 \otimes y$
    \item $x \otimes \qty(y_1 + y_2) = x \otimes y_1 + x \otimes y_2$
    \item $r \qty(x \otimes y) = \qty(r x) \otimes y = x \otimes \qty(r y)$
  \end{enumerate}
}


\begin{comment}

\dfn{テンソル空間}{}

\lsubsection{外積代数}

\dfn{外積}{}

\dfn{行列式}{}

\dfn{準双線型写像}{}

\dfn{Hermite}{}

\end{comment}

\lsection{記号論理の補足}

\lsubsection{意味論}

\dfn{恒真命題}{
	あらゆる解釈において真を与える命題を、恒真命題と呼ぶ。
}

\dfn{意味論的同値}{
  命題$\phi, \psi$について、あらゆる解釈においてその真偽が一致するとき、$\phi$と$\psi$は意味論的に同値であると言う。
}

\dfn{論理的帰結}{
  命題$\phi, \psi$について、$\phi$が真であるあらゆる解釈において$\psi$が真であるとき、$\phi \models \psi$で表す。

  特に$\psi$が恒真命題であるとき、$\models \psi$で表す。
}

\dfn{健全性}{
  ある推論体系(推論規則と命題結合子の解釈)を考える。

  あらゆる命題$\phi$について、この推論体系が以下を満たすとき、その推論体系は健全であるという。
  \eq*{
    \vdash \phi \text{ から、 } \models \phi
  }
}

\dfn{完全性}{
  ある推論体系(推論規則と命題結合子の解釈)を考える。

  あらゆる命題$\phi$について、この推論体系が以下を満たすとき、その推論体系は完全であるという。
  \eq*{
    \models \phi \text{ から、 } \vdash \phi
  }
}


% \thm{健全性定理}{
%   \subsecref{命題結合子と推論規則}、\subsecref{述語論理}により得られる推論は健全である。
% }{
%   $\phi \land \psi$は、$\phi$と$\psi$がともに真であるときに真であるとして、そうでないときは偽とする。

%   $\phi \rightarrow \psi$は、$\phi$が真で$\psi$が偽であるときに偽であるとして、そうでないときは真とする。

%   $\forall x \qty(\phi(x))$は、議論領域上のあらゆる$x$について$\phi(x)$が真であるときに真であるとして、そうでないときは偽とする。

%   このとき各推論規則は、真の前提から真の帰結を与える。
% }
\thmf{G\"{o}delの不完全性定理}{}

\lsection{論理学}

\rem{簡単のための記号の導入}{
  並列$,$を断りなく用いる。

  $()$で定める優先順序を断りなく用いる。
}


\lsubsection{命題と命題結合子}

\dfn{命題}{
  真偽が確定している言明を、命題と呼ぶ。
}

\dfn{否定}{
  命題$\phi$について、$\lnot \phi$は命題である。

  $\lnot \phi$を、「$\phi$でない」と呼ぶ。

  \vskip\baselineskip

  以下の約束をする。
  \begin{itemize}
    \item $\phi$が真であるとき、$\lnot \phi$は偽である。
    \item $\phi$が偽であるとき、$\lnot \phi$は真である。
  \end{itemize}
}

\dfn{含意}{
  命題$\phi, \psi$について、$\phi \rightarrow \psi$は命題である。

  $\phi \rightarrow \psi$を、「$\phi$ならば$\psi$」、「$\phi$は$\psi$に十分」、「$\psi$は$\phi$に必要」と呼ぶ。

  \vskip\baselineskip

  以下の約束をする。
  \begin{itemize}
    \item $\phi$が偽であるとき、$\phi \rightarrow \psi$は真である。
    \item $\psi$が真であるとき、$\phi \rightarrow \psi$は真である。
    \item $\phi$が真であり、$\psi$が偽であるとき、$\phi \rightarrow \psi$は偽である。
  \end{itemize}

  このとき、$\phi$を前件、$\psi$を後件と呼ぶ。
}

\rem{命題の定義記号}{
  定義記号$\defiff$を導入する。

  これは左辺を用いて表現された命題は、全てその左辺と一致する部分を右辺に置き換えて理解するという意味である。
}

\dfn{左含意}{
  命題$\phi, \psi$について、$\psi \leftarrow \phi$は、以下で定める命題である。
  \eq*{
    \psi \leftarrow \phi \defiff \phi \rightarrow \psi
  }

  同様に、「$\phi$ならば$\psi$」、「$\phi$は$\psi$に十分」、「$\psi$は$\phi$に必要」と呼ぶ。
}

\dfn{連言}{
  命題$\phi, \psi$について、$\phi \land \psi$は、以下で定める命題である。
  \eq*{
    \phi \land \psi \defiff \qty(\lnot \phi) \rightarrow \psi
  }

  $\phi \land \psi$を、「$\phi$かつ$\psi$」と呼ぶ。
}

\dfn{選言}{
  命題$\phi, \psi$について、$\phi \lor \psi$は、以下で定める命題である。
  \eq*{
    \phi \lor \psi \defiff \lnot \qty(\qty(\lnot \phi) \land \qty(\lnot \psi))
  }

  $\phi \lor \psi$を、「$\phi$または$\psi$」と呼ぶ。
}

\dfn{同値}{
  命題$\phi, \psi$について、$\phi \leftrightarrow \psi$は、以下で定める命題である。
  \eq*{
    \phi \leftrightarrow \psi \defiff \qty(\phi \rightarrow \psi) \land \qty(\psi \rightarrow \phi)
  }

  $\phi \leftrightarrow \psi$を、「$\phi$と$\psi$は同値」、「$\phi$と$\psi$は必要十分」と呼ぶ。
}

\dfn{命題結合子}{
  否定$\lnot$、連言$\land$、選言$\lor$、含意$\rightarrow$、同値$\leftrightarrow$を、総称して命題結合子と呼ぶ。
}

\rem{命題結合子の優先順序}{
  特に$()$で定められていなければ、以下に示す順序で優先されるものとする。
  \begin{enumerate}
    \item 否定$\lnot$
    \item 連言$\land$
    \item 選言$\lor$
    \item 含意$\rightarrow$
    \item 同値$\leftrightarrow$
  \end{enumerate}
}


\lsubsection{恒真命題}

\dfn{恒真命題}{
	命題結合子により結合されている命題$\psi$を考える。
	結合されている各命題の真偽によらず、$\psi$が真となる命題を、恒真命題と呼ぶ。
}

\dfn{矛盾}{
  偽である命題を矛盾と呼び、$\bot$で表す。
}

\cor*{
  命題$\phi, \psi, \chi$について、以下に示す全ては恒真命題である。
  \begin{enumerate}
    \item 同一律
    \begin{itemize}
      \item $\phi \rightarrow \phi$
    \end{itemize}
    \item 冪等律
    \begin{itemize}
      \item $\phi \leftrightarrow \phi \land \phi$
      \item $\phi \leftrightarrow \phi \lor \phi$
    \end{itemize}
    \item {\L}ukasiewiczの第一公理
    \begin{itemize}
      \item $\phi \rightarrow \qty(\psi \rightarrow \phi)$
    \end{itemize}
    \item 結合律
    \begin{itemize}
      \item $\phi \land \qty(\psi \land \chi) \leftrightarrow \qty(\phi \land \psi) \land \chi$
      \item $\phi \lor \qty(\psi \lor \chi) \leftrightarrow \qty(\phi \lor \psi) \lor \chi$
    \end{itemize}
    \item 交換律
    \begin{itemize}
      \item $\phi \land \psi \leftrightarrow \psi \land \phi$
      \item $\phi \lor \psi \leftrightarrow \psi \lor \phi$
    \end{itemize}
    \item 吸収律
    \begin{itemize}
      \item $\phi \lor \qty(\phi \land \psi) \leftrightarrow \phi$
      \item $\phi \land \qty(\phi \lor \psi) \leftrightarrow \phi$
    \end{itemize}
    \item 分配律
    \begin{itemize}
      \item $\phi \lor \qty(\psi \land \chi) \leftrightarrow \qty(\phi \lor \psi) \land \qty(\phi \lor \chi)$
      \item $\phi \land \qty(\psi \lor \chi) \leftrightarrow \qty(\phi \land \psi) \lor \qty(\phi \land \chi)$
    \end{itemize}
    \item 推移律
    \begin{itemize}
      \item $\qty(\phi \rightarrow \psi) \land \qty(\psi \rightarrow \chi) \rightarrow \qty(\phi \rightarrow \chi)$
    \end{itemize}
    \item De Morganの法則
    \begin{itemize}
      \item $\lnot \phi \lor \lnot \psi \leftrightarrow \lnot \qty(\phi \land \psi)$
      \item $\lnot \phi \land \lnot \psi \leftrightarrow \lnot \qty(\phi \lor \psi)$
    \end{itemize}
    \item 無矛盾律
    \begin{itemize}
      \item $\lnot \qty(\phi \land \lnot \phi)$
    \end{itemize}
    \item 不条理則
    \begin{itemize}
      \item $\bot \rightarrow \phi$
    \end{itemize}
    \item 二重否定除去
    \begin{itemize}
      \item $\lnot \lnot \phi \leftrightarrow \phi$
    \end{itemize}
    \item 排中律
    \begin{itemize}
      \item $\phi \lor \lnot \phi$
    \end{itemize}
    \item Peirce則
    \begin{itemize}
      \item $\qty(\qty(\phi \rightarrow \psi) \rightarrow \phi) \rightarrow \phi$
    \end{itemize}
    \item 背理法
    \begin{itemize}
      \item $\qty(\phi \land \lnot \psi \rightarrow \bot) \rightarrow \qty(\phi \rightarrow \psi)$
    \end{itemize}
    \item 対偶法
    \begin{itemize}
      \item $\qty(\phi \rightarrow \psi) \leftrightarrow \qty(\lnot \psi \rightarrow \lnot \phi)$
    \end{itemize}
    \item 選言三段論法
    \begin{itemize}
      \item $\qty(\phi \lor \psi) \land \lnot \phi \rightarrow \psi$
    \end{itemize}
  \end{enumerate}

  当然これらは代表的な例であり、全てではない。
}


\lsubsection{推論と推論規則}

\rem{古典論理}{
  以降、古典論理(NK)の推論規則を導入する。
}

\dfn{推論}{
  命題$\phi, \psi$について、$\phi$から$\psi$が推論されるとき、$\phi \vdash \psi$で明示的に表す。
}

\axi{連言の導入則}{
  命題$\phi, \psi, \chi$について、以下を定める。
  \eq*{
    \qty(\qty(\chi \vdash \phi), \qty(\chi \vdash \psi)) \vdash \qty(\chi \vdash \phi \land \psi)
  }
}

\axi{連言の除去則}{
  命題$\phi, \psi$について、以下の\num{2}つを定める。
  \eqg*{
    \phi \land \psi \vdash \phi \\*
    \phi \land \psi \vdash \psi
  }
}

\axi{選言の導入則}{
  命題$\phi, \psi$について、以下の\num{2}つを定める。
  \eqg*{
    \phi \vdash \phi \lor \psi \\*
    \psi \vdash \phi \lor \psi
  }
}

\axi{選言の除去則}{
  命題$\phi, \psi, \chi$について、以下を定める。
  \eq*{
    \qty(\qty(\phi \vdash \chi), \qty(\psi \vdash \chi)) \vdash \qty(\phi \lor \psi \vdash \chi)
  }
}

\axi{含意の導入則(演繹定理)}{
  命題$\phi, \psi, \chi$について、以下を定める。
  \eq*{
    \qty(\chi \land \phi \vdash \psi) \vdash \qty(\chi \vdash \qty(\phi \rightarrow \psi))
  }
}

\axi{含意の除去則(Modus Ponens)}{
  命題$\phi, \psi$について、以下を定める。
  \eq*{
    \phi \land \qty(\phi \rightarrow \psi) \vdash \psi
  }

  含意の導入、除去則より、演繹と含意を似たような意味で扱うことができる。
}

\axi{否定の導入則}{
  命題$\phi, \psi$について、以下を定める。
  \eq*{
    \qty(\phi \land \psi \vdash \bot) \vdash \qty(\phi \vdash \lnot \psi)
  }
}

\axi{否定の除去則(二重否定の除去)}{
  命題$\phi$について、以下を定める。
  \eq*{
    \lnot \lnot \phi \vdash \phi
  }
}


\lsubsection{述語論理}

\dfn{述語}{
  一つないし複数の変項を持ち、その変項の全てが確定したときに命題となる言明を、述語と呼ぶ。

  述語を表す記号である述語記号$\phi$と変項$x$を用いて$\phi(x)$で表す。

  \vskip\baselineskip

  述語記号が要求する変項の数を、その述語記号のアリティと呼ぶ。
}

\rem{変項の定義記号}{
  定義記号$\coloneqq$を導入する。

  これは左辺を用いて表現された変項は、全てその左辺と一致する部分を右辺に置き換えて理解するという意味である。
}

\dfn{全称量化子}{
  述語$\phi(x)$について、$\forall x \qty(\phi(x))$は命題である。
}

\axi{全称量化子の導入則}{
  変項$x$を出現させない述語$\psi$と述語記号$\phi$について、以下を定める。
  \eq*{
    \qty(\psi \vdash \phi(x)) \vdash \qty(\psi \vdash \forall x \qty(\phi(x)))
  }
}

\axi{全称量化子の除去則}{
  述語記号$\phi$について、以下を定める。
  \eq*{
    \forall x \qty(\phi(x)) \vdash \phi(y)
  }
}

\dfn{存在量化子}{
  述語$\phi(x)$について、$\exists x \qty(\phi(x))$は命題である。
}

\axi{存在量化子の導入則}{
  述語記号$\phi$について、以下を定める。
  \eq*{
    \phi(y) \vdash \exists x \qty(\phi(x))
  }
}

\axi{存在量化子の除去則}{
  述語記号$\phi$、変項$y$を出現させない述語$\psi$、命題$\chi$について、以下を定める。
  \eq*{
    \qty(\chi \land \phi(y) \vdash \psi) \vdash \qty(\chi \land \exists x \qty(\phi(x)) \vdash \psi)
  }
}

\cor{量化子と連言}{
  述語記号$\phi$について、以下が成り立つ。
  \eqg*{
    \forall x \qty(\phi(x) \land \psi(x)) \leftrightarrow \forall x \qty(\phi(x)) \land \forall x \qty(\psi(x)) \\*
    \exists x \qty(\phi(x) \land \psi(x)) \rightarrow \exists x \qty(\phi(x)) \land \exists x \qty(\psi(x))
  }
}

\cor{量化子と選言}{
  述語記号$\phi$について、以下が成り立つ。
  \eqg*{
    \forall x \qty(\phi(x) \lor \psi(x)) \leftarrow \forall x \qty(\phi(x)) \lor \forall x \qty(\psi(x)) \\*
    \exists x \qty(\phi(x) \lor \psi(x)) \leftrightarrow \exists x \qty(\phi(x)) \lor \exists x \qty(\psi(x))
  }
}

\cor{量化子と否定}{
  述語記号$\phi$について、以下が成り立つ。
  \eqg*{
    \forall x \qty(\lnot \phi(x)) \leftrightarrow \lnot \exists x \qty(\phi(x)) \\*
    \exists x \qty(\lnot \phi(x)) \leftrightarrow \lnot \forall x \qty(\phi(x))
  }
}

\dfn{等号}{
  アリティ\num{2}の述語記号$=$を定める。

  $=\qty(x, y)$を、簡単のために$x = y$で表す。
}

\axi{等号の導入則(反射律)}{
  変項$x$について、以下を定める。
  \eq*{
    x = x
  }
}

\axi{等号の除去則(代入原理)}{
  述語記号$\phi$、変項$x, y$について、以下を定める。
  \eq*{
    \phi(x) \land x = y \vdash \phi(y)
  }
}

\lem{等号の対称律}{
  変項$x, y$について、以下が成り立つ。
  \eq*{
    x = y \vdash y = x
  }
}{
  $x = x \land x = y$

  左命題の左辺に、右命題から得る代入を行って、$y = x$を得る。
}

\cor{等号の推移律}{
  変項$x, y, z$について、以下が成り立つ。
  \eq*{
    x = y \land y = z \vdash x = z
  }
}

\dfn{等号否定}{
  略記$\neq$を以下のように定める。
  \eq*{
    a \neq b \defiff \lnot \qty(a = b)
  }
}

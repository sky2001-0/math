\lsection{代数上の順序}

\lsubsection{順序群}

\dfn{順序群}{
  非自明な可換群$\qty(G, +)$上の全順序$\leq$が以下を満たすとき、順序対$\qty(\qty(G, +), \leq)$を順序群と呼ぶ。または単に$G$と書いて順序群を表す。
  \eq*{
    \forall a, b, c \in G \qty(a \leq b \rightarrow a + c \leq b + c)
  }
}

\cor*{
  順序群$G$について、以下が成り立つ。
  \eq*{
    \forall a, b, c \in G \qty(a < b \rightarrow a + c < b + c)
  }
}

\lem*{
  順序群$G$は最大元と最小元を持たない。
}{
  非自明性から$\exists a \in G \qty(a \neq 0_G)$であり、全順序性から$0_G < a \lor a < 0_G$である。

  $0_G = (-a) + a < 0_G + a = a$であるとき、\dfnref{順序群}より$-a < 0_G$である。

  任意の元$b \in G$について、$b < b + a$より最大元ではなく、$b + (-a) < b$より最小元でもない。

  $a < 0_G$についても同様である。
}

\dfn{正錐}{
  順序群$G$について、以下の集合を$G^+$で表す。
  \eq*{
    G^+ \coloneqq \qty{g \in G \mid 0_G < g}
  }
}

\dfnf{\textit{Archimedes}の原理}{Archimedesの原理}{
  順序群$G$が以下を満たすとき、$G$は\textit{Archimedes}的であると呼ぶ。
  \eq*{
    \forall x, y \in G^+ \exists n \in \Z^+ \qty(x < n y)
  }

  ただし$n y$の表記は、\thmref{Z-加群としての可換群}による。
}

\lem{順序群の三角性}{
  順序群$G$について、$G^+$が最小元を持たないとき以下が成り立つ。
  \eq*{
    \forall g \in G^+ \exists g' \in G^+ \qty(g' + g' \leq g)
  }
}{
  最小元を持たないので、$\exists g_1 \in G^+ \qty(g_1 < g)$である。

  $g_1 + g_1 \leq g$のとき、$g' = g_1$として明らか。

  $g_1 + g_1 > g$のとき、$g' \coloneqq g - g_1$について、$g' + g' = g + g - g_1 - g_1 < g$
}


\lsubsection{順序環}

\dfn{順序環}{
  順序群$\qty(\qty(R, +), \leq)$について、順序対$\qty(\qty(R, +), \times)$が可換環でありかつ以下を満たすとき、順序対$\qty(\qty(\qty(R, +), \times), \leq)$を順序環と呼ぶ。
  \eq*{
    \forall a, b \in R \qty(0_R < a \land 0_R < b \rightarrow 0_R < a \times b)
  }
}

\lem{順序環の性質}{
  順序環$R$について、以下が成り立つ。
  \eqg*{
    \forall a \in R \qty(a \neq 0_R \rightarrow 0_R < a a) \\*
    0_R < 1_R \\*
    \forall a \in R^\times \qty(0_R < a \rightarrow 0_R < a^{-1})
  }
}{
  $0_R < a$のとき順序環の定義より明らか。$a < 0_R$のとき$0_R < -a$であり、\lemref{環の性質}より$a a = (-a)(-a)$であるので示される。\\*

  零環でないので、$0_R \neq 1_R$である。すでに示した第一式より$0_R < 1_R 1_R = 1_R$\\*

  $a^{-1} \leq 0_R$とすると、$0_R \leq a (-a^{-1}) = -1_R$ゆえに$1_R \leq 0_R$となりすでに示した第二式に矛盾。
}

\lem{順序環の三角性}{
  順序環$R$について、以下が成り立つ。
  \eq*{
    \forall r \in R^+ \exists r' \in R^+ \qty(r' \times r' \leq r)
  }
}{
  $1_R \leq r$のとき、$r' = 1_R$として成り立つ。

  $0_R < r < 1_R$のとき、$r r < r$より、$r' = r$として成り立つ。
}

\thm{順序環は整域}{
  順序環は整域である。
}{
  整域でないとすると、$\exists a, b \in R \qty(a b = 0_R \land a \neq 0_R \land b \neq 0_R)$

  $r \coloneqq a, b \in R$について、$0_R < r \lor r < 0_R$で場合分けすることにより矛盾を得る。背理法より示される。
}

\dfn{順序体}{
  順序環が体であるとき、順序体と呼ぶ。
}

\thm{順序体の稠密性}{
  順序体$F$について、
  \eq*{
    \forall a, b \in F \qty(a < b \rightarrow \exists c \in F \qty(a < c \land c < b))
  }
}{
  $a, b \in F$として、$0_F < 1_F + 1_F$より、
  \eq*{
    c \coloneqq \qty(a + b) \times \qty(1_F + 1_F)^{-1}
  }
  は条件を満たす。
}
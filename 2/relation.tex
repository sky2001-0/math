\lsection{関係}

\lsubsection{自己関係}

\dfn{自己関係}{
  関係$\qty(\qty(X, Y), G)$について、$X = Y$であるとき、$X$上の自己関係、または単に$X$上の関係と呼ぶ。
}

\dfn{反射的}{
  集合$X$上の関係$\mathfrak{R}$が反射的であるとは、以下を満たすことである。
  \eq*{
    \forall x \in X \qty(\mathfrak{R}(x, x))
  }
}

\dfn{対称的}{
  集合$X$上の関係$\mathfrak{R}$が対称的であるとは、以下を満たすことである。
  \eq*{
    \forall x, y \in X\qty(\mathfrak{R}(x, y) \rightarrow \mathfrak{R}(y, x))
  }
}

\dfn{反対称的}{
  集合$X$上の関係$\mathfrak{R}$が反対称的であるとは、以下を満たすことである。
  \eq*{
    \forall x, y \in X \qty(\mathfrak{R}(x, y) \land \mathfrak{R}(y, x) \rightarrow x = y)
  }
}

\dfn{推移的}{
  集合$X$上の関係$\mathfrak{R}$が推移的であるとは、以下を満たすことである。
  \eq*{
    \forall x, y, z \in X \qty(\mathfrak{R}(x, y) \land \mathfrak{R}(y, z) \rightarrow \mathfrak{R}(x, z))
  }
}

\cor*{
  $\varnothing$上の関係$\mathfrak{R}$は、反射的、対称的、反対称的、推移的である。
}


\lsubsection{前順序}

\dfn{前順序}{
  反射的かつ推移的な自己関係$\preccurlyeq \coloneqq \qty(\qty(X, X), G)$を、前順序と呼ぶ。

  前順序であることを明示的に記号$\preccurlyeq$で表す。

  \vskip\baselineskip

  簡単のため、$\preccurlyeq \qty(x, y)$を$x \preccurlyeq y$でも表す。

  略記$\prec, \succcurlyeq, \succ$を以下のように定める。
  \eqg*{
    x \prec y \defiff x \preccurlyeq y \land x \neq y \\*
    x \succcurlyeq y \defiff y \preccurlyeq x \\*
    x \succ y \defiff y \prec x
  }

  順序対$\qty(X, \preccurlyeq)$を、前順序集合、または単に前順序と呼ぶ。
}

\rem{前順序の定義}{
  より簡単に、$x \preccurlyeq y \defiff \cdots$の形式によって前順序を定義する。
}

\cor*{
  前順序集合$X$について、以下が成り立つ。
  \eqg*{
    \forall x, y \in X \qty(x \preccurlyeq y \leftrightarrow x \prec y \lor x = y) \\*
    \forall x, y \in X \qty(x \succcurlyeq y \leftrightarrow x \succ y \lor x = y)
  }
}

\dfn{上界}{
  前順序集合$X$について、その部分集合$A$を考える。

  $A$が上に有界であるとは、以下を満たすことである。
  \eq*{
    \exists b \in X \forall a \in A \qty(a \preccurlyeq b)
  }

  このとき、$b$を$A$の上界と呼ぶ。
}

\dfn{下界}{
  前順序集合$X$について、その部分集合$A$を考える。

  $A$が下に有界であるとは、以下を満たすことである。
  \eq*{
    \exists b \in X \forall a \in A \qty(b \preccurlyeq a)
  }

  このとき、$b$を$A$の下界と呼ぶ。
}

\dfn{有界}{
  前順序集合$X$について、その部分集合$A$を考える。

  $A$が有界であるとは、$A$が上に有界かつ下に有界であることである。
}

\cor*{
  空でない前順序集合$X$について、$\varnothing$は有界である。
}

\dfn{有向集合}{
  前順序集合$\Lambda$について、任意の二元$\lambda, \mu \in \Lambda$から成る集合$\qty{\lambda, \mu}$が上に有界であるとき、前順序集合$\Lambda$を有向集合と呼ぶ。
}

\dfn{上方集合}{
  有向集合$\Lambda$とその元$\lambda_0 \in \Lambda$について、集合$\Lambda_{\succcurlyeq \lambda_0}, \Lambda_{\succ \lambda_0}$を以下で定める。
  \eqg*{
    \Lambda_{\succcurlyeq \lambda_0} \coloneqq \qty{\lambda \in \Lambda \mid \lambda_0 \preccurlyeq \lambda} \\*
    \Lambda_{\succ \lambda_0} \coloneqq \qty{\lambda \in \Lambda \mid \lambda_0 \prec \lambda}
  }
}

\dfn{前順序との両立}{
  集合$X$と$X$上の前順序$\preccurlyeq_{X}$、集合$Y$と$Y$上の前順序$\preccurlyeq_{Y}$について、写像$f \colon X \to Y$が$\preccurlyeq_{X}, \preccurlyeq_{Y}$と両立するとは、以下を満たすことである。
  \eq*{
    \forall x, y \in X \qty(x \preccurlyeq_{X} y \rightarrow f(x) \preccurlyeq_{Y} f(y))
  }
}


\lsubsection{同値関係}

\dfn{同値関係}{
  対称的な前順序を、同値関係と呼ぶ。
  同値関係であることを明示的に記号$\sim$で表す。
}

\cor*{
  等号は同値関係である。
  集合$X$上の同値関係であることを明示的に、$=_X$で表す。
}

\dfn{同値類}{
  空でない集合$X$上で定義された同値関係$\sim$と、$X$の要素$a$について、以下で定める集合を$a$の同値類と呼び、$\qty[a]$で表す。
  この$a$を特に代表元と呼ぶ。
  \eq*{
    \qty[a] \coloneqq \qty{x \in X \mid x \sim a}
  }
}

\cor*{
  空でない集合$X$上で定義された同値関係$\sim$について、以下が成り立つ。
  \eq*{
    X = \bigsqcup \qty{\qty[x] \mid x \in X}
  }
}

\dfn{商集合}{
  同値関係の定義された空でない集合$X$について、以下で定める集合を商集合と呼び、$X / \sim$で表す。
  \eq*{
    X / \sim \coloneqq \qty{\qty[x] \mid x \in X}
  }
}

\dfn{商写像}{
  写像$\qty[] \colon X \to X / \sim$を商写像と呼ぶ。
}

\cor*{
  商写像は全射である。
}

\dfn{写像に付随する同値関係}{
  写像$f \colon X \to Y$について、以下で定める$X$上の関係$\sim_{f}$は同値関係であり、これを写像$f$に付随する同値関係と呼ぶ。
  \eq*{
    x \sim_f y \defiff f(x) = f(y)
  }
}

\cor*{
  写像$f \colon X \to Y$について、$f$は$\sim_f, =_Y$と両立する。
}

\thm{両立}{
  空でない集合$X$上の同値関係$\sim_{X}$、集合$Y$上の同値関係$\sim_{Y}$について、
  写像$f \colon X \to Y$が$\sim_{X}, \sim_{Y}$と両立することは、写像$h \colon X / \sim_{X} \to Y / \sim_{Y}$が存在して以下を満たすことと必要十分である。
  \eq*{
    \qty[]_{Y} \circ f = h \circ \qty[]_{X}
  }

  さらに、上で定まる写像$h$は一意である。
}{
  $\qty[]_{X}$は全射より、右逆写像$r$が存在する。

  \vskip\baselineskip

  まず、必要性を示す。
  仮定より以下を満たす。
  \eq*{
    \forall x \in X / \sim_{X} \forall z, w \in x \qty(\qty[f(z)]_{Y} = \qty[f(w)]_{Y})
  }

  ゆえに$h \coloneqq \qty[]_{Y} \circ f \circ r$を考えると、$\qty[]_{Y} \circ f = h \circ \qty[]_{X}$を満たす。

  \vskip\baselineskip

  十分性を示す。
  $\forall x, y \in X $について、$x \sim_{X} y$ならば、$\qty[f(x)]_{Y} = h(\qty[x]_{X}) = h(\qty[y]_{X}) = \qty[f(y)]_{Y}$である。
  よって、$f(x) \sim_{Y} f(y)$

  \vskip\baselineskip

  一意であることを示す。
  相異なる\num{2}つが存在すると仮定すると、$h \circ \qty[]_{X} = h' \circ \qty[]_{X}$である。
  右から$r$をかけて、$h = h'$であり、矛盾する。
}

\thm{標準分解}{
  写像$f \colon X \to Y$について、全単射$\bar{f} \colon X / \sim_{f} \to \Im(f)$が存在する。
}{
  \thmref{両立}より、写像$\bar{f} \colon X / \sim_{f} \to Y$が存在して、$\Im(\bar{f}) = \Im(f)$である。

  単射であることを示す。
  $\bar{f}([x]) = \bar{f}([y])$のとき、$f(x) = f(y)$より、$[x] = [y]$である。
}

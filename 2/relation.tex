\lsection{関係}

\lsubsection{自己関係}

\dfn{自己関係}{
  関係$\qty(\qty(X, Y), G)$について、$X = Y$であるとき、$X$上の自己関係、または単に$X$上の関係と呼ぶ。

  簡単のために、$X$上の関係と書いて$G$を表すものとする。

  また、$X$上の関係$G$について、アリティ\num{2}の述語記号$G$を以下で定める。
  \eq*{
    G\qty(x, y) \defiff \qty(x, y) \in G
  }
}

\dfn{反射的}{
  集合$X$上の関係$G$が反射的であるとは、以下を満たすことである。
  \eq*{
    \forall x \in X \qty(G(x, x))
  }
}

\dfn{対称的}{
  集合$X$上の関係$G$が対称的であるとは、以下を満たすことである。
  \eq*{
    \forall x, y \in X\qty(G(x, y) \rightarrow G(y, x))
  }
}

\dfn{反対称的}{
  集合$X$上の関係$G$が反対称的であるとは、以下を満たすことである。
  \eq*{
    \forall x, y \in X \qty(G(x, y) \land G(y, x) \rightarrow x = y)
  }
}

\dfn{推移的}{
  集合$X$上の関係$G$が推移的であるとは、以下を満たすことである。
  \eq*{
    \forall x, y, z \in X \qty(G(x, y) \land G(y, z) \rightarrow G(x, z))
  }
}

\dfn{完全}{
  集合$X$上の関係$G$が完全であるとは、以下を満たすことである。
  \eq*{
    \forall x, y \in X \qty(x \neq y \rightarrow G(x, y) \lor G(y, x))
  }
}

\cor*{
  $\varnothing$上の関係は一意に定まり、それは、反射的、対称的、反対称的、推移的、完全である。
}

\dfn{写像が誘導する自己関係}{
  写像$f \colon X \to Y$と、$Y$上の自己関係$G_Y$を考える。

  このとき、以下で定める$X$上の自己関係$G_X$を考える。
  \eq*{
    G_X(x, y) \defiff G_Y(f(x), f(y))
  }

  この$G_X$を、$f$が$G_Y$から誘導する自己関係と呼ぶ。
}

\rem{包含写像が誘導する自己関係}{
  集合$X$上の自己関係$G$について、特に明示しなければ$X$の部分$A$の自己関係を、包含写像が誘導する自己関係で定めるものとする。
}

\cor{写像が誘導する自己関係}{
  写像$f \colon X \to Y$と、$Y$上の自己関係$G_Y$を考える。

  $f$が$G_Y$から誘導する自己関係$G_X$を考える。

  このとき、以下が成り立つ。
  \begin{enumerate}
    \item $G_Y$が反射的ならば、$G_X$は反射的である。
    \item $G_Y$が対称的ならば、$G_X$は対称的である。
    \item $f$が単射かつ$G_Y$が反対称的ならば、$G_X$は反対称的である。
    \item $G_Y$が推移的ならば、$G_X$は推移的である。
    \item $G_Y$が完全ならば、$G_X$は完全である。
  \end{enumerate}
}


\lsubsection{前順序}

\dfn{前順序}{
  集合$X$上の反射的かつ推移的な自己関係$G$を、前順序と呼ぶ。

  前順序であることを明示的に記号$\preccurlyeq$で表す。

  \vskip\baselineskip

  簡単のため、$\preccurlyeq \qty(x, y)$を$x \preccurlyeq y$でも表す。

  組$\qty(X, \preccurlyeq)$を、前順序集合、または単に前順序と呼ぶ。
}

\rem{前順序の定義}{
  より簡単に、$x \preccurlyeq y \defiff \cdots$の形式によって述語として前順序を定義できる。
}

\dfn{前順序に関わるいくつかの述語}{
  アリティ\num{2}の述語$\prec, \succcurlyeq, \succ$を以下のように定める。
  \eqg*{
    x \prec y \defiff x \preccurlyeq y \land x \neq y \\*
    x \succcurlyeq y \defiff y \preccurlyeq x \\*
    x \succ y \defiff y \prec x
  }
}

\cor*{
  前順序集合$X$について、以下が成り立つ。
  \eqg*{
    \forall x, y \in X \qty(x \preccurlyeq y \leftrightarrow x \prec y \lor x = y) \\*
    \forall x, y \in X \qty(x \succcurlyeq y \leftrightarrow x \succ y \lor x = y)
  }
}

\dfn{上界}{
  前順序集合$X$について、その部分集合$A$を考える。

  $A$が上に有界であるとは、以下を満たすことである。
  \eq*{
    \exists b \in X \forall a \in A \qty(a \preccurlyeq b)
  }

  このとき、$b$を$A$の上界と呼ぶ。
}

\dfn{下界}{
  前順序集合$X$について、その部分集合$A$を考える。

  $A$が下に有界であるとは、以下を満たすことである。
  \eq*{
    \exists b \in X \forall a \in A \qty(b \preccurlyeq a)
  }

  このとき、$b$を$A$の下界と呼ぶ。
}

\dfn{有界}{
  前順序集合$X$について、その部分集合$A$を考える。

  $A$が有界であるとは、$A$が上に有界かつ下に有界であることである。
}

\cor*{
  空でない前順序集合$X$について、$\varnothing$は有界である。
}

\dfn{有向集合}{
  前順序集合$\Lambda$が以下を満たすとき、$\Lambda$を有向集合と呼ぶ。
  \eq*{
    \forall \lambda, \mu \in \Lambda \qty(\text{$\qty{\lambda, \mu}$ は上に有界})
  }
}

\dfn{共終}{
  前順序集合$X$について、$X$の部分$A$が以下を満たすとき、$A$を共終な部分と呼ぶ。
  \eq*{
    \forall x \in X \exists a \in A \qty(x \preccurlyeq a)
  }
}

\dfn{上方集合}{
  前順序集合$X$とその元$x_0 \in X$について、集合$X_{\succcurlyeq x_0}, X_{\succ x_0}$を以下で定める。
  \eqg*{
    X_{\succcurlyeq x_0} \coloneqq \qty{x \in X \mid x_0 \preccurlyeq x} \\*
    X_{\succ x_0} \coloneqq \qty{x \in X \mid x_0 \prec x}
  }
}

\dfn{単調}{
  前順序集合$\qty(X, \preccurlyeq_{X}), \qty(Y, \preccurlyeq_{Y})$と、写像$f \colon X \to Y$について、以下を満たすとき、$f$は単調であるという。
  \eq*{
    \forall x, y \in X \qty(x \preccurlyeq_{X} y \rightarrow f(x) \preccurlyeq_{Y} f(y))
  }
}

\cor*{
  単調な$f \colon X \to Y, g \colon Y \to Z$について、$g \circ f$は単調である。
}

\dfn{狭義単調}{
  前順序集合$\qty(X, \preccurlyeq_{X}), \qty(Y, \preccurlyeq_{Y})$と、写像$f \colon X \to Y$について、以下を満たすとき、$f$は狭義単調であるという。
  \eq*{
    \forall x, y \in X \qty(x \prec_{X} y \rightarrow f(x) \prec_{Y} f(y))
  }
}

\cor*{
  狭義単調な写像は、単調である。
}


\lsubsection{前順序と圏論}

\dfn{前順序集合の圏}{
  単調写像を射とみなすと、単調写像であることは圏となる。

  この圏を前順序集合の圏と呼び、$\mathbf{Ord}$で表す。
}

\dfn{順序同型}{
  $\mathbf{Ord}$の同型射を順序同型と呼ぶ。
}

\cor*{
  順序同型ならば狭義単調である。
}

\dfn{前順序が誘導する圏}{
  前順序集合$\qty(X, \preccurlyeq)$について、$\preccurlyeq$の元$\qty(x, y)$を考える。
  $\dom(\qty(x, y)) = \qty(x, x), \cod(\qty(x, y)) = \qty(y, y)$として、合成を$\qty(y, z) \circ \qty(x, y) = \qty(x, z)$で定義すると、
  $G$は圏となる。

  このように定める圏$G$を、前順序$X$が誘導する圏と呼ぶ。

  誤解のない範囲で、前順序$X$が誘導する圏を前順序集合$X$で表す。

  ここで記号の濫用であるが、$\qty(x, x)$をして$x$を表し、$x$をして$\qty(x, x)$を表す。
}

\cor*{
  前順序が誘導する圏$X$について、以下が成り立つ。
  \eq*{
    \forall x, y \in X \forall f, g \in \Hom_X(x, y) \qty(f = g)
  }
}

\dfn{最大元}{
  前順序集合$X$について、$X$の終対象が存在するならば、この終対象$m$を$X$の最大元と呼ぶ。

  すなわち、以下である。
  \eq*{
    \forall x \in X \qty(x \preccurlyeq m)
  }
}

\dfn{最小元}{
  前順序集合$X$について、$X$の始対象が存在するならば、この始対象$m$を$X$の最小元と呼ぶ。

  すなわち、以下である。
  \eq*{
    \forall x \in X \qty(m \preccurlyeq x)
  }
}

\dfn{下限}{
  前順序集合$X$について、その部分集合$A$を考える。

  離散圏$A$から前順序が誘導する圏$X$への関手$\id$について、この積$x, u$が存在するならば、$x$を$A$の下限と呼ぶ。

  すなわち、以下である。
  \eq*{
    \forall a \in A \qty(x \preccurlyeq a) \land \forall x^\prime \in X \qty(\forall a \in A \qty(x^\prime \preccurlyeq a) \rightarrow x^\prime \preccurlyeq x)
  }
}

\dfn{上限}{
  前順序集合$X$について、その部分集合$A$を考える。

  離散圏$A$から前順序が誘導する圏$X$への関手$\id$について、この和$x, u$が存在するならば、$x$を$A$の上限と呼ぶ。

  すなわち、以下である。
  \eq*{
    \forall a \in A \qty(a \preccurlyeq x) \land \forall x^\prime \in X \qty(\forall a \in A \qty(a \preccurlyeq x^\prime) \rightarrow x \preccurlyeq x^\prime)
  }
}

\lem{単調像と上限}{
  前順序集合$X, Y$について、単調写像$f \colon X \to Y$を考える。

  $X$の部分$A$について、$A$の上限$x$と、$f(A)$の上限$y$が存在するとする。

  このとき、以下が成り立つ。
  \eq*{
    y \preccurlyeq f(x)
  }
}{
  $x$は上限より、$\forall a \in A \qty(a \preccurlyeq x)$であり、単調性より$\forall a \in A \qty(f(a) \preccurlyeq f(x))$である。

  ゆえに、$f(x)$は$f(A)$の上界であるので、$f(x) \preccurlyeq \sup f(A) = y$である。
}

\lem{単調像と下限}{
  前順序集合$X, Y$について、単調写像$f \colon X \to Y$を考える。

  $X$の部分$A$について、$A$の下限$x$と、$f(A)$の下限$y$が存在するとする。

  このとき、以下が成り立つ。
  \eq*{
    f(x) \preccurlyeq y
  }
}{
  \lemref{単調像と上限}と同様に示せる。
}

\dfn{射影系}{
  前順序集合$\Lambda$と圏$\bm{C}$について、$\Lambda^\opp$から$\bm{C}$への関手$X$を、射影系と呼ぶ。
}

\dfn{射影極限}{
  射影系$X$ついて、$X$の極限を射影極限と呼ぶ。
}

\dfn{帰納極限}{
  射影系$X$ついて、$X$の余極限を帰納極限と呼ぶ。
}


\lsubsection{同値関係}

\dfn{同値関係}{
  対称的な前順序を、同値関係と呼ぶ。
  同値関係であることを明示的に記号$\sim$で表す。
}

\cor*{
  等号は同値関係である。
  集合$X$上の同値関係であることを明示的に、$=_X$で表す。
}

\dfn{写像が誘導する同値関係}{
  写像$f \colon X \to Y$について、$f$が$=_Y$から誘導する$X$上の自己関係は同値関係である。

  この同値関係を、写像$f$が誘導する$X$上の同値関係と呼ぶ。
}

\cor*{
  写像$f \colon X \to Y$について、$f$が誘導する$X$上の同値関係$\sim_f$を考える。

  $\sim_f$について、$\sim_f, \qty(\pi_0, \pi_1)$は$f$の核対である。

  ただし、$\pi_0(\qty(x, y)) = x, \pi_1(\qty(x, y)) = y$とする。
}

\dfn{同値類}{
  集合$X$上の同値関係$\sim$と、$X$の要素$a$について、集合$X_{\sim a}$を$a$の同値類と呼ぶ。
}

\cor*{
  集合$X$上の同値関係$\sim$について、以下が成り立つ。
  \eq*{
    X = \bigsqcup \qty{X_{\sim x} \mid x \in X}
  }
}

\dfn{商集合}{
  集合$X$上の同値関係$\sim$について、以下で定める集合を商集合と呼び、$X / \sim$で表す。
  \eq*{
    X / \sim \coloneqq \qty{X_{\sim x} \mid x \in X}
  }
}

\dfn{商写像}{
  集合$X$上の同値関係$\sim$について、以下で定める写像$\pi$を$\sim$の商写像と呼ぶ。
  \eq*{
    \pi(x) \coloneqq X_{\sim x}
  }
}

\lem{集合の圏における余等化子の存在}{
  写像$f, g \colon X \to Y$について、
  積$Y \times Y, \pi$が存在するので、一意な射$q$が存在して$f = \pi(0) \circ q \land g = \pi(1) \circ q$である。

  以下の集合$\sim$を考える。
  \eq*{
    \sim \coloneqq \bigcap \qty{G^\prime \in Y \times Y \mid \Im(q) \subset G^\prime \land \text{$G^\prime$は$Y$上の同値関係}}
  }

  共通部分をとられる集合は$Y \times Y$を要素に持ち空ではないので、$\sim$は$Y$上の同値関係となる。

  $\sim$の商写像$u$について、$Y / \sim, u$は余等化子である。
}{
  $\sim$の定義より、$\forall x \in X \qty(f(x) \sim g(x))$である。
  ゆえに、$u \circ f = u \circ g$である。

  写像$h \colon Y \to Z$について、$h \circ f = h \circ g$であるとする。

  $h$が誘導する$Y$上の同値関係$\sim_h$について、$\forall x_0, x_1 \in X \qty(\qty(f(x_0), g(x_1)) \in \sim_f)$である。

  すなわち$\Im(q) \subset \sim_f$であり、$G$の定義より$G \subset \sim_f$である。

  したがって、$\forall y_0, y_1 \in Y \qty(y_0 \sim y_1 \rightarrow h(y_0) = h(y_1))$である。

  写像$k \coloneqq \qty(\qty(Y / \sim, Z), \qty{\qty(Y_{\sim y}, h(y)) \mid y \in Y})$は、$k \circ u = h$を満たす。

  $k^\prime \colon Y / \sim \to Z, k^\prime \neq k$を考えると明らかに満たさないので、$k$は一意である。
}

\begin{center}
  \begin{minipage}{0.4\linewidth}
    $\mathbf{Set}$ \\
    \begin{tikzcd}
      X \ar[dr, "q", dotted] \ar[rrd, "f", bend left] \ar[ddr, "g"', bend right] & & & \\
      & Y \times Y \ar[r, "\pi(0)"] \ar[d, "\pi(1)"'] & Y \ar[d, "u"] \ar[ddr, "h", bend left] & \\
      & Y \ar[r, "u"'] \ar[rrd, "h", bend right] & Y / \sim \ar[rd, "k", dotted] & \\
      & & & Z \\
    \end{tikzcd}
  \end{minipage}
\end{center}

\cor{商集合は余等化子}{
  集合$X$上の同値関係$\sim$、$\sim$の商写像$u$、積$X \times X, \pi$を考える。

  $X / \sim, u$は、$\pi_0, \pi_1 \colon G \to X$の余等化子である。
}

\thm{写像の標準分解}{
  写像$f \colon X \to Y$と、$\sim_f$の商写像$u$を考える。

  このとき一意な単射$g \colon X / \sim_f \to Y$が存在して、$f = g \circ u$が成り立つ。
}{
  写像$f$が誘導する同値関係$\sim_f$とする。

  このとき、$\sim_f, \qty(\pi_0, \pi_1)$は$f$の核対であり、$X / \sim_f, u$は$\pi_0, \pi_1$の余等化子である。

  余等化子の普遍性より、写像$g \colon X / \sim_f \to Y$が一意に存在して、$f = g \circ u$である。

  \vskip\baselineskip

  $g(x) = g(y)$とする。

  $u$は右簡約可能であり、\lemref{全射}より全射であるため、$\exists z, w \in X \qty(x = u(z) \land y = u(w))$である。

  今、$f(z) = g(u(z)) = g(x) = g(y) = g(u(w)) = f(w)$であるので、$z \sim_f w$である。

  したがって、$x = u(z) = X_{\sim_f z} = X_{\sim_f w} = u(w) = y$が成り立つ。

  ゆえに単射である。
}

\begin{center}
  \begin{minipage}{0.4\linewidth}
    \begin{tikzcd}
      \sim_f \ar[r, "\pi_0"] \ar[d, "\pi_1"'] & X \ar[d, "u"] \ar[ddr, "f", bend left] & \\
      X \ar[r, "u"'] \ar[rrd, "f", bend right] & X / \sim_f \ar[rd, "g", dotted] & \\
      & & Y \\
    \end{tikzcd}
  \end{minipage}
\end{center}

\lsection{写像}

\lsubsection{関係}

\dfn{関係}{
  集合$X, Y$と、集合系$G \subset X \times Y$について、順序対$\mathfrak{R} \coloneqq \qty(\qty(X, Y), G)$を関係と呼ぶ。

  \vskip\baselineskip

  記号の濫用であるが、$\mathfrak{R}$を以下で定めるアリティ\num{2}の述語としても扱う。
  \eq*{
    \mathfrak{R}(x, y) \defiff \qty(x, y) \in G
  }
}

\dfn{左一意的}{
  関係$\mathfrak{R}$が以下を満たすとき、$\mathfrak{R}$は左一意的であると言う。
  \eq*{
    \forall w, x \qty(\exists y \in Y \qty(\mathfrak{R}(w, y) \land \mathfrak{R}(x, y)) \rightarrow w = x)
  }
}

\dfn{右一意的}{
  関係$\mathfrak{R}$が以下を満たすとき、$\mathfrak{R}$は右一意的であると言う
  \eq*{
    \forall y, z \qty(\exists x \in X \qty(\mathfrak{R}(x, y) \land \mathfrak{R}(x, z)) \rightarrow y = z)
  }
}

\dfn{一対一}{
  関係$\mathfrak{R}$が左一意的かつ右一意的であるとき、$\mathfrak{R}$は一対一であると言う。
}

\dfn{左全域的}{
  関係$\mathfrak{R} = \qty(\qty(X, Y), G)$が以下を満たすとき、$\mathfrak{R}$は左全域的であると言う。
  \eq*{
    \forall x \in X \exists y \in Y \qty(\mathfrak{R}(x, y))
  }
}

\dfn{右全域的}{
  関係$\mathfrak{R} = \qty(\qty(X, Y), G)$が以下を満たすとき、$\mathfrak{R}$は右全域的であると言う。
  \eq*{
    \forall y \in Y \exists x \in X \qty(\mathfrak{R}(x, y))
  }
}


\lsubsection{写像}

\dfn{写像}{
  右一意的かつ左全域的な関係$f = \qty(\qty(X, Y), G)$を写像と呼ぶ。

  \vskip\baselineskip

  記号の濫用であるが、$f()$を以下で定める$x \in X$についての略記としても扱う。
  \eq*{
    f(x) \coloneqq \bigcup \qty{y \in Y \mid \mathfrak{R}(x, y)}
  }

  $f(x)$を$x$での$f$の値と呼ぶ。
}

\rem{関数クラスによる写像の定義}{
  集合$X, Y$と、関数クラス$F$について、
  順序対$\qty(\qty(X, Y), \qty{\qty(x, F(x)) \mid x \in X})$は写像である。

  このとき、記号の濫用であるが、この写像を$F$で表す。
}

\rem{集合の内包的定義3}{
  写像$f \colon X \to Y$について、以下を満たすアリティ\num{2}の述語記号$\psi(x, y)$を考える。
  \eq*{
    \forall x, y \qty(\psi(x, y) \defiff y = f(x))
  }

  $f$が写像であることから\axiref{置換の公理図式}の主張する集合$A$が存在する。

  $A$は、より簡潔に以下でも表す。
  \eq*{
    A = \qty{f(x) \mid x \in X}
  }
}

\dfn{恒等写像}{
  集合$X$について、以下で定める集合系$\Delta$を考える。
  \eq*{
    \Delta \coloneqq \qty{\qty(x, x) \mid x \in X}
  }

  写像$\qty(\qty(X, X), \Delta)$を恒等写像と呼び、$\id_X$で表す。
}

\cor*{
  集合$X, Y$について、以下が成り立つ。
  \eq*{
    X = Y \leftrightarrow \id_X = \id_Y
  }
}

\dfn{合成写像}{
  写像$f = \qty(\qty(X, Y), G_f), g = \qty(\qty(Y, Z), G_g)$について、以下で定める集合系$G$を考える。
  \eq*{
    G \coloneqq \qty{\qty(x, z) \mid \exists y \in Y \qty(y = f(x) \land z = g(y))}
  }

  関係$\qty(\qty(X, Z), G)$は写像であり、これを$f$と$g$の合成写像、または単に合成と呼び、$g \circ f$で表す。
}

\cor{写像の合成の結合法則}{
  写像$f = \qty(\qty(X, Y), G_f), g = \qty(\qty(Y, Z), G_g), h = \qty(\qty(Z, W), G_h)$について、以下が成り立つ。
  \eq*{
    \qty(h \circ g) \circ f = h \circ \qty(g \circ f)
  }
}

\rem{写像の相等}{
  写像の相等は、集合の相等により定義される。

  これは次のように言い換えることができる。
  写像$f, g$について、$f = g$とは、以下を満たすことである。
  \eq*{
    \dom(f) = \dom(g) \land \cod(f) = \cod(g) \land \forall x \in \dom(f) \qty(f(x) = g(x))
  }

  終域まで含めて相等とすることに注意されたい。
}

\rem{写像は射}{
  写像$f = \qty(\qty(X, Y), G)$を考える。
  $\dom(f)$を$\id_X$、$\cod(f)$を$\id_Y$として、合成を\dfnref{合成写像}で定義すると、写像は射の公理をみたす。

  ここで記号の濫用であるが、$\id_X$をして$X$を表し、$X$をして$\id_X$を表す。
}

\dfn{集合と写像の圏}{
  写像を射とみなすと、写像であることは、圏となる。
  この圏を、集合と写像の圏と呼び、$\mathbf{Set}$と表す。
}

\cor*{
  集合$X, Y$について、$\Hom_{\mathbf{Set}}(X, Y)$は以下を満たす。
  \eq*{
    f \in \Hom_{\mathbf{Set}}(X, Y) \leftrightarrow f \in \qty{\qty(\qty(X, Y), G) \mid G \subset X \times Y \land \text{$G$は右一意かつ左全域}}
  }
}

\dfn{写像の全体}{
  記号の濫用であるが、$X$から$Y$への写像全体がなす集合を、$\Hom_{\mathbf{Set}}(X, Y)$または$Y^X$で表す。
  \eq*{
    \Hom_{\mathbf{Set}}(X, Y) \coloneqq \qty{\qty(\qty(X, Y), G) \mid G \subset X \times Y \land \text{$G$は右一意かつ左全域}}
  }
}

\cor*{
  以下が成り立つ。
  \eqg*{
    \forall Y \qty(\Hom_{\mathbf{Set}}(\varnothing, Y) = \qty{\qty(\qty(\varnothing, Y), \varnothing)}) \\*
    \forall X \qty(X \neq \varnothing \rightarrow \Hom_{\mathbf{Set}}(X, \varnothing) = \varnothing)
  }
}


\lsubsection{像と原像}

\dfn{像}{
  写像$f \colon X \to Y$と、$X$の部分集合$A$について、以下で定める集合を$A$の$f$による像と呼び、$f(A)$で表す。
  \eq*{
    f(A) \coloneqq \qty{f(x) \mid x \in A}
  }

  また、$f(X)$を$\Im(f)$でも表す。
}

\lem{像の性質}{
  写像$f \colon X \to Y$について、以下が成り立つ。
  \eqg*{
    \forall A_1, A_2 \subset X \qty(A_1 \subset A_2 \rightarrow f(A_1) \subset f(A_2)) \\*
    \forall B \subset \P(X) \qty(f\qty(\bigcup B) = \bigcup \qty{f(A) \mid A \in B}) \\*
    \forall B \subset \P(X) \qty(f\qty(\bigcap B) \subset \bigcap \qty{f(A) \mid A \in B}) \\*
    \forall A_1, A_2 \subset X \qty(f(A_1) \setminus f(A_2) \subset f(A_1 \setminus A_2))
  }
}{
  第一式を示す。

  $y \in f(A_1)$のとき、$\exists x \in A_1 \qty(y = f(x))$、$\exists x \in A_2 \qty(y = f(x))$、ゆえに$y \in f(A_2)$

  \vskip\baselineskip

  第二式を示す。
  \eqg*{
    y \in f\qty(\bigcup B) \\*
    \exists x \in \bigcup B \qty(y = f(x)) \\*
    \exists A \in B \exists x \in A \qty(y = f(x)) \\*
    \exists A \in B \qty(y \in f(A)) \\*
    y \in \bigcup \qty{f(A) \mid A \in B}
  }

  上からも下からも成り立つので、\axiref{外延性の公理}より成り立つ。

  \vskip\baselineskip

  第三式を示す。

  $y \in f\qty(\bigcap B)$を考える。

  $\exists x \qty(x \in \bigcup B \land \forall A \in B \qty(x \in A) \land y = f(x))$である。

  $\exists x \qty(x \in \bigcup B \land y = f(x))$より、$y \in f\qty(\bigcup B)$であり、第二式より$y \in \bigcup \qty{f(A) \mid A \in B}$である。

  $\exists x \qty(\forall A \in B \qty(x \in A) \land y = f(x))$より、$\forall A \in B \qty(y \in f(A))$である。

  ゆえに、$y \in \bigcap \qty{f(A) \mid A \in B}$である。

  \vskip\baselineskip

  第四式を示す。

  $y \in f(A_1) \setminus f(A_2)$を考える。

  $\exists x \qty(y = f(x) \land x \in A_1) \land \forall x \qty(y = f(x) \rightarrow x \notin A_2)$である。

  ただちに、$\exists x \qty(y = f(x) \land x \in A_1 \setminus A_2)$である。

  したがって、$y \in f \qty(A_1 \setminus A_2)$である。
}

\dfn{原像}{
  写像$f \colon X \to Y$と集合$A$について、以下で定める集合を、集合$A$の原像と呼び、$f^{-1}(A)$で表す。
  \eq*{
    f^{-1}(A) = \qty{x \in X \mid f(x) \in A}
  }
}

\cor{像と原像}{
  写像$f \colon X \to Y$について、以下が成り立つ。
  \eqg*{
    \forall A \subset X \qty(f^{-1} \qty(f(A)) \supset A) \\*
    \forall B \qty(f \qty(f^{-1}(B)) \subset B)
  }
}

\lem{原像の性質}{
  写像$f$について、以下が成り立つ。
  \eqg*{
    \forall A_1, A_2 \qty(A_1 \subset A_2 \rightarrow f^{-1}(A_1) \subset f^{-1}(A_2)) \\*
    \forall B \qty(f^{-1}\qty(\bigcup B) = \bigcup \qty{f^{-1}(A) \mid A \in B}) \\*
    \forall B \qty(f^{-1}\qty(\bigcap B) = \bigcap \qty{f^{-1}(A) \mid A \in B}) \\*
    \forall A_1, A_2 \qty(f^{-1}(A_1) \setminus f^{-1}(A_2) = f^{-1}(A_1 \setminus A_2))
  }
}{
  第一式を示す。

  $x \in f^{-1}(A_1)$について、$f(x) \in A_1 \subset A_2$ゆえに$x \in f^{-1}(A_2)$

  \vskip\baselineskip

  第二式を示す。
  \eqg*{
    x \in f^{-1}\qty(\bigcup B) \\*
    f(x) \in \bigcup B \\*
    \exists A \in B \qty(f(x) \in A) \\*
    \exists A \in B \qty(x \in f^{-1}(A)) \\*
    x \in \bigcup \qty{f^{-1}(A) \mid A \in B}
  }

  上からも下からも成り立つので、\axiref{外延性の公理}より成り立つ。

  \vskip\baselineskip

  第三式を示す。
  \eqg*{
    x \in f^{-1}\qty(\bigcap B) \\*
    f(x) \in \bigcap B \\*
    f(x) \in \bigcup B \land \forall A \in B \qty(f(x) \in A) \\*
    x \in f^{-1}\qty(\bigcup B) \land \forall A \in B \qty(x \in f^{-1}(A)) \\*
    x \in \bigcup \qty{f^{-1}(A) \mid A \in B} \land \forall A \in B \qty(x \in f^{-1}(A)) \\*
    x \in \bigcap \qty{f^{-1}(A) \mid A \in B}
  }

  \num{4}行目から\num{5}行目に第二式を用いている。

  上からも下からも成り立つので、\axiref{外延性の公理}より成り立つ。

  \vskip\baselineskip

  第四式について、
  \eqg*{
    x \in f^{-1}(A_1) \setminus f^{-1}(A_2) \\*
    f(x) \in A_1 \land f(x) \notin A_2 \\*
    f(x) \in A_1 \setminus A_2 \\*
    x \in f^{-1} \qty(A_1 \setminus A_2)
  }

  上からも下からも成り立つので、\axiref{外延性の公理}より成り立つ。
}


\lsubsection{いくつかの重要な集合と写像}

\dfn{包含写像}{
  集合$X$と、その部分$A$について、以下で定める関係$\iota_{A, X}$は写像である。
  \eq*{
    \iota_{A, X} \coloneqq \qty(\qty(A, X), \qty{\qty(x, x) \mid x \in A})
  }

  この$\iota_{A, X}$を包含写像と呼ぶ。
}

\dfn{制限写像}{
  写像$f = \qty(\qty(X, Y), G)$と、$X$の部分$A$について、
  以下で定める写像$f \rvert_A$を、$f$の$A$への制限写像、または単に制限と呼ぶ。
  \eq*{
    f \rvert_A \coloneqq f \circ \iota_{A, X}
  }
}

\cor*{
  単集合は$\mathbf{Set}$の終対象である。
}

\cor*{
  空集合は$\mathbf{Set}$の始対象である。
}

\dfn{空写像}{
  空集合は$\mathbf{Set}$の始対象であるので、終域$Y$について、始域が空集合である写像は一意に定まる。

  この写像を$Y$の空写像と呼ぶ。
}

\lem{集合の圏における等化子の存在}{
  写像$f, g \colon X \to Y$を考える。

  以下を満たす$\Eq(f, g), u$は、$f, g$の等化子である。
  \eqg*{
    \Eq(f, g) \coloneqq \qty{x \in X \mid f(x) = g(x)} \\*
    u \coloneqq \iota_{\Eq(f, g), X}
  }
}{
  $f \circ h = g \circ h$なる$h \colon Z \to X$を考える。

  $\forall z \in Z \qty(f(h(z)) = g(h(z)))$より、$\Im(h) \subset \Eq(f, g)$である。

  ゆえに$k = \qty(\qty(Z, \Eq(f, g)), \qty{\qty(z, h(z)) \mid z \in Z})$は写像である。

  $h = u \circ k$である。

  $k^\prime \colon Z \to \Eq(f, g)$について、$k^\prime \neq k$とすると直ちに$h \neq u \circ k^\prime$である。

  したがって一意である。
}

\dfn{小さい圏}{
  圏$\bm{C}$について、$\bm{C}$の射を対象とする離散圏$\hat{\bm{C}}$が集合であるとき、$\bm{C}$を小さいと呼ぶ。
}

\cor*{
  小さい圏$\bm{J}$について、$\Obj_{\bm{J}}$は小さい。
}

\cor*{
  小さい離散圏$\bm{J}$と、$\bm{J}$から$\mathbf{Set}$への関手$X$を考える。

  このとき、以下で定める$\prod X, \pi$は$X$の積である。
  \eqg*{
    \prod X \coloneqq \qty{f \in \Hom_{\mathbf{Set}}\qty(\bm{J}, \bigcup \qty{X(j) \mid j \in \bm{J}}) \mid \forall j \in \bm{J} \qty(f(j) \in X(j))} \\*
    \pi(j)(f) \coloneqq f(j)
  }
}

\cor*{
  集合$X, Y$について、以下で定める離散圏$\bm{2}$と$\bm{2}$から$\mathbf{Set}$への関手$F$を考える。
  \eqg*{
    f \in \bm{2} \leftrightarrow f = 0 \lor f = 1 \\*
    F(0) = X \land F(1) = Y
  }

  このとき、$X \times Y, \pi$は$F$の積である。

  ただし、$\pi(0)(\qty(x, y)) = x, \pi(1)(\qty(x, y)) = y$とする。
}

\cor*{
  小さい離散圏$\bm{J}$と、$\bm{J}$から$\mathbf{Set}$への関手$X$を考える。

  このとき、以下で定める$x, u$は$X$の和である。
  \eqg*{
    x \coloneqq \bigcup \qty{\qty{\qty(j, x) \mid x \in X(j)} \mid j \in \bm{J}} \\*
    u(j)(x) \coloneqq \qty(j, x)
  }
}


\lsubsection{選択}

\lem{選択公理が与える写像}{
  集合$X, Y$とアリティ\num{2}の述語記号$\psi$について、$\forall x \in X \exists y \in Y \qty(\psi(x, y))$が成り立つとする。

  このとき、以下を満たす写像$f \colon X \to Y$が存在する。
  \eq*{
    \forall x \in X \forall y \in Y \qty(f(x) = y \rightarrow \psi(x, y))
  }
}{
  $X = \varnothing$のとき、空写像が存在する。

  \vskip\baselineskip

  $X \neq \varnothing$のときを考える。
  以下の集合$Z$を考える。
  \eq*{
    Z \coloneqq \qty{\qty{\qty(x, y) \mid y \in Y \land \psi(x, y)} \mid x \in X}
  }

  $Z$について\axiref{選択の公理}の主張する集合$A$が存在する。

  関係$\qty(\qty(X, Y), A)$は、仮定より$X$について左全域的で、$A$の定義より右一意的である。
}

\thm{選択関数の存在}{
  任意の集合$X$について、写像$f \colon X \setminus \qty{\varnothing} \to \bigcup X$が存在して、以下が成り立つ。
  \eq*{
    \forall x \in X \setminus \qty{\varnothing} \qty(f(x) \in x)
  }
}{
  $\forall x \in X \setminus \qty{\varnothing}$について、$\exists y \qty(y \in x)$である。
  この$y$は、$y \in x$より、$y \in \bigcup X$である。

  ゆえに、\lemref{選択公理が与える写像}より存在する。
}

\lem{非空からなる積は非空}{
  空でない集合$J$から$\mathbf{Set}$への関手$X$について、以下が成り立つ。
  \eq*{
    \forall j \in J \qty(X(j) \neq \varnothing) \rightarrow \prod X \neq \varnothing
  }
}{
  $Z \coloneqq \qty{X(j) \mid j \in J}$について、定義より$Z \neq \varnothing \land \varnothing \notin Z$である。

  \thmref{選択関数の存在}より、写像$f \colon Z \to \bigcup Z$が存在して、$\forall z \in Z \qty(f(z) \in z)$である。

  写像$g \colon J \to \bigcup Z$を$g(j) \coloneqq f(X(j))$で定義すると、$\forall j \in J \qty(g(j) \in X(j))$であるので、$g \in \prod X$である。

  ゆえに、$\prod X \neq \varnothing$である。
}


\lsubsection{単射と全射}

\lem{単射}{
  写像$f$について、以下の\num{3}つは同値である。
  \begin{enumerate}
    \item $f$が左簡約可能
    \item $f$が左一意的(単射)
    \item $f$が、空写像または左可逆
  \end{enumerate}
}{
  $1. \to 2.$を示す。

  $f$が空写像であるとき、左一意的である。
  $f$が空写像でないときを考える。

  $x \in \dom(f)$について、写像$g_x \colon \qty{\varnothing} \to \dom(f), g(\varnothing) \coloneqq x$を考える。

  $f(x) = f(y)$であるとき、$f \circ g_x = f \circ g_y$であり、左簡約可能より$g_x = g_y$、すなわち$x = y$である。

  \vskip\baselineskip

  $2. \to 3.$を示す。
  左一意的かつ空写像でないならば、左可逆であることを示す。

  空でないので$\exists x_0 \in \dom(f)$より、以下で定める集合$G$が存在する。
  \eq*{
    G \coloneqq \qty{\qty(y, x) \in \cod(f) \times \dom(f) \mid y = f(x)} \cup \qty{\qty(y, x_0) \mid y \in \cod(f) \setminus \Im(f)}
  }

  写像$g \coloneqq \qty(\qty(\cod(f), \dom(f)), G)$は、定義より$f$の左逆写像となる。

  \vskip\baselineskip

  $3. \to 1.$を示す。

  $f$が空写像であるときを考える。

  $f$に右から合成可能、すなわち、終域を空集合とする写像は、$g_0 \colon \varnothing \to \varnothing$のみであるので、左簡約可能である。

  $f$が左可逆であるとき、左可逆ならば左簡約可能であるため、明らか。
}

\cor*{
  包含写像は単射である。
}

\lem*{
  集合$X, Y$について、単射$f \colon X \to Y$が存在するならば、単射$g \colon \P(X) \to \P(Y)$が存在する。
}{
  $g(A) \coloneqq \qty{f(x) \mid x \in A}$で定義する写像は、単射である。
}

\lem{全射}{
  写像$f$について、以下の\num{3}つは同値である。
  \begin{enumerate}
    \item $f$が右簡約可能
    \item $f$が右全域的(全射)
    \item $f$が右可逆
  \end{enumerate}
}{
  $1. \to 2.$を示す。

  右全域的でないとすると、$\exists y \in Y \forall x \in X \qty(f(x) \neq y)$である。

  $g, h \in \Map(Y, \qty{\varnothing, \qty{\varnothing}})$について、$g(y) = \varnothing, h(y) = \qty{\varnothing}$であり、
  $\forall w \in Y \setminus \qty{y} \qty(g(w) = h(w))$とする。

  $g \circ f = h \circ f$であるが、$g \neq h$であり、右簡約可能ではない。

  対偶法より示される。

  \vskip\baselineskip

  $2. \to 3.$を示す。

  \lemref{選択公理が与える写像}の与える写像は、右逆写像である。

  \vskip\baselineskip

  $3. \to 1.$は、右可逆ならば右簡約可能であるため、明らか。
}

\dfn{全単射}{
  写像$f$が同型射であるとき、$f$を全単射と呼ぶ。
}

\dfn{逆写像}{
  写像$f$の逆射を逆写像と呼ぶ。
}

\lem*{
  全単射$f \colon X \to Y$と$Y$の部分集合$A$について、原像$f^{-1}(A)$と逆写像の像$f^{-1}(A)$は一致する。
}{
  全単射より、$\forall y \in A \exists x \in X \qty(y = f(x))$である。
  定義より明らか。
}

\thm{Cantorの対角線論法}{
  単射$f \colon \P(A) \to A$は存在しない。
}{
  存在すると仮定する。

  以下の集合$Y$を考える。
  \eq*{
    Y \coloneqq \qty{f(X) \mid X \subset A \land f(X) \notin X}
  }

  このとき$Y \subset A$である。

  $f(Y) \notin Y$とすると、定義より$f(Y) \in Y$である。
  ゆえに矛盾。

  $f(Y) \in Y$とすると、$\exists Z \subset A \qty(f(Z) \notin Z \land f(Z) = f(Y))$であるが、$f$の単射性より$Y = Z$ゆえに$f(Y) \notin Y$である。
  ゆえに矛盾。

  背理法より示される。
}

\lsection{写像}

\axi{選択公理}{
  \eq*{
    \forall X \qty(\varnothing \notin X \land \forall x \forall y \qty(x \in X \land y \in X \land x \neq y \rightarrow x \cap y = \varnothing) \rightarrow \exists A \forall x \qty(x \in X \rightarrow \exists t \qty(x \cap A = \qty{t})))
  }
}


\lsubsection{写像}

\dfn{写像}{
  右一意的かつ左全域的な関係$f = \qty(\qty(X, Y), G)$を写像と呼ぶ。

  \vskip\baselineskip

  記号の濫用であるが、$f()$を以下で定める$x \in X$についての略記としても扱う。
  \eq*{
    f(x) \coloneqq \bigcup \qty{y \in Y \mid \mathfrak{R}(x, y)}
  }

  $f(x)$を$x$での$f$の値と呼ぶ。
}

\rem{写像の構成的な定義}{
  集合$X, Y$と、$X$の要素$x$から$Y$の要素$y$を対応させる規則$f$を考えると、
  順序対$\qty(\qty(X, Y), \qty{\qty(x, f(x)) \mid x \in X})$は写像である。

  このとき、記号の濫用であるが、この写像を$f$で表す。

  より簡単に、$f \colon X \to Y, f(x) \coloneqq \cdots$の形式によって写像を定義する。
}

\rem{部分写像}{
  左全域性が担保されていない写像を部分写像と呼ぶ。

  この場合においても、写像の構成的な定義と同様の定義方法が行える。
}

\rem{集合の内包的定義3}{
  写像$f \colon X \to Y$について、以下を満たすアリティ\num{2}の述語記号$\psi(x, y)$を考える。
  \eq*{
    \forall x, y \qty(\psi(x, y) \defiff y = f(x))
  }

  $f$が写像であることから\axiref{置換の公理図式}の主張する集合$A$が存在する。

  $A$は、より簡潔に以下でも表す。
  \eq*{
    A = \qty{f(x) \mid x \in X}
  }
}

\dfn{写像の定義域と終域}{
  写像$f = \qty(\qty(X, Y), G)$を考える。

  $X$を定義域と呼び、$\dom(f)$で表す。

  $Y$を終域と呼び、$\ran(f)$で表す。
}

\dfn{写像の全体}{
  集合$X, Y$について、$Y^X$で表す集合を以下で定める。
  \eq*{
    Y^X \coloneqq \qty{\qty(\qty(X, Y), G) \mid \exists G \qty(\text{$\qty(\qty(X, Y), G)$は写像})}
  }

  また$Y^X$は、$\Map(X, Y)$とも表す。
}

\cor*{
  以下が成り立つ。
  \eqg*{
    \forall Y \qty(Y^\varnothing = \qty{\qty(\qty(\varnothing, Y), \varnothing)}) \\*
    \forall X \qty(X \neq \varnothing \rightarrow \varnothing^X = \varnothing)
  }
}

\dfn{空写像}{
  定義域が空集合である写像は、一意に定まる。

  このような写像を空写像と呼ぶ。
}


\dfn{制限写像}{
  写像$f = \qty(\qty(X, Y), G)$と、$X$の部分$A$について、以下で定める集合系$G_A$を考える。
  \eq*{
    G_A \coloneqq \qty{\qty(x, y) \in G \mid x \in A}
  }

  写像$\qty(\qty(A, Y), G_A)$を、$f$の$A$への制限写像、または単に制限と呼び、$f \rvert_A$で表す。
}

\dfn{恒等写像}{
  集合$X$について、以下で定める集合系$\Delta$を考える。
  \eq*{
    \Delta \coloneqq \qty{\qty(x, x) \mid x \in X}
  }

  写像$\qty(\qty(X, X), \Delta)$を恒等写像と呼び、$\id{X}$で表す。
}

\dfn{合成写像}{
  写像$f \colon X \to Y, g \colon Y \to Z$について、以下で定める集合系$G$を考える。
  \eq*{
    G \coloneqq \qty{\qty(x, z) \mid \exists y \in Y \qty(y = f(x) \land z = g(y))}
  }

  写像$\qty(\qty(X, Z), G)$を$f$と$g$の合成写像、または単に合成と呼び、$g \circ f$で表す。
}

\cor{写像の合成の結合法則}{
  写像$f \colon X \to Y, g \colon Y \to Z, h \colon Z \to W$について、以下が成り立つ。
  \eq*{
    \qty(h \circ g) \circ f = h \circ \qty(g \circ f)
  }
}

\thm{選択公理が与える写像}{
  集合$X, Y$とアリティ\num{2}の述語記号$\psi$について、$\forall x \in X \exists y \in Y \qty(\psi(x, y))$が成り立つとする。

  このとき、以下を満たす写像$f \colon X \to Y$が存在する。
  \eq*{
    \forall x \in X \forall y \in Y \qty(f(x) = y \rightarrow \psi(x, y))
  }
}{
  $X = \varnothing$のとき、空写像が存在する。

  \vskip\baselineskip

  $X \neq \varnothing$のときを考える。
  以下の集合$Z$を考える。
  \eq*{
    Z \coloneqq \qty{\qty{\qty(x, y) \mid y \in Y \land \psi(x, y)} \mid x \in X}
  }

  $Z$について\axiref{選択公理}の主張する集合$A$が存在する。

  関係$\qty(\qty(X, Y), A)$は、仮定より$X$について左全域的で、$A$の定義より右一意的である。
}


\lsubsection{像と原像}

\dfn{像}{
  写像$f \colon X \to Y$と、$X$の部分集合$A$について、以下で定める集合を$A$の$f$による像と呼び、$f(A)$で表す。
  \eq*{
    f(A) \coloneqq \qty{f(x) \mid x \in A}
  }

  また、$f(X)$を$\Im(f)$でも表す。
}

\thm{像の性質}{
  写像$f \colon X \to Y$について、以下が成り立つ。
  \eqg*{
    \forall A_1, A_2 \subset X \qty(A_1 \subset A_2 \rightarrow f(A_1) \subset f(A_2)) \\*
    \forall B \subset \P(X) \qty(f \qty(\bigcup B) = \bigcup \qty{f(A) \mid A \in B}) \\*
    \forall B \subset \P(X) \qty(f \qty(\bigcap B) \subset \bigcap \qty{f(A) \mid A \in B}) \\*
    \forall A_1, A_2 \subset X \qty(f(A_1) \setminus f(A_2) \subset f(A_1 \setminus A_2))
  }
}{
  第一式について、
  $y \in f(A_1)$のとき、$\exists x \in A_1 \qty(y = f(x))$、$\exists x \in A_2 \qty(y = f(x))$、ゆえに$y \in f(A_2)$

  \vskip\baselineskip

  第二式について、
  \eqg*{
    y \in \bigcup \qty{f(A) \mid A \in B} \\*
    \exists A \in B \qty(y \in f(A)) \\*
    \exists A \in B \exists x \in A \qty(y = f(x)) \\*
    \exists x \in \bigcup B \qty(y = f(x)) \\*
    y \in f \qty(\bigcup B)
  }

  上からも下からも成り立つので、\axiref{外延性の公理}より成り立つ。

  \vskip\baselineskip

  第三式について、第二式を用いて、
  \eqg*{
    y \in f \qty(\bigcap B) \\*
    \exists x \qty(x \in \bigcap B \land y = f(x)) \\*
    \exists x \qty(x \in \bigcup B \land \forall A \qty(A \in B \rightarrow x \in A) \land y = f(x)) \\*
    \exists x \qty(x \in \bigcup B \land y = f(x)) \land \exists x \qty(\forall A \qty(A \in B \rightarrow x \in A) \land y = f(x)) \\*
    y \in f \qty(\bigcup B) \land \forall A \qty(A \in B \rightarrow y \in f(A)) \\*
    y \in \bigcup \qty{f(A) \mid A \in B} \land \forall A \qty(A \in B \rightarrow y \in f(A)) \\*
    y \in \bigcap \qty{f(A) \mid A \in B}
  }

  \vskip\baselineskip

  第四式について、
  \eqg*{
    y \in f(A_1) \setminus f(A_2) \\*
    \exists x \qty(y = f(x) \land x \in A_1) \land \forall x \qty(y = f(x) \rightarrow x \notin A_2) \\*
    \exists x \qty(y = f(x) \land x \in A_1 \setminus A_2) \\*
    y \in f \qty(A_1 \setminus A_2)
  }
}

\dfn{原像}{
  写像$f \colon X \to Y$と集合$A$について、以下で定める集合を、集合$A$の原像と呼び、$f^{-1}(A)$で表す。
  \eq*{
    f^{-1}(A) = \qty{x \in X \mid f(x) \in A}
  }
}

\cor{像と原像}{
  写像$f \colon X \to Y$について、以下が成り立つ。
  \eqg*{
    \forall A \subset X \qty(f^{-1} \qty(f(A)) \supset A) \\*
    \forall B \qty(f \qty(f^{-1}(B)) \subset B)
  }
}

\thm{原像の性質}{
  写像$f$について、以下が成り立つ。
  \eqg*{
    \forall A_1, A_2 \qty(A_1 \subset A_2 \rightarrow f^{-1}(A_1) \subset f^{-1}(A_2)) \\*
    \forall B \qty(f^{-1} \qty(\bigcup B) = \bigcup \qty{f^{-1}(A) \mid A \in B}) \\*
    \forall B \qty(f^{-1} \qty(\bigcap B) = \bigcap \qty{f^{-1}(A) \mid A \in B}) \\*
    \forall A_1, A_2 \qty(f^{-1}(A_1) \setminus f^{-1}(A_2) = f^{-1}(A_1 \setminus A_2))
  }
}{
  第一式について、
  $x \in f^{-1}(A_1)$のとき、$f(x) \in A_1 \subset A_2$ゆえに$x \in f^{-1}(A_2)$

  \vskip\baselineskip

  第二式について、
  \eqg*{
    x \in \bigcup \qty{f^{-1}(A) \mid A \in B} \\*
    \exists A \in B \qty(x \in f^{-1}(A)) \\*
    \exists A \in B \qty(f(x) \in A) \\*
    f(x) \in \bigcup B \\*
    x \in f^{-1} \qty(\bigcup B)
  }

  上からも下からも成り立つので、\axiref{外延性の公理}より成り立つ。

  \vskip\baselineskip

  第三式について、第二式を用いて、
  \eqg*{
    x \in f^{-1} \qty(\bigcap B) \\*
    f(x) \in \bigcap B \\*
    f(x) \in \bigcup B \land \forall A \qty(A \in B \rightarrow f(x) \in A) \\*
    x \in f^{-1} \qty(\bigcup B) \land \forall A \qty(A \in B \rightarrow x \in f^{-1}(A)) \\*
    x \in \bigcup \qty{f^{-1}(A) \mid A \in B} \land \forall A \qty(A \in B \rightarrow x \in f^{-1}(A)) \\*
    x \in \bigcap \qty{f^{-1}(A) \mid A \in B}
  }

  上からも下からも成り立つので、\axiref{外延性の公理}より成り立つ。

  \vskip\baselineskip

  第四式について、
  \eqg*{
    x \in f^{-1}(A_1) \setminus f^{-1}(A_2) \\*
    f(x) \in A_1 \land f(x) \notin A_2 \\*
    f(x) \in A_1 \setminus A_2 \\*
    x \in f^{-1} \qty(A_1 \setminus A_2)
  }

  上からも下からも成り立つので、\axiref{外延性の公理}より成り立つ。
}


\lsubsection{単射と全射}

\dfn{単射}{
  写像$f$が左一意的であるとき、$f$は単射であると言う。

  単射な写像を、単に単射と呼ぶ。
}

\cor*{
  単射$f \colon X \to Y, g \colon Y \to Z$について、$g \circ f$は単射である。
}

\lem*{
  集合$X, Y$について、単射$f \colon X \to Y$が存在するならば、単射$g \colon \P(X) \to \P(Y)$が存在する。
}{
  $g(A) \coloneqq \qty{f(x) \mid x \in A}$で定義する写像は、単射である。
}

\dfn{全射}{
  写像$f$が右全域的であるとき、$f$は全射であると言う。

  全射な写像を、単に全射と呼ぶ。
}

\dfn{全単射}{
  写像$f$が、単射かつ全射であるとき、$f$は全単射であると言う。

  全単射な写像を、単に全単射と呼ぶ。
}

\cor*{
  恒等写像は全単射である。
}

\dfn{左逆写像}{
  写像$f \colon X \to Y$について、以下を満たす写像$g \colon Y \to X$を左逆写像と呼ぶ。
  \eq*{
    g \circ f = \id{X}
  }
}

\cor*{
  左逆写像は全射である。
}

\thm{単射と左逆写像}{
  空でない集合$X$と、写像$f \colon X \to Y$について、$f$が単射であることは、写像$f$が左逆写像$g$を持つことと同値である。
}{
  必要性を示す。

  空でないので$\exists x_0 \in X$より、以下で定める集合$G$が存在する。
  \eq*{
    G \coloneqq \qty{\qty(y, x) \in Y \times X \mid y = f(x)} \cup \qty{\qty(y, x_0) \mid y \in Y \setminus \Im(f)}
  }

  写像$g \coloneqq \qty(\qty(Y, X), G)$は、定義より$f$の左逆写像となる。

  \vskip\baselineskip

  十分性を示す。

  単射でない、すなわち$\exists x, y \in X \qty(x \neq y \land f(x) = f(y))$とする。

  定義より、$x = g(f(x)) = g(f(y)) = y$であるので矛盾。

  背理法より示される。
}

\dfn{右逆写像}{
  写像$f \colon X \to Y$について、以下を満たす写像$g \colon Y \to X$を右逆写像と呼ぶ。
  \eq*{
    f \circ g = \id{Y}
  }
}

\cor*{
  右逆写像は単射である。
}

\thm{全射と右逆写像}{
  写像$f \colon X \to Y$について、$f$が全射であることは、写像$f$が右逆写像$g$を持つことと同値である。
}{
  必要性を示す。
  \thmref{選択公理が与える写像}の与える写像は、右逆写像である。

  \vskip\baselineskip

  十分性を示す。

  全射でないとすると、$\exists y \in Y \forall x \in X \qty(f(x) \neq y)$で仮定に反する。

  背理法より示される。
}

\dfn{逆写像}{
  写像$f \colon X \to Y$について、写像$g \colon Y \to X$が$f$の左逆写像かつ右逆写像であるとき、逆写像と呼ぶ。
}

\cor*{
  逆写像は全単射である。
}

\thm{全単射と逆写像}{
  写像$f \colon X \to Y$について、$f$が全単射であることは、写像$f$が逆写像$g$を持つことと同値である。
}{
  必要性を示す。$f$は全射より、$\Im(f) = Y$

  以下のような集合系$G$が存在する。
  \eq*{
    G \coloneqq \qty{\qty(y, x) \in Y \times X \mid y = f(x)}
  }

  写像$g = \qty(\qty(Y, X), G)$は定義より$f$の逆写像となる。

  \vskip\baselineskip

  十分性を示す。

  $X$が空のとき、逆写像を持つので$Y$は空である。
  ゆえに全単射。

  $X$が空でないとき、\thmref{単射と左逆写像}と\thmref{全射と右逆写像}より$f$は全単射。
}

\lem*{
  写像$f \colon X \to Y$について、$f$の逆写像が存在するならば、一意に定まる。

  ここから、$f$の逆写像を$f^{-1}$で表す。
}{
  逆写像$g, h$について、$g(y) = h \circ f \circ g (y) = h(y)$である。
  ゆえに一意。
}

\cor*{
  全単射$f \colon X \to Y$について、
  \eq*{
    \qty(f^{-1})^{-1} = f
  }
}

\lem*{
  全単射$f \colon X \to Y$と$Y$の部分集合$A$について、原像$f^{-1}(A)$と逆写像の像$f^{-1}(A)$は一致する。
}{
  全単射より、$\forall y \in A \exists x \in A \qty(y = f(x))$。
  定義より明らか。
}

\thm{Cantorの対角線論法}{
  単射$f \colon \P(A) \to A$は存在しない。
}{
  存在すると仮定する。

  以下の集合$Y$を考える。
  \eq*{
    Y \coloneqq \qty{f(X) \mid X \in \P(A) \land f(X) \notin X}
  }
  このとき$Y \in \P(A)$である。

  $f(Y) \notin Y$とすると、定義より$f(Y) \in Y$。ゆえに矛盾。

  $f(Y) \in Y$とすると、$f$の単射性より$Y \in \P(A) \land f(Y) \notin Y$。ゆえに矛盾。

  背理法より示される。
}

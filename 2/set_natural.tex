\lsection{有限と可算}

\lsubsection{有限}

\dfn{有限集合}{
  集合$X$について、ある自然数$n \in \N$が存在して全単射$f \colon n \to X$が構成できるとき、$X$を有限集合と呼ぶ。
  または、単に$X$は有限であると呼ぶ。
}

\lem{全射と要素数}{
  $\forall n, m \in \N$について、$n$から$m$への全射が存在するならば、$m \leq n$
}{
  $n = 0$のとき、$m \neq 0$ならば全射が存在しない。$m = 0$より示される。

  \vskip\baselineskip

  ある$n$で成り立つとき、全射$f \colon s(n) \to m$が存在することを考える。

  $0^{s(n)} = \varnothing$より、$m \neq 0$である。

  $\exists k \in n \qty(f(k) = f(n))$のとき、自明な全射$f \rvert_n$を構成できて、仮定より$m \leq n \leq s(n)$

  $\forall k \in n \qty(f(k) \neq f(n))$のとき、以下のような全射$g \colon n \to p(m)$を構成できる。
  \eq*{
    g(x) =
    \begin{cases}
      f(x) & \qty(f(x) < f(n)) \\*
      p(f(x)) & \qty(f(x) > f(n))
    \end{cases}
  }

  仮定より$p(m) \leq n$より、$m = s(p(m)) \leq s(n)$

  \vskip\baselineskip

  \thmref{数学的帰納法}より、任意の$n$について成り立つ。
}

\lem{単射と要素数}{
  自然数$n, m \in \N$について、$n$から$m$への単射が存在するならば、$n \leq m$
}{
  $n = 0$のとき、明らか。

  $n \neq 0$のとき、\thmref{単射と左逆写像}より$m$から$n$への左逆写像すなわち全射を構成できて、\lemref{全射と要素数}より$n \leq m$
}

\thm{鳩の巣原理}{
  自然数$n, m \in \N$について、$m < n$ならば$n$から$m$への単射が存在しない。
}{
  \lemref{単射と要素数}の対偶である。
}

\lem*{
  有限集合$X$について、全単射$f \colon n \to X$が構成できる自然数$n \in \N$は一意に定まる。
}{
  \lemref{全射と要素数}、\lemref{単射と要素数}より、背理法より示される。
}

\dfn{要素数}{
  有限集合$X$について、\mlemref{0}より主張される自然数を要素数$\abs{X}$と呼ぶ。

  また、集合$X$が有限集合であることを主張するときは、アリティ1の述語記号$\abs{X} < \infty$を用いる。
}

\lem{単射と有限}{
  自然数$n$と集合$A$について、以下の2つは同値である。
  \begin{enumerate}
    \item $A$は有限集合であり、$\abs{A} \leq n$
    \item 単射$f \colon A \to n$が存在する。
  \end{enumerate}
}{
  $1. \rightarrow 2.$を示す。

  全単射$f_1 \colon A \to \abs{A}$が存在して、単射$f_2 \colon \abs{A} \to n$が存在する。

  ゆえに、単射$f_2 \circ f_1 \colon A \to n$が存在する。

  \vskip\baselineskip

  $2. \rightarrow 1.$を示す。

  $n = 0$のとき、$n^A \neq \varnothing$より、$A = \varnothing$である。

  ある$n$について成り立つとする。

  $\forall a \in A \qty(f(a) \neq n)$のとき、自明な単射$\tilde{f} \colon A \to n$が存在する。
  よって$\abs{A} \leq n \leq s(n)$

  $\exists a \in A \qty(f(a) = n)$のとき、単射$\tilde{f} \colon A \setminus \qty{a} \to n$が構成できる。

  仮定より$\exists m \in \N \qty(m \leq n)$であり、全単射$g \colon A \setminus \qty{a} \to m$が存在する。

  ここで、以下のような全単射$h \colon A \to s(m)$を構成できる。
  \eq*{
    h(x) =
    \begin{cases}
      m & \qty(x = a) \\*
      g(f(x)) & \qty(x \neq a)
    \end{cases}
  }

  $\abs{A} = s(m) \leq s(n)$を得るので成り立つ。
}

\lem{有限集合の部分}{
  有限集合$X$について、$X$の部分$Y$は有限であり、$\abs{Y} \leq \abs{X}$である。
}{
  単射$f \colon Y \to X, f(y) = y$と、$n \in \N$で全単射$g \colon X \to n$が存在する。

  ゆえに、単射$g \circ f \colon Y \to n$が存在して、\lemref{単射と有限}より成り立つ。
}

\lem{有限集合の像}{
  有限集合$X$、集合$Y$、写像$f \colon X \to Y$について、像$f(X)$は有限であり、$\abs{f(X)} \leq \abs{X}$である。
}{
  \thmref{全射と右逆写像}より、$f \colon X \to f(X)$は右逆写像すなわち単射$f^{-1} \colon f(X) \to X$を持つ。

  $n \in \N$と全単射$g \colon X \to n$が存在するので、単射$g \circ f^{-1} \colon f(X) \to n$が存在して、\lemref{単射と有限}より成り立つ。
}

\lem{有限集合の商}{
  有限集合$X$と、$X$上の同値関係$\sim$について、商集合$X / \sim$は有限であり、$\abs{X / \sim} \leq \abs{X}$である。
}{
  商写像$\qty[]$について、$X / \sim = \qty[X]$である。\lemref{有限集合の像}より成り立つ。
}

\lem{有限集合の和}{
  有限集合$X, Y$について、$X \cap Y = \varnothing$ならば、$X \cup Y$は有限集合となり、以下が成り立つ。
  \eq*{
    \abs{X \cup Y} = \abs{X} + \abs{Y}
  }
}{
  $X, Y$は有限より、全単射$f \colon \abs{X} \to X, f' \colon \abs{Y} \to Y$が存在する。

  ゆえに、以下のような全単射$g \colon \abs{X} + \abs{Y} \to X \cup Y$を構成できる。
  \eq*{
    g(x) =
    \begin{cases}
      g(x) & \qty(x < \abs{X}) \\*
      g'(x - \abs{X}) & \qty(\abs{X} \leq x < \abs{X} + \abs{Y})
    \end{cases}
  }
}

\lem{有限集合の和と共通部分}{
  有限集合$X, Y$について、$X \cup Y, X \cap Y$は有限集合となり、以下が成り立つ。
  \eq*{
    \abs{X \cup Y} + \abs{X \cap Y} = \abs{X} + \abs{Y}
  }
}{
  単射$f \colon X \cap Y \to X$が構成できるので、\lemref{単射と有限}より$X \cap Y$は有限集合である。

  \lemref{有限集合の和}より、$\abs{X} = \abs{X \cap Y} + \abs{X \setminus Y}$

  \lemref{有限集合の和}より、$\abs{X \cup Y} = \abs{\qty(X \setminus Y) \cup Y} = \abs{X \setminus Y} + \abs{Y}$

  したがって、$\abs{X \cup Y} + \abs{X \cap Y} + \abs{X \setminus Y} = \abs{X \setminus Y} + \abs{Y} + \abs{X}$

  \dfnref{自然数の減法}より成り立つ。
}

\lem{有限集合の直積}{
  有限集合$X, Y$について、$X \times Y$は有限集合となり、以下が成り立つ。
  \eq*{
    \abs{X \times Y} = \abs{X} \times \abs{Y}
  }
}{
  全単射$x \colon \abs{X} \to X$が存在する。

	$\forall n \in \abs{X}$について、$Z_n \coloneqq \qty{\qty(x(n), y) \mid y \in Y}$を考えると、$\abs{Z_n} = \abs{Y}$である。

	$X \times Y = \bigsqcup \qty{Z_n \mid n \in \abs{X}}$が成り立つ。

	\lemref{有限集合の和}より、$\abs{X}$についての帰納法を用いて、$\abs{X \times Y} = \abs{X} \times \abs{Y}$
}

\thm{有限有向集合の上界}{
  空でない有向集合の有限部分集合$A$は、上界を持つ。
}{
  $\abs{A} = 1$のとき、$A$は唯一つの元を持ち、上界である。

  $\abs{A} = 2$のとき、$a_0 \neq a_1$を用いて$A = \qty{a_0, a_1}$と書ける。有向集合であることより上界が存在する。

  $\abs{A} = n$のとき成り立つとする。

  $\abs{A} = s(n)$について、$A$のある元$a$を考える。集合$A \setminus \qty{a}$は要素数が$n$であるため、仮定より上界$a_n$を持つ。

  ゆえに、今、集合$\qty{a, a_n}$は上界$a_{s(n)}$を持ち、推移律から$A$の上界となる。

  要素数$\abs{A}$について、\thmref{数学的帰納法}より示される。
}

\thm{有限全順序集合の最大元}{
  空でない全順序集合の有限部分集合$A$は、最大元と最小元を持つ。
}{
  $\abs{A} = 1$のとき、$A$は唯一つの元を持ち、最大元かつ最小元である。

  $\abs{A} = 2$のとき、$a_0 \neq a_1$を用いて$A = \qty{a_0, a_1}$と書ける。全順序性より$a_0 > a_1 \lor a_0 < a_1$である。

  $a_0 > a_1$のとき、最大元$a_0$、最小元$a_1$である。$a_0 < a_1$のとき、最大元$a_1$、最小元$a_0$である。

  $\abs{A} = n$のとき成り立つとする。

  $\abs{A} = s(n)$について、$A$のある元$a$を考える。集合$A \setminus \qty{a}$は要素数が$n$であるため、仮定より最大元$a_{\text{max}}$と最小元$a_{\text{min}}$を持つ。

  ゆえに、$A$は最大元$\max \qty{a, a_{\text{max}}}$を持ち、$A$は最小元$\min \qty{a, a_{\text{min}}}$を持つ。

  要素数$\abs{A}$について、\thmref{数学的帰納法}より示される。
}

\thm{自然数の上界と有限}{
  $\N$の部分集合$A$について、$A$が上に有界ならば$A$は有限である。
}{
  上界$n$について考える。

  $n = 0$のとき、$A \neq \varnothing$ならば全射が存在しないので、示される。

  ある$n$で成り立つとき、$A$が有限であることを考える。

  $s(n)$が上界で、$n$が上界でないとき、$s(n) \in A$である。

  仮定より、$\exists m \in \N$で、全単射$f_0 \colon m \to A \setminus \qty{s(n)}$が存在する。

  したがって、以下のような全単射$f \colon s(m) \to A$を構成できる。
  \eq*{
    f(x) =
    \begin{cases}
      f_0(x) & \qty(x \neq m) \\*
      s(n) & \qty(x = m)
    \end{cases}
  }

  \thmref{数学的帰納法}より、任意の$n$について成り立つ。
}


\lsubsection{可算}

\dfn{可算}{
  集合$X$について、$X$が空集合または全射$f \colon \N \to X$が存在するとき、$X$を可算集合と呼ぶ。
  または、単に$X$は可算であると呼ぶ。

  また、集合$X$が可算集合であることを主張するときは、アリティ1の述語記号$\abs{X} \leq \aleph_0$を用いる。
}

\cor*{
  可算集合の部分集合は可算集合である。
}

\dfn{有限列}{
  自然数$n \in \N$から集合$X$へのネットを、$X$上の有限列、または組、$n$-組と呼ぶ。
}

\dfn{点列}{
  自然数$\N$から集合$X$へのネットを、$X$上の点列と呼ぶ。
}

\dfn{点列の部分列}{
  単射な写像$\varphi \in \N^\N$を用いて表せる部分ネット$\qty(x_{\varphi(m)})_{m \in \N}$を点列$\qty(x_n)_{n \in \N}$の部分列と呼ぶ。
}

\dfn{非交叉列}{
  点列$\qty(x_n)_{n \in \N}$について以下が成り立つとき、$\qty(x_n)_{n \in \N}$を非交叉列と呼ぶ。
  \eq*{
    \forall n, m \in \N \qty(n \neq m \rightarrow x_n \cap x_m = \varnothing)
  }
}

\dfn{単調列}{
  半順序集合$X$上の点列$\qty(x_n)_{n \in \N}$について、以下のいずれかを満たす点列を単調列と呼ぶ。
  \eqg*{
    \forall n \in \N \qty(x_n < x_{s(n)}) \\*
    \forall n \in \N \qty(x_n \leq x_{s(n)}) \\*
    \forall n \in \N \qty(x_n > x_{s(n)}) \\*
    \forall n \in \N \qty(x_n \geq x_{s(n)})
  }

  特にそれぞれ、第一式を満たす点列を狭義単調増加列、第二式を広義単調増加列、第三式を狭義単調減少列、第四式を広義単調減少列と呼ぶ。
}

\thm{無限集合は可算部分を含む}{
  有限でない集合$X$について、単射$a \colon \N \to X$が存在する。
}{
  $\forall A \in \P(X) \setminus \qty{\varnothing} \exists a \in A \qty(a \in A)$である。

  \thmref{選択公理が与える写像}より定まる写像$f \colon \P(X) \setminus \qty{\varnothing} \to X, f(X) \in A$を考える。

  $X$は空でないので、$\exists a_0 \in X$である。

  $\forall n \in \N$について、点列$a_{s(n)} \coloneqq f(X \setminus \qty{a_m \mid m \in s(n)})$を考えることができる。

  なぜなら、$\exists n \in \N \qty(X \setminus \qty{a_m \mid m \in s(n)} = \varnothing)$とすると、$X$は有限となり仮定に反する。

  定義より$\forall n, m \in \N \qty(a_n \neq a_m)$

  単射$a \colon \N \to X$を持つ。
}

\thm{自然数の直積は可算}{
  $\N \times \N$は可算である。
}{
  全単射$\varphi \colon \N \times \N \to \N, \varphi \qty(n, m) = \qty(\qty(n + m) \times \qty(n + m + 1)) \divisionsymbol 2 + n$が存在する。

  ただし、$a \divisionsymbol b$を除算の商を表すものとする。
}


\lsubsection{集合と写像}

\lem*{
  集合$X$について、自然な全単射$\sigma \colon \P(X) \to 2^X$が存在する。
}{
  $x \in X$について、以下のように$\sigma$を定める。
  \eq*{
    \sigma(A)(x) =
    \begin{cases}
      1 & \qty(x \in A) \\*
      0 & \qty(x \notin A)
    \end{cases}
  }

  以下で定める$\sigma^{-1}$は逆写像である。
  \eq*{
    \sigma^{-1}(f) = \qty{x \in X \mid f(x) = 1}
  }

  \thmref{全単射と逆写像}より全単射である。
}

\lem{自然数から集合への写像の全体}{
  集合$X$と$n \in \N_{\geq 1}$について、全単射$\sigma \colon X^n \times X \to X^{s(n)}$が存在する。

  また、全単射$\sigma' \colon X \to X^1$が存在する。
}{
  以下で定める$\sigma$は全単射である。
  \eq*{
    \sigma(\qty(x_m)_{m \in n}, x)(l) =
    \begin{cases}
      x_l & \qty(l \in n) \\*
      x & \qty(l = n)
    \end{cases}
  }

  以下で定める$\sigma'$は全単射である。
  \eq*{
    \sigma(x)(0) = x
  }
}

\thm*{
  集合$X, Y, Z$について、自然な全単射$\sigma \colon \qty(X \times Y) \times Z \to X \times \qty(Y \times Z)$が存在する。
}{
  $\sigma \qty(\qty(x, y), z) = \qty(x, \qty(y, z))$は全単射である。
}

\thm{Bernsteinの定理}{
  集合$A, B$について、$A$から$B$への単射$f$が存在して、$B$から$A$の単射$g$が存在するならば、$A$から$B$への全単射が存在する。
}{
  以下のような点列$\qty(C_n)_{n \in \N}$を考える。
  \eq*{
    \begin{cases}
      C_0 \coloneqq A \setminus g(B) \\*
      C_{s(n)} \coloneqq g(f(C_n))
    \end{cases}
  }

  点列の値域全体の和集合$C \coloneqq \bigcup \qty{C_n \mid n \in \N}$を考える。

  今、\thmref{De Morganの法則}より$A \setminus C \subset g(B)$と、$g$の単射性から、以下のような写像$h \in B^A$が構成できる。

  \eq*{
    h(x) \coloneqq
    \begin{cases}
      f(x) & \qty(x \in C) \\*
      g^{-1}(x) & \qty(x \in A \setminus C)
    \end{cases}
  }

  \vskip\baselineskip

  単射でないと仮定すると、$x \in C_n \land y \in A \setminus C$について、
  \eq*{
    h(x) = h(y) \rightarrow g(f(x)) = y \in C_{s(n)}
  }

  矛盾するので、背理法より単射。

  \vskip\baselineskip

  $D \coloneqq \bigcup \qty{C_{s(n)} \mid n \in \N}$について、$D \subset g(B)$であり、\thmref{像の性質}より、
  \eq*{
    g^{-1}(D) = \bigcup \qty{g^{-1}(C_{s(n)}) \mid n \in \N} = \bigcup \qty{f(C_n) \mid n \in \N} = f(C)
  }

  今、$g(B) \subset A$より、
  \eq*{
    A \setminus C = A \setminus \qty(C_0 \cup D) = \qty(A \setminus C_0) \setminus D = \qty(A \setminus \qty(A \setminus g(B))) \setminus D = g(B) \setminus D
  }

  したがって$g$の単射性と\thmref{原像の性質}より
  \eq*{
    h(A \setminus C) = g^{-1}(A \setminus C) = g^{-1}(g(B) \setminus D) = B \setminus g^{-1}(D) = B \setminus f(C)
  }

  ゆえに全射。
}

\thm{濃度の比較可能定理}{
  集合$A, B$について、単射$f \colon A \to B$または単射$f' \colon B \to A$が存在する。
}{
  以下で定義する半順序集合$\qty(\mathcal{M}, \subset)$を考える。
  \eq*{
    \mathcal{M} \coloneqq \qty{G \in \P(X \times Y) \mid \forall \qty(x_1, y_1), \qty(x_2, y_2) \in G \qty(x_1 = x_2 \leftrightarrow y_1 = y_2)}
  }

  $\mathcal{M}$の全順序部分$C$を考える。

  $\forall \qty(x_1, y_1), \qty(x_2, y_2) \in \bigcup C \exists G_1, G_2 \in C \qty(\qty(x_1, y_1) \in G_1 \land \qty(x_2, y_2) \in G_2)$である。

  ここで$G_1 \subset G_2 \lor G_2 \subset G_1$より、$x_1 = x_2 \leftrightarrow y_1 = y_2$である。

  ゆえに、$\bigcup C \in \mathcal{M}$は上界である。

  したがって、$\mathcal{M}$は帰納的である。\\*

  \thmref{Zornの補題}より、極大元$G_0$が存在する。

  $G_0$によって特徴づけられる二項関係$R_0$を考える。

  $\exists \qty(x, y) \in \qty(A \setminus \dom(R)) \times \qty(B \setminus \ran(R))$とする。

  このとき、$G_1 = \qty{\qty(x, y)} \cup G_0 \in \mathcal{M}$となるので、$G_0$の極大性に反する。\\*

  ゆえに、$A = \dom(R) \lor B = \ran(R)$である。

  $A = \dom(R)$のとき、$G_0$の特徴づける写像は単射である。

  $B = \ran(R)$のとき、$G_0$の特徴づける写像は全射である。\thmref{全射と右逆写像}より単射な右逆写像が存在する。
}

\lsection{圏論}

\lsubsection{射の公理}

\rem{対象と恒等射の同一視}{
  本ノートでは対象を恒等射と同一視する。
  一般的でないことに注意されたい。
}

\dfn{射}{
  圏論では、\num{6}つの公理(
    \axiref{始域は冪等}、\axiref{終域は冪等}、\axiref{対象}、\axiref{合成射の始域と終域}、\axiref{合成射の結合}、\axiref{対象は恒等射}
  )が与えられる。

  項を射と呼ぶ。
}

\dfn{始域}{
  アリティ\num{1}の関数記号$\dom()$を考える。

  射$f$について、$\dom(f)$を$f$の始域と呼ぶ。
}

\axi{始域は冪等}{
  \eq*{
    \forall f \qty(\dom(\dom(f)) = \dom(f))
  }
}

\dfn{終域}{
  アリティ\num{1}の関数記号$\cod()$を考える。

  射$f$について、$\cod(f)$を$f$の終域と呼ぶ。
}

\axi{終域は冪等}{
  \eq*{
    \forall f \qty(\cod(\cod(f)) = \cod(f))
  }
}

\axi{始域の冪等性と終域の冪等性}{
  \eq*{
    \forall f \qty(f = \dom(f) \leftrightarrow f = \cod(f))
  }
}

\dfn{合成}{
  アリティ\num{2}の関数記号$\circ \qty()$を考える。

  $\circ \qty(f, g)$を、簡単のために$g \circ f$で表す。

  射$f, g$について、$g \circ f$を$f$と$g$の合成と呼ぶ。
}

\dfn{可合成}{
  アリティ\num{2}の述語記号$\Comp()$を、以下で定める。
  \eq*{
    \Comp(f, g) \defiff \cod(f) = \dom(g)
  }
}

\axi{合成射の始域と終域}{
  \eq*{
    \forall f, g \qty(\Comp(f, g) \rightarrow \dom(g \circ f) = \dom(f) \land \cod(g \circ f) = \cod(g))
  }
}

\axi{合成射の結合}{
  \eq*{
    \forall f, g, h \qty(\Comp(f, g) \land \Comp(g, h) \rightarrow h \circ \qty(g \circ f) = \qty(h \circ g) \circ f)
  }
}

\axi{対象は恒等射}{
  \eq*{
    \forall f \qty(f = f \circ \dom(f) \land f = \cod(f) \circ f)
  }
}

\dfn{対象}{
  以下を満たす射$a$を、対象、または、恒等射と呼ぶ。
  \eq*{
    a = \dom(a)
  }
}


\lsubsection{圏と関手}

\dfn{圏}{
  類$\bm{C}$が圏であるとは、以下を満たすことである。
  \eqg*{
    \forall f \in \bm{C} \qty(\dom(f) \in \bm{C} \land \cod(f) \in \bm{C}) \\*
    \forall f, g \in \bm{C} \qty(\Comp(f, g) \rightarrow g \circ f \in \bm{C})
  }
}

\dfn{対象の類}{
  以下で定める類$\Obj_{\bm{C}}$を、圏$\bm{C}$の対象の類と呼ぶ。
  \eq*{
    f \in \Obj_{\bm{C}} \defiff f \in \bm{C} \land f = \dom(f)
  }
}

\cor*{
  圏$\bm{C}$について、$\Obj_{\bm{C}}$は圏である。
}

\dfn{離散圏}{
  圏$\bm{C}$が離散であるとは、以下を満たすことである。
  \eq*{
    \forall f \qty(f \in \bm{C} \leftrightarrow f \in \Obj_{\bm{C}})
  }

  このとき、$\bm{C}$を離散圏と呼ぶ。
}

\dfn{Hom類}{
  圏$\bm{C}$と、$\bm{C}$の対象$a, b$について、類$\Hom_{\bm{C}}(a, b)$を以下で定める。
  \eq*{
    f \in \Hom_{\bm{C}}(a, b) \defiff f \in \bm{C} \land \dom(f) = a \land \cod(f) = b
  }
}

\rem{射の域の明示}{
  $f \in \Hom_{\bm{C}}(a, b)$を、簡単のため、$\bm{C}$の射$f \colon a \to b$と表記する。
}

\dfn{End類}{
  圏$\bm{C}$と、$\bm{C}$の対象$a$について、類$\End_{\bm{C}}(a)$を以下で定める。
  \eq*{
    f \in \End_{\bm{C}}(a) \defiff f \in \Hom_{\bm{C}}(a, a)
  }
}

\dfnf{圏$\bm{1}$}{圏1}{
  以下を満たす圏を、$\bm{1}$で表す。
  \eq*{
    \exists! a \in \bm{1}
  }

  圏$\bm{1}$の唯一の射を、$\ast$で表す。
}

\dfnf{圏$\bm{\rightarrow \leftarrow}$}{圏コスパン}{
  $m_0, m_1$について、以下で定める類$\bm{\rightarrow \leftarrow}$は、圏である。
  \begin{enumerate}
    \item $f \in \bm{\rightarrow \leftarrow} \, \leftrightarrow \, f = m_0 \lor f = m_1 \lor f = \dom(m_0) \lor f = \dom(m_1) \lor f = \cod(m_0)$
    \item $m_0, m_1 \notin \Obj_{\bm{\rightarrow \leftarrow}}$
    \item $\cod(m_0) = \cod(m_1)$
    \item $m_0 \neq m_1 \land \dom(m_0) \neq \dom(m_1) \land \dom(m_0) \neq \cod(m_0) \land \dom(m_1) \neq \cod(m_0)$
  \end{enumerate}
}

\dfnf{圏$\bm{\rightrightarrows}$}{圏並行射}{
  $m_0, m_1$について、以下で定める類$\bm{\rightrightarrows}$は、圏である。
  \begin{enumerate}
    \item $f \in \bm{\rightrightarrows} \, \leftrightarrow \, f = m_0 \lor f = m_1 \lor f = \dom(m_0) \lor f = \cod(m_0)$
    \item $m_0, m_1 \notin \Obj_{\bm{\rightrightarrows}}$
    \item $\dom(m_0) = \dom(m_1) \land \cod(m_0) = \cod(m_1)$
    \item $m_0 \neq m_1 \land \dom(m_0) \neq \cod(m_0)$
  \end{enumerate}
}

\dfn{関手}{
  アリティ\num{1}の関数記号$F$が、圏$\bm{C}$から圏$\bm{D}$への関手であるとは、以下の\num{3}つを満たすことである。
  \eqg*{
    \forall f \in \bm{C} \qty(F(f) \in \bm{D}) \\*
    \forall f \in \bm{C} \qty\Big(\dom(F(f)) = F(\dom(f)) \land \cod(F(f)) = F(\cod(f))) \\*
    \forall f, g \in \bm{C} \qty\Big(\Comp(f, g) \rightarrow F(g \circ f) = F(g) \circ F(f))
  }
}

\cor*{
  圏$\bm{C}, \bm{D}$について、$\bm{C}$から$\bm{D}$への関手$F$を考える。
  このとき、以下が成り立つ。
  \eq*{
    \forall a \in \Obj_{\bm{C}} \qty(F(a) \in \Obj_{\bm{D}})
  }
}

\cor*{
  圏$\bm{C}, \bm{D}$について、$\bm{C}$から$\bm{D}$への関手$F$を考える。
  このとき、以下が成り立つ。
  \eq*{
    \forall f \in \bm{C} \qty(\text{$f$が同型射} \rightarrow \text{$F(f)$が同型射})
  }
}

\dfn{関手の合成}{
  圏$\bm{C}, \bm{D}, \bm{E}$を考える。

  $\bm{C}$から$\bm{D}$への関手$F$と、$\bm{D}$から$\bm{E}$への関手$G$について、以下で示すアリティ\num{1}の関数記号$H$は、$\bm{C}$から$\bm{E}$への関手である。
  \eq*{
    H(f) \coloneqq G(F(f))
  }

  この$H$を、$F$と$G$の合成と呼び、$G \circ F$で表す。
}

\dfn{恒等関手}{
  圏$\bm{C}$について、関数記号$\id$は、$\bm{C}$から$\bm{C}$への関手である。

  圏論において、$\id$を恒等関手と呼ぶ。
  また、$\bm{C}$から$\bm{C}$への恒等関手であることを、明示的に$\id_{\bm{C}}$で表す。
}

\dfn{圏同型}{
  圏$\bm{C}, \bm{D}$を考える。

  $\bm{C}$から$\bm{D}$への関手$F$と、$\bm{D}$から$\bm{C}$への関手$G$が存在して、以下が成り立つとき、$\bm{C}$と$\bm{D}$は圏同型であると言う。
  \eq*{
    G \circ F = \id_{\bm{C}} \land F \circ G = \id_{\bm{D}}
  }
}


\lsubsection{自然変換と関手圏}

\dfn{自然変換}{
  圏$\bm{C}, \bm{D}$と、$\bm{C}$から$\bm{D}$への関手$F, F^\prime$を考える。

  アリティ\num{1}の関数記号$\eta$が、$F$から$F^\prime$への自然変換であるとは、以下の\num{2}つ全てを満たすことである。
  \eqg*{
    \forall a \in \Obj_{\bm{C}} \qty(\eta(a) \in \Hom_{\bm{D}}(F(a), F^\prime(a))) \\*
    \forall f \in \bm{C} \qty\Big(F^\prime(f) \circ \eta(\dom(f)) = \eta(\cod(f)) \circ F(f))
  }
}

\begin{center}
  \begin{minipage}{0.15\linewidth}
    $\bm{C}$ \\
    \begin{tikzcd}
      a \ar[r, "f"] & b \\
    \end{tikzcd}
  \end{minipage}
  \begin{minipage}{0.25\linewidth}
    $\bm{D}$ \\
    \begin{tikzcd}
      F(a) \ar[r, "F(f)"] \ar[d, "\eta(a)"'] & F(b) \ar[d, "\eta(b)"] \\
      F^\prime(a) \ar[r, "F^\prime(f)"'] & F^\prime(b) \\
    \end{tikzcd}
  \end{minipage}
\end{center}

\dfn{恒等な自然変換}{
  圏$\bm{C}, \bm{D}$と、$\bm{C}$から$\bm{D}$への関手$F$について、関数記号$F$は、$F$から$F$への自然変換である。

  $F$が自然変換であることを明示的に、関手$F$に対する恒等な自然変換$F$と呼ぶ。
}

\cor{恒等な自然変換は自然同型}{
  関手$F$に対する恒等な自然変換$F$は、$F$から$F$への自然同型である。
}

\dfn{自然変換の垂直合成}{
  圏$\bm{C}, \bm{D}$と、$\bm{C}$から$\bm{D}$への関手$F, F^\prime, F^{\prime \prime}$を考える。

  $F$から$F^\prime$への自然変換$\eta$と$F^\prime$から$F^{\prime \prime}$への自然変換$\theta$について、
  以下で定義するアリティ\num{1}の関数記号$\theta \circ \eta$は$F$から$F^{\prime \prime}$への自然変換である。
  \eq*{
    \qty(\theta \circ \eta)(a) \coloneqq \theta(a) \circ \eta(a)
  }

  $\theta \circ \eta$を、$\eta$と$\theta$の垂直合成と呼ぶ。
}

\begin{center}
  \begin{minipage}{0.15\linewidth}
    $\bm{C}$ \\
    \begin{tikzcd}
      a \ar[r, "f"] & b \\
    \end{tikzcd}
  \end{minipage}
  \begin{minipage}{0.25\linewidth}
    $\bm{D}$ \\
    \begin{tikzcd}
      F(a) \ar[r, "F(f)"] \ar[d, "\eta(a)"'] \ar[dd, bend right=60, red, "\qty(\theta \circ \eta)(a)"'] & F(b) \ar[d, "\eta(b)"] \ar[dd, bend left=60, red, "\qty(\theta \circ \eta)(b)"] \\
      F^\prime(a) \ar[r, "F^\prime(f)"'] \ar[d, "\theta(a)"'] & F^\prime(b) \ar[d, "\theta(b)"] \\
      F^{\prime \prime}(a) \ar[r, "F^{\prime \prime}(f)"'] & F^{\prime \prime}(b) \\
    \end{tikzcd}
  \end{minipage}
\end{center}

\cor*{
  圏$\bm{C}, \bm{D}$と、$\bm{C}$から$\bm{D}$への関手$F, F^\prime$を考える。

  $F$から$F^\prime$への自然変換$\eta$について、以下が成り立つ。
  \eq*{
    \forall a \in \Obj_{\bm{C}} \qty\Big(\eta(a) = \qty(\eta \circ F)(a) \land \eta(a) = \qty(F^\prime \circ \eta)(a))
  }
}

\cor*{
  圏$\bm{C}, \bm{D}$と、$\bm{C}$から$\bm{D}$への関手$F, F^\prime, F^{\prime \prime}, F^{\prime \prime \prime}$を考える。

  $F$から$F^\prime$への自然変換$\eta$、$F^\prime$から$F^{\prime \prime}$への自然変換$\theta$、$F^{\prime \prime}$から$F^{\prime \prime \prime}$への自然変換$\epsilon$について、以下が成り立つ。
  \eq*{
    \forall a \in \Obj_{\bm{C}} \qty\Big(\qty(\epsilon \circ \qty(\theta \circ \eta))(a) = \qty(\qty(\epsilon \circ \theta) \circ \eta)(a))
  }
}

\dfn{自然変換の水平合成}{
  圏$\bm{C}, \bm{D}, \bm{E}$と、$\bm{C}$から$\bm{D}$への関手$F, F^\prime$、$\bm{D}$から$\bm{E}$への関手$G, G^\prime$を考える。

  $F$から$F^\prime$への自然変換$\eta$と、$G$から$G^\prime$への自然変換$\theta$について、以下で定義するアリティ\num{1}の関数記号$\theta \ast \eta$は$G \circ F$から$G^\prime \circ F^\prime$への自然変換である。
  \eq*{
    \qty(\theta \ast \eta)(a) \coloneqq \theta(F^\prime(a)) \circ G(\eta(a))
  }

  $\theta \ast \eta$を、$\eta$と$\theta$の水平合成と呼ぶ。
}

\begin{center}
  \begin{minipage}{0.15\linewidth}
    $\bm{C}$ \\
    \begin{tikzcd}
      a \ar[r, "f"] & b \\
    \end{tikzcd}
  \end{minipage}
  \begin{minipage}{0.25\linewidth}
    $\bm{D}$ \\
    \begin{tikzcd}
      F(a) \ar[r, "F(f)"] \ar[d, "\eta(a)"'] & F(b) \ar[d, "\eta(b)"] \\
      F^\prime(a) \ar[r, "F^\prime(f)"'] & F^\prime(b) \\
    \end{tikzcd}
  \end{minipage}
  \begin{minipage}{0.5\linewidth}
    $\bm{E}$ \\
    \begin{tikzcd}
      & G^\prime(F(a)) \ar[rr, "G^\prime(F(f))"] \ar[dd] & & G^\prime(F(b)) \ar[dd, "G^\prime(\eta(b))"] \\
      G(F(a)) \ar[rr, crossing over, "G(F(f))", near end] \ar[dd, "G(\eta(a))"'] \ar[ur, "\theta(F(a))"] \ar[rd, "\qty(\theta \ast \eta)(a)", red] & & G(F(b)) \ar[ur, "\theta(F(b))"] \ar[rd, "\qty(\theta \ast \eta)(b)", red] \\
      & G^\prime(F^\prime(a)) \ar[rr] & & G^\prime(F^\prime(b)) \\
      G(F^\prime(a)) \ar[rr, "G(F^\prime(f))"'] \ar[ur, "\theta(F^\prime(a))", near end] & & G(F^\prime(b)) \ar[from=uu, crossing over, "G(\eta(b))", near end] \ar[ur, "\theta(F^\prime(b))"'] \\
    \end{tikzcd}
  \end{minipage}
\end{center}

\cor*{
  圏$\bm{C}, \bm{D}, \bm{E}$と、$\bm{C}$から$\bm{D}$への関手$F, F^\prime$、$\bm{D}$から$\bm{E}$への関手$G, G^\prime$を考える。

  $F$から$F^\prime$への自然変換$\eta$と、$G$から$G^\prime$への自然変換$\theta$について、以下が成り立つ。
  \eq*{
    \forall a \in \Obj_{\bm{C}} \qty\Big(\qty(\theta \ast F)(a) = \theta(F(a)) \land \qty(G \ast \eta)(a) = G(\eta(a)))
  }
}

\lem*{
  圏$\bm{C}, \bm{D}, \bm{E}$と、$\bm{C}$から$\bm{D}$への関手$F, F^\prime$、$\bm{D}$から$\bm{E}$への関手$G, G^\prime$を考える。

  $F$から$F^\prime$への自然変換$\eta$と、$G$から$G^\prime$への自然変換$\theta$について、以下が成り立つ。
  \eq*{
    \forall a \in \bm{C} \qty\Big(\theta(F^\prime(a)) \circ G(\eta(a)) = G^\prime(\eta(a)) \circ \theta(F(a)))
  }
}{
  $\theta$は$G$から$G^\prime$への自然変換であるので、明らか。
}

\thm{相互交換法則}{
  圏$\bm{C}, \bm{D}, \bm{E}$と、$\bm{C}$から$\bm{D}$への関手$F, F^\prime, F^{\prime \prime}$、$\bm{D}$から$\bm{E}$への関手$G, G^\prime, G^{\prime \prime}$を考える。

  $F$から$F^\prime$への自然変換$\sigma$、$F^\prime$から$F^{\prime \prime}$への自然変換$\tau$、$G$から$G^\prime$への自然変換$\eta$、$G^\prime$から$G^{\prime \prime}$への自然変換$\theta$について、以下が成り立つ。
  \eq*{
    \qty(\theta \circ \eta) \ast \qty(\tau \circ \sigma) = \qty(\theta \ast \tau) \circ \qty(\eta \ast \sigma)
  }
}{
  $\bm{C}$の任意の対象$a$について、以下が成り立つ。
  \eqa*{
    \qty(\qty(\theta \circ \eta) \ast \qty(\tau \circ \sigma))(a)
    &= \qty(\theta \circ \eta)\qty(F^{\prime \prime}(a)) \circ G \qty(\qty(\tau \circ \sigma)(a)) \\*
    &= \theta\qty(F^{\prime \prime}(a)) \circ \eta\qty(F^{\prime \prime}(a)) \circ G\qty(\tau(a)) \circ G \qty(\sigma(a)) \\*
    &= \theta\qty(F^{\prime \prime}(a)) \circ G^\prime\qty(\tau(a)) \circ \eta\qty(F^\prime(a)) \circ G \qty(\sigma(a)) \\*
    &= \qty(\theta \ast \tau)(a) \circ \qty(\eta \ast \sigma)(a) \\*
    &= \qty(\qty(\theta \ast \tau) \circ \qty(\eta \ast \sigma))(a)
  }

  ただし第二行から第三行へは、$\eta$が$G$から$G^\prime$への自然変換であることを用いた。
}

\begin{center}
  \begin{minipage}{0.8\linewidth}
    $\bm{E}$ \\
    \begin{tikzcd}
      G(F(a)) \ar[rr, "\eta(F(a))"] \ar[dd, "G(\sigma(a))"'] \ar[dddd, bend right=90, red, "G\qty((\tau \circ \sigma)(a))"'] \ar[rrdd, red, "(\eta \ast \sigma) (a)"] & & G^\prime(F(a)) \ar[rr, "\theta(F(a))"] \ar[dd, "G^\prime(\sigma(a))"] & & G^{\prime \prime}(F(a)) \ar[dd, "G^{\prime \prime}(\sigma(a))"] \\
      & & & & \\
      G(F^\prime(a)) \ar[rr, "\eta(F^\prime(a))"] \ar[dd, "G(\tau(a))"'] & & G^\prime(F^\prime(a)) \ar[rr, "\theta(F^\prime(a))"] \ar[dd, "G^\prime(\tau(a))"] \ar[rrdd, red, "(\theta \ast \tau) (a)"] & & G^{\prime \prime}(F^\prime(a)) \ar[dd, "G^{\prime \prime}(\tau(a))"] \\
      & & & & \\
      G(F^{\prime \prime}(a)) \ar[rr, "\eta(F^{\prime \prime}(a))"'] \ar[rrrr, bend right, red, "\qty(\theta \circ \eta) \qty(F^{\prime \prime} (a))"'] & & G^\prime(F^{\prime \prime}(a)) \ar[rr, "\theta(F^{\prime \prime}(a))"'] & & G^{\prime \prime}(F^{\prime \prime}(a)) \\
    \end{tikzcd}
  \end{minipage}
\end{center}

\rem{自然変換は射}{
  圏$\bm{D}$から圏$\bm{C}$への関手$F, G$について、$F$から$G$への自然変換$\eta$を考える。

  $\dom(\eta)$を$F$に対する恒等な自然変換$F$、$\cod(\eta)$を$G$に対する恒等な自然変換$G$として、合成を垂直合成で定義すると、
  自然変換は射の公理をみたす。
}

\dfn{関手圏}{
  圏$\bm{C}, \bm{D}$について、自然変換を射とみなすと、
  $\bm{D}$から$\bm{C}$への関手間の自然変換であることは、圏となる。

  この圏に以下の相等の定義を加えたものを、$\bm{D}$から$\bm{C}$への関手圏と呼び、$\Func(\bm{D}, \bm{C})$または$\bm{C}^{\bm{D}}$と表す。
  \begin{enumerate}
    \item 対象(関手)$F, F^\prime$について、$\forall f \in \bm{C} \qty(F(f) = F^\prime(f)) \rightarrow F = F^\prime$
    \item 射(自然変換)$\eta, \theta$について、$\dom(\eta) = \dom(\theta) \land \cod(\eta) = \cod(\theta) \land \forall a \in \Obj_{\bm{C}} \qty(\eta(a) = \theta(a)) \rightarrow \eta = \theta$
  \end{enumerate}

  \num{1}行目は恒等な自然変換ではなく関手として扱っていることに注意されたい。
}

\dfn{対角関手}{
  圏$\bm{C}, \bm{D}$について、$\bm{C}$から$\Func(\bm{D}, \bm{C})$への以下で定める関手$\Delta$を考える。
  \begin{enumerate}
    \item $c \in \Obj_{\bm{C}}$について、$\bm{D}$から$\bm{C}$への関手$\Delta(c), \forall d \in \bm{D} \qty(\Delta(c)(d) \coloneqq c)$を与える。
    \item $f \notin \Obj_{\bm{C}}$について、$\Delta(\dom(f))$から$\Delta(\cod(f))$への自然変換$\forall d \in \Obj_{\bm{D}} \qty(\Delta(f)(d) \coloneqq f)$を与える。
  \end{enumerate}

  このように定めた関手$\Delta$を対角関手と呼ぶ。
}


\lsubsection{双対}

\dfn{反対射}{
  射$f$について、以下を満たすアリティ\num{1}の関数記号$\opp$を考える。
  簡単のために、$\opp(f)$を$f^\opp$で表す。
  \eqg*{
    \forall f \qty(\qty(f^\opp)^\opp = f) \\*
    \forall f \qty(f = \dom(f) \rightarrow f = f^\opp) \\*
    \forall f \qty(\dom(f^\opp) = \cod(f)^\opp) \\*
    \forall f \qty(\cod(f^\opp) = \dom(f)^\opp) \\*
    \forall f, g \qty(\Comp(f, g) \leftrightarrow \Comp(g^\opp, f^\opp)) \\*
    \forall f, g \qty(\Comp(f, g) \rightarrow f^\opp \circ g^\opp = \qty(g \circ f)^\opp)
  }
}

\dfn{反対圏}{
  圏$\bm{C}$について、以下で定める類$\bm{C}^\opp$は圏である。
  \eq*{
    f \in \bm{C}^\opp \defiff f^\opp \in \bm{C}
  }
}

\dfn{反対関手}{
  $\bm{C}$から$\bm{D}$への関手$F$について、以下で定める$\bm{C}^\opp$から$\bm{D}^\opp$への関手を、$F^\opp$で表す。
  \eq*{
    F^\opp(f) \coloneqq \qty(F(f^\opp))^\opp
  }
}

\rem{双対性}{
  圏論における命題は、$\dom()$と$\cod()$を入れ替えて、合成$g \circ f$を$f \circ g$に置き換えても、その真偽は変わらない。

  より簡単に言えば、$\bm{C}$で成り立つ命題は、$\bm{C}^\opp$でも成り立つ。
}

\cor{圏1の双対}{
  $\bm{1}^\opp$は、$\bm{1}$である。
}

\cor{関手圏の双対}{
  $\Func(\bm{D}, \bm{C})^\opp$と$\Func(\bm{D}^\opp, \bm{C}^\opp)$は圏同型である。
}


\lsubsection{簡約と逆}

\dfn{左簡約可能}{
  圏$\bm{C}$と、$\bm{C}$の射$f$について、以下を満たすとき、$f$は$\bm{C}$において左簡約可能であるという。
  \eq*{
    \forall g, h \in \bm{C} \qty\Big(\Comp(g, f) \land \Comp(h, f) \land f \circ g = f \circ h \rightarrow g = h)
  }

  $\bm{C}$において左簡約可能な射を、$\bm{C}$のモノ射と呼ぶ。
  特に圏$\bm{C}$が明らかな場合は、単にモノ射と呼ぶ。
}

\dfn{右簡約可能}{
  圏$\bm{C}$と、$\bm{C}$の射$f$について、$f^\opp$が$\bm{C}^\opp$で左簡約可能であるとき、$f$は$\bm{C}$において右簡約可能であるという。

  $\bm{C}$において右簡約可能な射を、$\bm{C}$のエピ射と呼ぶ。
  特に圏$\bm{C}$が明らかな場合は、単にエピ射と呼ぶ。

}

\dfn{左可逆}{
  圏$\bm{C}$と、$\bm{C}$の射$f$について、以下を満たすとき、$f$は$\bm{C}$において左可逆であるという。
  \eq*{
    \exists g \in \bm{C} \qty(\Comp(g, f) \land g \circ f = \dom(f))
  }
}

\cor*{
  圏$\bm{C}$について、$\bm{C}$において左可逆な射は、$\bm{C}$において左簡約可能である。
}

\dfn{右可逆}{
  圏$\bm{C}$と、$\bm{C}$の射$f$について、$f^\opp$が$\bm{C}^\opp$で左可逆であるとき、$f$は$\bm{C}$において右可逆であるという。
}

\cor*{
  圏$\bm{C}$について、$\bm{C}$において右可逆な射は、$\bm{C}$において右簡約可能である。
}

\cor*{
  圏$\bm{C}$と、$\bm{C}$の射$f, g$について、$\Comp(f, g)$であるとする。
  このとき、以下が成り立つ。
  \begin{enumerate}
    \item $f, g$がともに$\bm{C}$において左簡約可能ならば、$g \circ f$は$\bm{C}$において左簡約可能である。
    \item $f, g$がともに$\bm{C}$において右簡約可能ならば、$g \circ f$は$\bm{C}$において右簡約可能である。
    \item $f, g$がともに$\bm{C}$において左可逆ならば、$g \circ f$は$\bm{C}$において左可逆である。
    \item $f, g$がともに$\bm{C}$において右可逆ならば、$g \circ f$は$\bm{C}$において右可逆である。
  \end{enumerate}
}

\cor*{
  圏$\bm{C}$と、$\bm{C}$の射$f, g$について、$\Comp(f, g)$であるとする。
  このとき、以下が成り立つ。
  \begin{enumerate}
    \item $g \circ f$が$\bm{C}$において左簡約可能ならば、$f$は$\bm{C}$において左簡約可能である。
    \item $g \circ f$が$\bm{C}$において右簡約可能ならば、$g$は$\bm{C}$において右簡約可能である。
  \end{enumerate}
}

\dfn{逆射}{
  射$f$について、以下を満たす射$g$を$f$の逆射と呼ぶ。
  \eq*{
    \Comp(f, g) \land \Comp(g, f) \land g \circ f = \dom(f) \land f \circ g = \cod(f)
  }
}

\lem{逆射の一意性}{
  射$g, h$が$f$の逆射であるとき、$g = h$である。
}{
  以下より成り立つ。

  $g = g \circ \dom(g) = g \circ \cod(f) = g \circ f \circ h = \dom(f) \circ h = \cod(h) \circ h = h$より、成り立つ。
}

\rem{逆射}{
  $f$の逆射は、\lemref{逆射の一意性}より一意に定まるので、これを$f^{-1}$と表す。
}

\cor*{
  逆射を持つ射$f$について、以下が成り立つ。
  \eq*{
    \qty(f^{-1})^{-1} = f
  }
}

\lem{同型射}{
  圏$\bm{C}$と、$\bm{C}$の射$f$について、以下の\num{2}つは同値である。
  \begin{enumerate}
    \item $f$は、$\bm{C}$に逆射を持つ。
    \item $f$は、$\bm{C}$において、左可逆かつ右可逆である。
  \end{enumerate}
}{
  $1. \to 2.$は明らか。

  \vskip\baselineskip

  $2. \to 1.$を示す。

  左可逆より、射$g$が存在して$g \circ f = \dom(f)$であり、右可逆より、射$h$が存在して$f \circ h = \cod(f)$である。

  $g = g \circ (f \circ h) = \qty(g \circ f) \circ h = h$であるので、$g = h$

  ゆえに、$g$は$f \circ g = \cod(f)$を満たす。
}

\dfn{同型射}{
  圏$\bm{C}$について、$\bm{C}$において左可逆かつ右可逆な射を、$\bm{C}$の同型射と呼ぶ。
}

\cor{恒等射は同型射}{
  圏$\bm{C}$について、$\bm{C}$の恒等射$f$の逆射は自身であり、ゆえに$f$は$\bm{C}$の同型射である。
}

\cor*{
  圏$\bm{C}, \bm{D}$と、$\bm{C}$から$\bm{D}$への関手$F$を考える。

  $\bm{C}$の同型射$f$について、$F(f)$は$\bm{D}$の同型射である。
}

\dfn{自然同型}{
  圏$\bm{C}, \bm{D}$を考える。
  関手圏$\Func(\bm{D}, \bm{C})$の同型射を自然同型と呼ぶ。
}

\cor*{
  圏$\bm{C}, \bm{D}$と、$\bm{D}$から$\bm{C}$への関手$F, F^\prime$を考える。

  自然変換$\eta \colon F \to F^\prime$が自然同型であるとは、以下と同値である。
  \eq*{
    \forall a \in \Obj_{\bm{D}} \qty(\text{$\eta(a)$は$\bm{C}$の同型射})
  }
}

\dfn{圏同値}{
  圏$\bm{C}, \bm{D}$を考える。

  $\bm{D}$から$\bm{C}$への関手$F$と、$\bm{C}$から$\bm{D}$への関手$G$、
  自然同型$\epsilon \colon F \circ G \to \id_{\bm{C}}, \theta \colon \id_{\bm{D}} \to G \circ F$が存在するとき、
  $\bm{C}$と$\bm{D}$は圏同値であると言う。
}

\begin{center}
  \begin{minipage}{0.4\linewidth}
    $\bm{D}$ \\
    \begin{tikzcd}
      a \ar[rr, "f"] \ar[d, bend left, "\theta(a)"] & & b \ar[d, bend left, "\theta(b)"] \\
      \qty(G \circ F)(a) \ar[rr, "\qty(G \circ F)(f)"] \ar[u, bend left, "\theta(a)^{-1}"] & & \qty(G \circ F)(b) \ar[u, bend left, "\theta(b)^{-1}"] \\
    \end{tikzcd}
  \end{minipage}
  \begin{minipage}{0.4\linewidth}
    $\bm{C}$ \\
    \begin{tikzcd}
      c \ar[rr, "g"] \ar[d, bend left, "\epsilon(c)^{-1}"] & & d \ar[d, bend left, "\epsilon(d)^{-1}"] \\
      \qty(F \circ G)(c) \ar[rr, "\qty(F \circ G)(g)"] \ar[u, bend left, "\epsilon(c)"] & & \qty(F \circ G)(d) \ar[u, bend left, "\epsilon(d)"] \\
    \end{tikzcd}
  \end{minipage}
\end{center}

\cor{圏同型ならば圏同値}{
  圏$\bm{C}, \bm{D}$について、$\bm{C}$と$\bm{D}$が圏同型ならば、$\bm{C}$と$\bm{D}$は圏同値である。
}


\lsubsection{普遍}

\dfn{普遍射}{
  圏$\bm{C}, \bm{D}$と、$\bm{D}$から$\bm{C}$への関手$F$、$\bm{C}$の対象$a$を考える。

  $F$から$a$への普遍射とは、$\bm{D}$の対象$x$と圏$\bm{C}$の射$u \colon F(x) \to a$の組であって、以下を満たすものである。
  \eq*{
    \forall y \in \Obj_{\bm{D}} \forall f \in \Hom_{\bm{C}}\qty(F(y), a) \exists! g \in \Hom_{\bm{D}}(y, x) \qty(f = u \circ F(g))
  }
}

\begin{center}
  \begin{minipage}{0.3\linewidth}
    $\bm{D}$ \\
    \begin{tikzcd}
      y \ar[r, "g", dotted, red] & x \\
      & \\
    \end{tikzcd}
  \end{minipage}
  \begin{minipage}{0.3\linewidth}
    $\bm{C}$ \\
    \begin{tikzcd}
      F(y) \ar[r, "F(g)", dotted] \ar[dr, "f"'] & F(x) \ar[d, "u"] \\
      & a \\
    \end{tikzcd}
  \end{minipage}
\end{center}

\cor{普遍射に同型ならば普遍射}{
  $\bm{D}$から$\bm{C}$への関手$F$、$\bm{C}$の対象$a$について、$x, u$が$F$から$a$への普遍射であるとする。

  同型射$h \colon x^\prime \to x$が存在するとき、$x^\prime, u \circ F(h)$は普遍射である。
}

\lem{普遍射の一意性}{
  $\bm{D}$から$\bm{C}$への関手$F$、$\bm{C}$の対象$a$について、$x, u$と$x^\prime, u^\prime$が$F$から$a$への普遍射であるとする。

  このとき、以下を満たす。
  \eq*{
    \exists! h \in \Hom_{\bm{D}}(x^\prime, x) \qty(u^\prime = u \circ F(h))
  }

  さらに上で与える$h$は同型射である。
}{
  $x, u$は普遍射であるので、$x^\prime, u^\prime$について定義より、
  $\bm{D}$の射$g \colon x^\prime \to x$であって$u^\prime = u \circ F(g)$なる射が一意に存在する。

  同様に$x^\prime, u^\prime$は普遍射であるので、$x, u$について定義より、
  $\bm{D}$の射$g^\prime \colon x \to x^\prime$であって$u = u^\prime \circ F(g^\prime)$なる射が一意に存在する。

  ゆえに、$u = u^\prime \circ F(g^\prime) = u \circ F(g) \circ F(g^\prime) = u \circ F(g \circ g^\prime)$が成り立つ。

  $x, u$は普遍射であるので、$x, u$について定義より、
  $\bm{D}$の射$i \colon x \to x$であって$u = u \circ F(i)$なる射が一意に存在する。

  $x, g \circ g^\prime$は$i$の条件を満たすので、$i = x = g \circ g^\prime$

  同様に、$x^\prime = g^\prime \circ g$である。

  したがって、$g^\prime$は$g$の逆射である。
}

\dfn{終対象}{
  圏$\bm{C}$を考える。

  $\bm{C}$から圏$\bm{1}$への関手は一意に定まり、これを$U$とする。

  $U$から$\ast$への普遍射$x, u$が存在するならば、$u = \ast$となり、$x$を$\bm{C}$の終対象と呼ぶ。
}

\begin{center}
  \begin{minipage}{0.3\linewidth}
    $\bm{C}$ \\
    \begin{tikzcd}
      y \ar[r, "g", dotted, red] & x \\
    \end{tikzcd}
  \end{minipage}
  \begin{minipage}{0.3\linewidth}
    $\bm{1}$ \\
    \begin{tikzcd}
      \ast = U(y) \ar[r, "\ast = U(g)", dotted] \ar[dr, "\ast"'] & \ast = U(x) \ar[d, "\ast"] \\
      & \ast \\
    \end{tikzcd}
  \end{minipage}
\end{center}

\dfn{極限}{
  圏$\bm{C}, \bm{D}$と、$\bm{C}$から$\Func(\bm{D}, \bm{C})$への対角関手$\Delta$を考える。

  $\bm{D}$から$\bm{C}$への関手$X$について、$\Delta$から$X$への普遍射$x, u$が存在するならば、$x, u$を$X$の極限と呼ぶ。
}

\begin{center}
  \begin{minipage}{0.2\linewidth}
    $\bm{D}$ \\
    \begin{tikzcd}
      a \ar[r, "f"'] & b \\
    \end{tikzcd}
  \end{minipage}
  \begin{minipage}{0.3\linewidth}
    $\bm{C}$ \\
    \begin{tikzcd}
      & y \ar[d, "q", dotted, red] \ar[ddl, "p(a)"', bend right=45] \ar[rdd, "p(b)", bend left=45] & \\
      & x \ar[ld, "u(a)"', near start] \ar[rd, "u(b)", near start] & \\
      X(a) \ar[rr, "X(f)"'] & & X(b) \\
    \end{tikzcd}
  \end{minipage}
  \begin{minipage}{0.2\linewidth}
    $\Func(\bm{D}, \bm{C})$ \\
    \begin{tikzcd}
      \Delta(y) \ar[r, "\Delta(q)", dotted] \ar[dr, "p"'] & \Delta(x) \ar[d, "u"] \\
      & X \\
    \end{tikzcd}
  \end{minipage}
\end{center}

\dfn{積}{
  離散圏$\bm{J}$、圏$\bm{C}$と、$\bm{J}$から$\bm{C}$への関手$X$を考える。

  このとき、$X$の極限$x, u$が存在するならば、$x, u$を$X$の積と呼ぶ。

  特に、この$x$を$\prod X$と表す。
}

\begin{center}
  \begin{minipage}{0.3\linewidth}
    $\bm{C}$ \\
    \begin{tikzcd}
      y \ar[r, "q", dotted, red] \ar[dr, "p(j)"'] & \prod X \ar[d, "u(j)"] \\
      & X(j) \\
    \end{tikzcd}
  \end{minipage}
  \begin{minipage}{0.3\linewidth}
    $\Func(\bm{J}, \bm{C})$ \\
    \begin{tikzcd}
      \Delta(y) \ar[r, "\Delta(q)", dotted] \ar[dr, "p"'] & \Delta\qty(\prod X) \ar[d, "u"] \\
      & X \\
    \end{tikzcd}
  \end{minipage}
\end{center}

\dfn{引き戻し}{
  圏$\bm{C}$と、$\bm{C}$の射$f, g$を考える。
  $\cod(f) = \cod(g)$とする。

  以下で定める圏$\bm{\rightarrow \leftarrow}$から圏$\bm{C}$への関手$X$を考える。
  \eqg*{
    X(m_0) = f \land X(m_1) = g
  }

  このとき$X$の極限$x, u$が存在するならば、$x, \qty(u(\dom(m_0)), u(\dom(m_1)))$を$f, g$の引き戻しと呼ぶ。

  これは、$\qty(u(\dom(m_0)), u(\dom(m_1)))$が与えられれば自然性から$u$が与えられるからである。
}

\begin{center}
  \begin{minipage}{0.2\linewidth}
    $\bm{\rightarrow \leftarrow}$ \\
    \begin{tikzcd}
      & b^\prime \ar[d, "m_1"] \\
      a^\prime \ar[r, "m_0"'] & c^\prime
    \end{tikzcd}
  \end{minipage}
  \begin{minipage}{0.35\linewidth}
    $\bm{C}$ \\
    \begin{tikzcd}
      & y \ar[dd, "q", red, dotted] \ar[dddl, "p(a^\prime)"'] \ar[ddrr, "p(b^\prime)"] \ar[dddr, "p(c^\prime)"] & & \\
      & & & \\
      & x \ar[ld, "u(a^\prime)"'] \ar[rr, "u(b^\prime)"] \ar[rd, "u(c^\prime)"'] & & b \ar[ld, "g"] \\
      a \ar[rr, "f"'] & & c & \\
    \end{tikzcd}
  \end{minipage}
  \begin{minipage}{0.2\linewidth}
    $\Func(\bm{\rightarrow \leftarrow}, \bm{C})$ \\
    \begin{tikzcd}
      \Delta(y) \ar[r, "\Delta(q)", dotted] \ar[dr, "p"'] & \Delta(x) \ar[d, "u"] \\
      & \id
    \end{tikzcd}
  \end{minipage}
\end{center}

\dfn{核対}{
  圏$\bm{C}$と、$\bm{C}$の射$f$を考える。

  $f, f$の引き戻し$x, \qty(\pi_0, \pi_1)$が存在するならば、$x, \qty(\pi_0, \pi_1)$を$f$の核対と呼ぶ。
}

\cor*{
  $f$の核対$x, \qty(\pi_0, \pi_1)$が存在するとする。

  このとき、$\pi_0, \pi_1$はともに右可逆である。
}

\lem{核対と左簡約可能性}{
  圏$\bm{C}$と、$\bm{C}$の射$f$について、$f$の核対$R, \qty(\pi_0, \pi_1)$が存在するとする。

  このとき、以下の\num{3}つは同値である。
  \begin{enumerate}
    \item $f$は左簡約可能
    \item $\pi_0 = \pi_1$
    \item $\pi_0, \pi_1$はともに同型射
  \end{enumerate}
}{
  $1. \to 2.$を示す。

  核対より$f \circ \pi_0 = f \circ \pi_1$であり、左簡約可能より$\pi_0 = \pi_1$である。

  \vskip\baselineskip

  $2. \to 3.$を示す。

  $f \circ \dom(f) = f \circ \dom(f)$より、普遍性から射$u \colon \dom(f) \to R$が存在して、$\pi_0 \circ u = \dom(f) = \pi_1 \circ u$である。

  $\pi_0 = \pi_1$より、$\pi_1 = \pi_0 = \pi_1 \circ u \circ \pi_0 = \pi_0 \circ u \circ \pi_0$である。

  $f \circ \pi_0 = f \circ \pi_0$より、普遍性から一意な$e \colon R \to R$が存在して、$\pi_0 \circ e = \pi_0 = \pi_1 \circ e$である。

  $e = R$が成り立つため、一意性から$e = R = u \circ \pi_0$である。

  したがって、$\pi_0$は同型射である。

  \vskip\baselineskip

  $3. \to 2.$を示す。

  $f \circ \pi_0 = f \circ \pi_0$より、一意な射$e \colon R \to R$が存在して、$\pi_0 = \pi_0 \circ e = \pi_1 \circ e$である。

  $\pi_0$は同型射より、$e = R$である。ゆえに、$\pi_0 = \pi_1$である。

  \vskip\baselineskip

  $2. \to 1.$を示す。

  $\bm{C}$の射$g, h$について、$f \circ g = f \circ h$とする。

  普遍性より、一意な射$q \colon \dom(g) \to x$が存在して、$g = \pi_0 \circ q \land h = \pi_1 \circ q$である。

  $\pi_0 = \pi_1$より、$g = h$である。
}

\dfn{等化子}{
  圏$\bm{C}$と、$\bm{C}$の射$f, g$を考える。
  $\dom(f) = \dom(g) \land \cod(f) = \cod(g)$とする。

  以下で定める圏$\bm{\rightrightarrows}$から圏$\bm{C}$への関手$X$を考える。
  \eqg*{
    X(m_0) = f \land X(m_1) = g
  }

  このとき$X$の極限$x, u$が存在するならば、$x, u(\dom(m_0))$を$f, g$の等化子と呼ぶ。

  これは、$u(\dom(m_0))$が与えられれば自然性から$u$が与えられるからである。
}

\begin{center}
  \begin{minipage}{0.2\linewidth}
    $\bm{\rightrightarrows}$ \\
    \begin{tikzcd}
      a \ar[r, "m_0"', bend right] \ar[r, "m_1", bend left] & b^\prime \\
    \end{tikzcd}
  \end{minipage}
  \begin{minipage}{0.4\linewidth}
    $\bm{C}$ \\
    \begin{tikzcd}
      & y \ar[dd, "q", red, dotted] \ar[dddl, "p(a^\prime)"'] \ar[ddrr, "p(a^\prime)"] \ar[dddr, "p(b^\prime)"] & & \\
      & & & \\
      & \Eq(f, g) \ar[ld, "u(a^\prime)"'] \ar[rr, "u(a^\prime)"] \ar[rd, "u(b^\prime)"'] & & a \ar[ld, "g"] \\
      a \ar[rr, "f"'] & & b & \\
    \end{tikzcd}
  \end{minipage}
  \begin{minipage}{0.2\linewidth}
    $\Func(\bm{\rightrightarrows}, \bm{C})$ \\
    \begin{tikzcd}
      \Delta(y) \ar[r, "\Delta(q)", dotted] \ar[dr, "p"'] & \Delta\qty(\Eq(f, g)) \ar[d, "u"] \\
      & \id \\
    \end{tikzcd}
  \end{minipage}
\end{center}

\lem{等化子から得るモノ射}{
  $f$と$g$の等化子$\Eq(f, g), \pi$が存在するとする。

  このとき、$\pi$は左簡約可能である。
}{
  $\pi \circ h_1 = \pi \circ h_2$なる射$h_1, h_2 \colon y \to \Eq(f, g)$と、射$w \coloneqq \pi \circ h_1$を考える。

  このとき、$f \circ w = f \circ \pi \circ h_1 = g \circ \pi \circ h_1 = g \circ w$より、等化子の普遍性から$w = \pi \circ h$なる$h$が一意に存在する。

  ゆえに、$h_1 = h_2$である。

  したがって、$\pi$は左簡約可能である。
}

\thm{極限の存在定理}{
  以下を満たす圏$\bm{J}, \bm{C}$を考える。
  \begin{enumerate}
    \item $\Obj_{\bm{J}}$から$\bm{C}$への任意の関手は、積を持つ。
    \item $\bm{J}$の射を対象とする離散圏$\hat{\bm{J}}$について、$\hat{\bm{J}}$から$\bm{C}$への任意の関手は、積を持つ。
    \item $\bm{C}$の任意の射$f, g$について$\dom(f) = \dom(g) \land \cod(f) = \cod(g)$ならば、等化子を持つ。
  \end{enumerate}

  このとき、$\bm{J}$から$\bm{C}$への関手$X$は極限を持つ。
}{
  $\hat{\bm{J}}$から$\Obj_{\bm{J}}$への関手$c$、$c(\hat{f}) = \cod(f)$と、$\Obj_{\bm{J}}$から$\bm{J}$への自明な関手$I$を考える。

  $P = X \circ I$と、$R = X \circ I \circ c$を考える。
  $P$の積を$\prod P, \pi$、$R$の積を$\prod R, r$とする。

  $\bm{C}$から$\Func(\hat{\bm{J}}, \bm{C})$への対角関手$\Delta_1$を考える。

  自然変換$s \colon \Delta_1(\prod P) \to R$、$s(j) = \pi(\cod(j))$を考える。
  積の定義より、一意な射$\bar{s} \colon \prod P \to \prod R$が存在して、$s = r \circ \Delta_1(\bar{s})$である。

  自然変換$t \colon \Delta_1(\prod P) \to R$、$t(j) = X(j) \circ \pi(\dom(j))$を考える。
  積の定義より、一意な射$\bar{t} \colon \prod P \to \prod R$が存在して、$t = r \circ \Delta_1(\bar{t})$である。

  仮定より、$\bar{s}$と$\bar{t}$の等化子$\Eq(\bar{s}, \bar{t}), u$が存在する。

  \vskip\baselineskip

  $\bm{C}$から$\Func(\bm{J}, \bm{C})$への対角関手$\Delta_2$を考える。

  $\lambda \coloneqq \pi \circ \Delta_2(u)$が、$\Delta_2(\Eq(\bar{s}, \bar{t}))$から$X$への自然変換であることを示す。

  $\bm{J}$の射$j$について、等化子と積の定義より以下が成り立つ。
  \eqa*{
    X(j) \circ \lambda(\dom(j))
    &= X(j) \circ \pi(\dom(j)) \circ \Delta_2(u)(\dom(j)) \\*
    &= t(j) \circ u \\*
    &= r(j) \circ \bar{t} \circ u \\*
    &= r(j) \circ \bar{s} \circ u \\*
    &= s(j) \circ u \\*
    &= \pi(\cod(j)) \circ u \\*
    &= \lambda(\cod(j))
  }

  \vskip\baselineskip

  $\Eq(\bar{s}, \bar{t}), \lambda$が極限であることを示す。

  $\bm{C}$の対象$y$と、自然変換$p \colon \Delta_2(y) \to X$を考える。

  $\prod P$は積より、一意な$\bm{C}$の射$w \colon y \to \prod P$が存在して、$\forall j \in \Obj_{\bm{J}} \qty(p(j) = \pi(j) \circ w)$が成り立つ。

  等化子より、一意な$\bm{C}$の射$l \colon y \to \Eq(\bar{s}, \bar{t})$が存在して、$w = u \circ l$である。

  したがって、一意な$\bm{C}$の射$l$が存在して、$p = \lambda \circ \Delta_2(l)$である。
}

\begin{center}
  \begin{minipage}{0.35\linewidth}
    $\Func(\bm{\hat{J}}, \bm{C})$ \\
    \begin{tikzcd}
      & \Delta_1(\prod P) \ar[d, "\Delta_1(\bar{t})"'] \ar[rdd, "t"] & \\
      \Delta_1(\prod P) \ar[r, "\Delta_1(\bar{s})"] \ar[rrd, "s"'] & \Delta_1(\prod R) \ar[rd, "r", near start] \\
      & & R \\
    \end{tikzcd}
  \end{minipage}
  \begin{minipage}{0.5\linewidth}
    $\bm{C}$ \\
    \begin{tikzcd}
      \Eq(\bar{s}, \bar{t}) \ar[rr, "u"] \ar[dd, "u"'] & & \prod P \ar[dd, "\bar{t}"] \ar[rd, "\pi(\dom(j))"] \ar[rddd, "t(j)"'] & \\
      & & & X(\dom(j)) \ar[dd, "X(j)"] \\
      \prod P \ar[rr, "\bar{s}"] \ar[rrrd, "s(j) = \pi(\cod(j))"', near end] & & \prod R \ar[rd, "r(j)"] & \\
      & & & R(j) = X(\cod(j)) \\
    \end{tikzcd}
  \end{minipage}
\end{center}


\lsubsection{余普遍}

\dfn{余普遍射}{
  圏$\bm{C}, \bm{D}$と、$\bm{D}$から$\bm{C}$への関手$F$、$\bm{C}$の対象$a$を考える。

  $F^\opp$から$a$への普遍射$x, u$が存在するならば、$x, u^\opp$を$a$から$F$への余普遍射と呼ぶ。
}

\begin{center}
  \begin{minipage}{0.15\linewidth}
    $\bm{D}$ \\
    \begin{tikzcd}
      y & x \ar[l, "g"', dotted, red] \\
      &
    \end{tikzcd}
  \end{minipage}
  \begin{minipage}{0.15\linewidth}
    $\bm{D}^\opp$ \\
    \begin{tikzcd}
      y \ar[r, "g^\opp", dotted, red] & x \\
      &
    \end{tikzcd}
  \end{minipage}
  \begin{minipage}{0.3\linewidth}
    $\bm{C}^\opp$ \\
    \begin{tikzcd}
      F^\opp(y) \ar[rr, "F^\opp(g^\opp)", dotted] \ar[drr, "f"'] & & F^\opp(x) \ar[d, "u"] \\
      & & a
    \end{tikzcd}
  \end{minipage}
  \begin{minipage}{0.3\linewidth}
    $\bm{C}$ \\
    \begin{tikzcd}
      F(y) & & F(x) \ar[ll, "F(g)"', dotted] \\
      & & a \ar[u, "u^\opp"'] \ar[ull, "f^\opp"]
    \end{tikzcd}
  \end{minipage}
\end{center}

\dfn{始対象}{
  圏$\bm{C}$を考える。

  $\bm{C}^\opp$の終対象が存在するならば、これを$\bm{C}$の始対象と呼ぶ。
}

\dfn{余極限}{
  圏$\bm{C}, \bm{D}$と、$\bm{C}$から$\Func(\bm{D}, \bm{C})$への対角関手$\Delta$を考える。

  $\bm{D}$から$\bm{C}$への関手$X$について、$X$から$\Delta$への余普遍射が存在するならば、これを$X$の余極限と呼ぶ。
}

\dfn{和}{
  離散圏$\bm{J}$、圏$\bm{C}$と、$\bm{J}$から$\bm{C}$への関手$X$を考える。

  このとき、$X$の余極限が存在するならば、これを$X$の和と呼ぶ。
}

\dfn{押し出し}{
  圏$\bm{C}$と、$\bm{C}$の射$f, g$を考える。
  $\dom(f) = \dom(g)$とする。

  このとき、$f^\opp, g^\opp$の引き戻しが存在するならば、これを$f, g$の押し出しと呼ぶ。
}

\dfn{余等化子}{
  圏$\bm{C}$と、$\bm{C}$の射$f, g$を考える。
  $\dom(f) = \dom(g) \land \cod(f) = \cod(g)$とする。

  このとき、$f^\opp, g^\opp$の等化子が存在するならば、これを$f, g$の余等化子と呼ぶ。
}

\lem{余等化子から得るエピ射}{
  $f$と$g$の余等化子$x, u$について、$u$は右簡約可能である。
}{
  \lemref{等化子から得るモノ射}より、双対性から明らか。
}

\thm{余極限の存在定理}{
  以下を満たす圏$\bm{J}, \bm{C}$を考える。
  \begin{enumerate}
    \item $\Obj_{\bm{J}}$から$\bm{C}$への任意の関手は、和を持つ。
    \item $\bm{J}$の射を対象とする離散圏$\hat{\bm{J}}$について、$\hat{\bm{J}}$から$\bm{C}$への任意の関手は、和を持つ。
    \item $\bm{C}$の任意の射$f, g$について$\dom(f) = \dom(g) \land \cod(f) = \cod(g)$ならば、余等化子を持つ。
  \end{enumerate}

  このとき、$\bm{J}$から$\bm{C}$への関手$X$は余極限を持つ。
}{
  \thmref{極限の存在定理}より、双対性より明らか。
}


\lsubsection{随伴}

\dfn{左随伴関手}{
  圏$\bm{C}, \bm{D}$を考える。

  $\bm{D}$から$\bm{C}$への関手$F$について、$\bm{C}$から$\bm{D}$への関手$G$と、自然変換$\epsilon \colon F \circ G \to \id_{\bm{C}}$が存在して、以下を満たすとする。
  \eq*{
    \forall c \in \Obj_{\bm{C}} \qty(\text{$G(c), \epsilon(c)$は、$F$から$c$への普遍射である。})
  }

  このとき$F$を、右随伴が$G$で余単位を$\epsilon$とする左随伴関手と呼ぶ。
}

\lem{随伴の一意性}{
  圏$\bm{C}, \bm{D}$と、$\bm{D}$から$\bm{C}$への関手$F$、$\bm{C}$から$\bm{D}$への関手$G, G^\prime$と、
  自然変換$\epsilon \colon F \circ G \to \id_{\bm{C}}, \epsilon^\prime \colon F \circ G^\prime \to \id_{\bm{C}}$を考える。

  $F$は、右随伴が$G$で余単位を$\epsilon$とする左随伴関手であり、右随伴が$G^\prime$で余単位を$\epsilon^\prime$とする左随伴関手であるとする。

  このとき自然同型$\theta \colon G \to G^\prime$が存在して、以下が成り立つ。
  \eq*{
    \epsilon^\prime \circ \qty(F \ast \theta) = \epsilon
  }
}{
  $\forall c \in \Obj_{\bm{C}}$について、$G^\prime(c), \epsilon^\prime(c)$は$F$から$c$への普遍射である。

  ゆえに一意な$\bm{D}$の射$\alpha$が存在して、$\epsilon(c) = \epsilon^\prime(c) \circ F(\alpha)$が成り立つ。

  $c$に対して$\alpha$を与えるアリティ\num{1}の関数記号を、$\theta$とする。

  \vskip\baselineskip

  $\theta$が$G$から$G^\prime$への自然変換であることを示す。

  $\bm{C}$の射$f$を考える。
  $\epsilon, \epsilon^\prime$の自然性と、$\theta$の定義より以下が成り立つ。
  \eqa*{
    \epsilon^\prime(\cod(f)) \circ F\qty(\theta(\cod(f)) \circ G(f))
    &= \epsilon^\prime(\cod(f)) \circ \qty(F \ast \theta)(\cod(f)) \circ \qty(F \circ G)(f) \\*
    &= \epsilon(\cod(f)) \circ \qty(F \circ G)(f) \\*
    &= f \circ \epsilon(\dom(f)) \\*
    &= f \circ \epsilon^\prime(\dom(f)) \circ \qty(F \ast \theta)(\dom(f)) \\*
    &= \epsilon^\prime(\cod(f)) \circ \qty(F \circ G)(f) \circ \qty(F \ast \theta)(\dom(f)) \\*
    &= \epsilon^\prime(\cod(f)) \circ F\qty(G(f) \circ \theta(\dom(f)))
  }

  普遍性より、$\theta(\cod(f)) \circ G(f) = G(f) \circ \theta(\dom(f))$である。

  ゆえに、$\theta$は自然変換である。

  \vskip\baselineskip

  自然同型であることを示す。

  同様に自然変換$\zeta \colon G^\prime \to G$を考えることができる。

  定義より、$\forall c \in \bm{C}$について以下が成り立つ。
  \eqa*{
    \epsilon(c)
    &= \epsilon^\prime(c) \circ F(\theta(c)) \\*
    &= \epsilon(c) \circ F(\zeta(c)) \circ F(\theta(c)) \\*
    &= \epsilon(c) \circ F(\zeta(c) \circ \theta(c))
  }

  普遍性より、$\zeta(c) \circ \theta(c) = G(c)$である。

  すなわち、$\zeta$は$\theta$の逆射である。
}

\thm{三角恒等式}{
  圏$\bm{C}, \bm{D}$と、$\bm{D}$から$\bm{C}$への関手$F$、$\bm{C}$から$\bm{D}$への関手$G$、自然変換$\epsilon \colon F \circ G \to \id_{\bm{C}}$を考える。

  $F$は、右随伴が$G$で余単位を$\epsilon$とする左随伴関手であるとする。

  このとき、自然変換$\eta \colon \id_{\bm{D}} \to G \circ F$が存在して、以下を満たす。
  \eqg*{
    F = \qty(\epsilon \ast F) \circ \qty(F \ast \eta) \\*
    G = \qty(G \ast \epsilon) \circ \qty(\eta \ast G)
  }
}{
  まず第一式と、自然変換であることを示す。

  $d \in \Obj_{\bm{D}}$について、$\qty(G \circ F)(d), \qty(\epsilon \ast F)(d)$は$F$から$F(d)$への普遍射である。

  したがって、$d$と恒等射$F(d)$について、射$g$が一意に存在して、$F(d) = \qty(\epsilon \ast F)(d) \circ F(g)$である。

  $d$に対して、この$g$を与える関数記号を$\eta$とする。

  $\bm{D}$の射$f \colon d \to d^\prime$を考える。
  \eqa*{
    \qty(\epsilon \ast F)(d^\prime) \circ F\qty(\qty(G \circ F)(f) \circ \eta(d))
    &= \qty(\epsilon \ast F)(d^\prime) \circ \qty(F \circ G \circ F)(f) \circ \qty(F \ast \eta)(d) \\*
    &= F(f) \circ \qty(\epsilon \ast F)(d) \circ \qty(F \ast \eta)(d) \\*
    &= F(f) \\*
    &= \qty(\epsilon \ast F)(d^\prime) \circ \qty(F \ast \eta)(d^\prime) \circ F(f) \\*
    &= \qty(\epsilon \ast F)(d^\prime) \circ F\qty(\eta(d^\prime) \circ f)
  }

  $\qty(G \circ F)(d^\prime), \qty(\epsilon \ast F)(d^\prime)$は、$F$から$F(d^\prime)$への普遍射であるため、
  $d$と$F(f)$について、$F(f) = \qty(\epsilon \ast F)(d^\prime) \circ F(e)$なる$e$が一意に存在する。

  したがって、$e = \qty(G \circ F)(f) \circ \eta(d) = \eta(d^\prime) \circ f$である。

  ゆえに、$\eta$は自然変換である。

  \vskip\baselineskip

  第二式を示す。

  $c \in \Obj_{C}$を考える。

  $\epsilon$は自然変換であるので、射$w \colon \qty(F \circ G)(c) \to c$について$\epsilon(c) \circ \qty(F \circ G)(w) = w \circ \epsilon\qty(F \circ G(c))$である。

  $w = \epsilon(c)$として、$\epsilon(c) \circ \qty(F \circ G)(\epsilon(c)) = \epsilon(c) \circ \epsilon\qty(F \circ G(c))$である。

  第一式より、以下が成り立つ。
  \eqa*{
    \epsilon(c)
    &= \epsilon(c) \circ F(G(c)) \\*
    &= \epsilon(c) \circ \qty(\epsilon \ast F)(G(c)) \circ \qty(F \ast \eta)(G(c)) \\*
    &= \epsilon(c) \circ \epsilon\qty(\qty(F \circ G)(c)) \circ F \qty(\qty(\eta \ast G)(c)) \\*
    &= \epsilon(c) \circ \qty(F \circ G)\qty(\epsilon(c)) \circ F \qty(\qty(\eta \ast G)(c)) \\*
    &= \epsilon(c) \circ F \qty\big(\qty(\qty(G \ast \epsilon) \circ \qty(\eta \ast G))(c))
  }

  $G(c), \epsilon(c)$は$F$から$c$への普遍射であるため、
  $G(c)$と$\epsilon(c) \colon F(G(c)) \to c$について、一意な射$i \colon G(c) \to G(c)$が存在して、$\epsilon(c) = \epsilon(c) \circ F(i)$である。

  $i = G(c)$は明らかに条件を満たすので、一意性より以下を得る。
  \eq*{
    \qty(\qty(G \ast \epsilon) \circ \qty(\eta \ast G))(c) = G(c)
  }
}

\begin{center}
  \begin{minipage}{0.35\linewidth}
    $\bm{D}$ \\
    \begin{tikzcd}
      d \ar[rr, "\eta(d)", dotted] \ar[d, "f"'] & & \qty(G \circ F)(d) \ar[d, "\qty(G \circ F)(f)"] \\
      d^\prime \ar[rr, "\eta(d^\prime)", dotted] & & \qty(G \circ F)(d^\prime) \\
    \end{tikzcd}
  \end{minipage}
  \begin{minipage}{0.5\linewidth}
    $\bm{C}$ \\
    \begin{tikzcd}
      F(d) \ar[rr, "\qty(F \ast \eta)(d)"] \ar[rrrd, "F(d)"'] \ar[dd, "F(f)"'] & & \qty(F \circ G \circ F)(d) \ar[rd, "\qty(\epsilon \ast F)(d)"] \ar[dd, "\qty(F \circ G \circ F)(f)"] & \\
      & & & F(d) \ar[dd, "F(f)"] \\
      F(d^\prime) \ar[rr, "\qty(F \ast \eta)(d^\prime)"] \ar[rrrd, "F(d^\prime)"'] & & \qty(F \circ G \circ F)(d^\prime) \ar[rd, "\qty(\epsilon \ast F)(d^\prime)"] & \\
      & & & F(d^\prime) \\
    \end{tikzcd}
  \end{minipage}
\end{center}

\dfn{右随伴関手}{
  圏$\bm{C}, \bm{D}$と、$\bm{D}$から$\bm{C}$への関手$F$、$\bm{C}$から$\bm{D}$への関手$G$、自然変換$\eta \colon \id_{\bm{D}} \to G \circ F$を考える。

  $G^\opp$が、右随伴が$F^\opp$で余単位を$\eta^\opp$とする左随伴関手であるとする。

  このとき$G$を、左随伴が$F$で単位を$\eta$とする右随伴関手と呼ぶ。
}

\lem{三角恒等ならば随伴}{
  圏$\bm{C}, \bm{D}$を考える。

  $\bm{D}$から$\bm{C}$への関手$F$、$\bm{C}$から$\bm{D}$への関手$G$と、
  自然変換$\epsilon \colon F \circ G \to \id_{\bm{C}}, \eta \colon \id_{\bm{D}} \to G \circ F$について、以下を満たすとする。
  \eqg*{
    F = \qty(\epsilon \ast F) \circ \qty(F \ast \eta) \\*
    G = \qty(G \ast \epsilon) \circ \qty(\eta \ast G)
  }

  このとき、$F$は右随伴が$G$で余単位を$\epsilon$とする左随伴関手となり、$G$は左随伴が$F$で単位を$\eta$とする右随伴関手となる。
}{
  双対性から、$F$が左随伴関手であることを示せば十分である。

  \vskip\baselineskip

  $\bm{C}$の任意の対象$a$について、$G(a), \epsilon(a)$が$F$から$a$への普遍射となることを示す。

  $\bm{D}$の対象$y$と、$\bm{C}$の射$f \colon F(y) \to a$を考える。

  このとき、$g \coloneqq G(f) \circ \eta(y)$を考えると、$\epsilon$の自然性と$F = \qty(\epsilon \ast F) \circ \qty(F \ast \eta)$より、以下が成り立つ。
  \eqa*{
    \epsilon(a) \circ F(g)
    &= \epsilon(a) \circ \qty(F \circ G)(f) \circ \qty(F \ast \eta)(y) \\*
    &= f \circ \qty(\epsilon \ast F)(y) \circ \qty(F \ast \eta)(y) \\*
    &= f \circ F(y) \\*
    &= f
  }

  \vskip\baselineskip

  次に、$\bm{D}$の射$g^\prime \colon y \to G(a)$が存在して、$f = \epsilon(a) \circ F(g^\prime)$であるとする。

  このとき、$G = \qty(G \ast \epsilon) \circ \qty(\eta \ast G)$と$\eta$の自然性より、以下が成り立つ。
  \eqa*{
    g^\prime
    &= G(a) \circ g^\prime \\*
    &= \qty(G \ast \epsilon)(a) \circ \qty(\eta \ast G)(a) \circ g^\prime \\*
    &= \qty(G \ast \epsilon)(a) \circ \qty(G \circ F)(g^\prime) \circ \eta(y) \\*
    &= G \qty(\epsilon(a) \circ F(g^\prime)) \circ \eta(y) \\*
    &= G(f) \circ \eta(y) \\*
    &= g
  }

  ゆえに一意である。
}

\begin{center}
  \begin{minipage}{0.45\linewidth}
    $\bm{D}$ \\
    \begin{tikzcd}
      & y \ar[dd, "\eta(y)"] \ar[rrdd, "g = G(f) \circ \eta(y)", red, dotted] & & \\
      & & & \\
      & \qty(G \circ F)(y) \ar[rr, "G(f)"] & & G(a) \\
      & & & \\
    \end{tikzcd}
  \end{minipage}
  \begin{minipage}{0.45\linewidth}
    $\bm{C}$ \\
    \begin{tikzcd}
      & F(y) \ar[dd, "\qty(F \ast \eta)(y)"] \ar[dddl, "F(y)"'] \ar[dddr, "f"] \ar[rrdd, "F(g)", dotted] & & \\
      & & & \\
      & \qty(F \circ G \circ F)(y) \ar[ld, "\qty(\epsilon \ast F)(y)"] \ar[rr, "\qty(F \circ G)(f)"] & & \qty(F \circ G)(a) \ar[ld, "\epsilon(a)"] \\
      F(y) \ar[rr, "f"'] & & a & \\
    \end{tikzcd}
  \end{minipage}
\end{center}

\lem{圏同値ならば随伴}{
  圏$\bm{C}, \bm{D}$について、$\bm{C}$と$\bm{D}$が圏同値であるとする。

  \dfnref{圏同値}の主張する$F, G, \epsilon, \theta$について、$F$は、$G$を右随伴で$\epsilon$を余単位とする左随伴関手である。
}{
  ???
}

\thm{左随伴は余極限を保存する}{
  圏$\bm{J}, \bm{C}, \bm{D}$と、$\bm{J}$から$\bm{D}$への関手$X$、$\bm{D}$から$\bm{C}$への関手$F$、$\bm{C}$から$\bm{D}$への関手$G$と、
  自然変換$\epsilon \colon F \circ G \to \id_{\bm{C}}, \eta \colon \id_{\bm{D}} \to G \circ F$を考える。

  $F$は右随伴が$G$で余単位を$\epsilon$とする左随伴関手であるとする。

  $x, u$が$X$の余極限であるならば、$F(x), F \ast u$は$F \circ X$の余極限である。
}{
  $\bm{D}$から$\Func(\bm{J}, \bm{D})$への対角関手を$\Delta_{\bm{D}}$、$\bm{C}$から$\Func(\bm{J}, \bm{C})$への対角関手を$\Delta_{\bm{C}}$とする。

  \vskip\baselineskip

  $\bm{C}$の対象$y$と、$F \circ X$から$\Delta_{\bm{C}}(y)$への自然変換$f$を考える。

  $G(y), \epsilon(y)$は、$F$から$y$への普遍射である。

  したがって$\bm{J}$の任意の対象$j$について、$\epsilon(y) \circ F(w(j)) = f(j)$を満たす射$w(j)$が一意に存在する。

  ここで、$f$は$F \circ X$から$\Delta_{\bm{C}}(y)$への自然性であるので、$\bm{J}$の任意の対象$j, j^\prime$について以下が成り立つ。
  \eqa*{
    \epsilon(y) \circ F(w(j^\prime) \circ X(i))
    &= \epsilon(y) \circ F(w(j^\prime)) \circ \qty(F \circ X)(i) \\*
    &= f(j^\prime) \circ \qty(F \circ X)(i) \\*
    &= f(j) \\*
    &= \epsilon(y) \circ F(w(j))
  }

  一意性より、$w(j^\prime) \circ X(i) = w(j)$である。
  よって、$w$は$X$から$\Delta_{\bm{D}}$への自然変換である。

  $x, u$は$X$から$\Delta_{\bm{D}}$への余普遍射であるので、$w = \Delta_{\bm{D}}(h) \circ u$を満たす$\bm{D}$の射$h$が一意に存在する。

  ここで、$g \coloneqq \epsilon(y) \circ F(h)$とすると、$\bm{J}$の任意の対象$j$について以下が成り立つ。
  \eqa*{
    g \circ \qty(F \ast u)(j)
    &= \epsilon(y) \circ F(h) \circ \qty(F \ast u)(j) \\*
    &= \epsilon(y) \circ F(h \circ u(j)) \\*
    &= \epsilon(y) \circ \qty(F \ast w)(j) \\*
    &= f(j)
  }

  したがって、$\Delta_{\bm{C}}(g) \circ \qty(F \ast u) = f$である。

  \vskip\baselineskip

  次に、$\bm{C}$の射$g^\prime$が存在して、$\Delta_{\bm{C}}(g^\prime) \circ \qty(F \ast u) = f$であるとする。

  $G(y), \epsilon(y)$は$F$から$y$への普遍射であるため、$g^\prime = \epsilon(y) \circ F(h^\prime)$を満たす$\bm{D}$の射$h^\prime$が一意に存在する。

  $\bm{J}$の任意の対象$j$について、以下が成り立つ。
  \eqa*{
    \epsilon(y) \circ F\qty(h \circ u(j))
    &= f(j) \\*
    &= g^\prime(j) \circ \qty(F \ast u)(j) \\*
    &= \epsilon(y) \circ F(h^\prime) \circ \qty(F \ast u)(j) \\*
    &= \epsilon(y) \circ F\qty(h^\prime \circ u(j))
  }

  再び$G(y), \epsilon(y)$は$F$から$y$への普遍射であるため、一意性より$h \circ u(j) = h^\prime \circ u(j)$である。
  ゆえに、$\Delta_{\bm{D}}(h^\prime) \circ u = \Delta_{\bm{D}}(h) \circ u$である。

  $x, u$は、$X$の余極限であるため、一意性より$h = h^\prime$である。
  したがって、$g = g^\prime$である。
}

\begin{center}
  \begin{minipage}{0.25\linewidth}
    $\bm{J}$ \\
    \begin{tikzcd}
      & & \\
      j \ar[rr, "i"] & & j^\prime \\
    \end{tikzcd}
  \end{minipage}
  \begin{minipage}{0.3\linewidth}
    $\bm{D}$ \\
    \begin{tikzcd}
      G(y) & & \\
      & x \ar[lu, "h"'] & \\
      X(j) \ar[uu, "w(j)"] \ar[rr, "X(i)"'] \ar[ru, "u(j)"] & & X(j^\prime) \ar[lu, "u(j^\prime)"'] \ar[lluu, "w(j^\prime)"', bend right=50] \\
    \end{tikzcd}
  \end{minipage}
  \begin{minipage}{0.3\linewidth}
    $\bm{C}$ \\
    \begin{tikzcd}
      & \qty(F \circ G)(y) \ar[ld, "\epsilon(y)"'] & \\
      y & & F(x) \ar[lu, "F(h)"'] \ar[ll, "g"', red, dotted, near start] \\
      & \qty(F \circ X)(j) \ar[lu, "f(j)"] \ar[uu, "\qty(F \ast w)(j)", near start] \ar[ru, "\qty(F \ast u)(j)"'] & \\
    \end{tikzcd}
  \end{minipage}
\end{center}

\begin{center}
  \begin{minipage}{0.35\linewidth}
    $\Func(\bm{J}, \bm{D})$ \\
    \begin{tikzcd}
      \Delta_{\bm{D}}\qty(G(y)) & \\
      & \Delta_{\bm{D}}(x) \ar[lu, "\Delta_{\bm{D}}(h)"'] \\
      X \ar[ru, "u"'] \ar[uu, "w"] & \\
    \end{tikzcd}
  \end{minipage}
  \begin{minipage}{0.35\linewidth}
    $\Func(\bm{J}, \bm{C})$ \\
    \begin{tikzcd}
      & \Delta_{\bm{C}}\qty(\qty(F \circ G)(y)) \ar[ld, "\Delta_{\bm{C}}\qty(\epsilon(y))"'] & \\
      \Delta_{\bm{C}}(y) & & \Delta_{\bm{C}}\qty(F(x)) \ar[lu, "\Delta_{\bm{C}}\qty(F(h))"'] \ar[ll, "\Delta_{\bm{C}}(g)"', dotted, near start] \\
      & F \circ X \ar[lu, "f"] \ar[uu, "F \ast w", near start] \ar[ru, "F \ast u"'] & \\
    \end{tikzcd}
  \end{minipage}
\end{center}

\lsection{圏論}

\lsubsection{射の公理}

\rem{対象と恒等射の同一視}{
  本ノートでは、対象を恒等射と同一視するという、少数派の視点をとることに注意されたい。
}

\dfn{射}{
  圏論では、\num{6}つの公理(
    \axiref{始域は冪等}、\axiref{終域は冪等}、\axiref{対象}、\axiref{合成射の始域と終域}、\axiref{合成射の結合}、\axiref{対象は恒等射}
  )が与えられる。

  項を射と呼ぶ。
}

\dfn{始域}{
  アリティ\num{1}の関数記号$\dom()$を考える。

  射$f$について、$\dom(f)$を$f$の始域と呼ぶ。
}

\axi{始域は冪等}{
  \eq*{
    \forall f \qty(\dom(\dom(f)) = \dom(f))
  }
}

\dfn{終域}{
  アリティ\num{1}の関数記号$\cod()$を考える。

  射$f$について、$\cod(f)$を$f$の終域と呼ぶ。
}

\axi{終域は冪等}{
  \eq*{
    \forall f \qty(\cod(\cod(f)) = \cod(f))
  }
}

\axi{対象}{
  \eq*{
    \forall f \qty(f = \dom(f) \leftrightarrow f = \cod(f))
  }
}

\dfn{合成}{
  アリティ\num{2}の関数記号$\circ \qty()$を考える。

  $\circ \qty(f, g)$を、簡単のために$g \circ f$で表す。

  射$f, g$について、$g \circ f$を$f$と$g$の合成と呼ぶ。
}

\dfn{可合成}{
  アリティ\num{2}の述語記号$\Comp()$を、以下で定める。
  \eq*{
    \Comp(f, g) \defiff \cod(f) = \dom(g)
  }
}

\axi{合成射の始域と終域}{
  \eq*{
    \forall f, g \qty(\Comp(f, g) \rightarrow \dom(g \circ f) = \dom(f) \land \cod(g \circ f) = \cod(g))
  }
}

\axi{合成射の結合}{
  \eq*{
    \forall f, g, h \qty(\Comp(f, g) \land \Comp(g, h) \rightarrow h \circ \qty(g \circ f) = \qty(h \circ g) \circ f)
  }
}

\axi{対象は恒等射}{
  \eq*{
    \forall f \qty(f = f \circ \dom(f) \land f = \cod(f) \circ f)
  }
}

\dfn{対象}{
  以下を満たす射$a$を、恒等射、または、対象と呼ぶ。
  \eq*{
    a = \dom(a)
  }
}


\lsubsection{双対}

\dfn{反対射}{
  射$f$について、以下を満たすアリティ\num{1}の関数記号$\opp$を考える。
  簡単のために、$\opp(f)$を$f^\opp$で表す。
  \eqg*{
    \forall f \qty(\qty(f^\opp)^\opp = f) \\*
    \forall f \qty(\dom(f^\opp) = \cod(f)^\opp) \\*
    \forall f \qty(\cod(f^\opp) = \dom(f)^\opp) \\*
    \forall f, g \qty(\Comp(f, g) \rightarrow f^\opp \circ g^\opp = \qty(g \circ f)^\opp)
  }
}

\dfn{反対圏}{
  圏$\bm{C}$について、以下で定める類$\bm{C}^\opp$は圏である。
  \eq*{
    f \in \bm{C}^\opp \defiff f^\opp \in \bm{C}
  }
}

\dfn{反対関手}{
  $\bm{C}$から$\bm{D}$への関手$F$について、以下で定める$\bm{C}^\opp$から$\bm{D}^\opp$への関手を、$F^\opp$で表す。
  \eq*{
    F^\opp(f) \coloneqq \qty(F(f^\opp))^\opp
  }
}

\rem{双対性}{
  圏論における命題は、$\dom()$と$\cod()$を入れ替えて、合成$g \circ f$を$f \circ g$に置き換えても、その真偽は変わらない。

  より簡単に言えば、$\bm{C}$で成り立つ命題は、$\bm{C}^\opp$でも成り立つ。
}


\lsubsection{圏と関手}

\dfn{圏}{
  類$\bm{C}$が圏であるとは、以下を満たすことである。
  \eqg*{
    \forall f \in \bm{C} \qty(\dom(f) \in \bm{C} \land \cod(f) \in \bm{C}) \\*
    \forall f, g \in \bm{C} \qty(\Comp(f, g) \rightarrow g \circ f \in \bm{C})
  }
}

\dfn{対象の類}{
  以下で定める類$\Obj_{\bm{C}}$を、圏$\bm{C}$の対象の類と呼ぶ。
  \eq*{
    f \in \Obj_{\bm{C}} \defiff f \in \bm{C} \land f = \dom(f)
  }
}

\dfn{離散圏}{
  圏$\bm{C}$が離散であるとは、以下を満たすことである。
  \eq*{
    \forall f \qty(f \in \bm{C} \leftrightarrow f \in \Obj_{\bm{C}})
  }

  このとき、$\bm{C}$を離散圏と呼ぶ。
}

\dfn{Hom類}{
  圏$\bm{C}$と、$\bm{C}$の対象$a, b$について、類$\Hom_{\bm{C}}(a, b)$を以下で定める。
  \eq*{
    f \in \Hom_{\bm{C}}(a, b) \defiff f \in \bm{C} \land \dom(f) = a \land \cod(f) = b
  }
}

\rem{射の域の明示}{
  $f \in \Hom_{\bm{C}}(a, b)$を、簡単のため、$\bm{C}$の射$f \colon a \to b$、と表記する。
}

\dfn{End類}{
  圏$\bm{C}$と、$\bm{C}$の対象$a$について、類$\End_{\bm{C}}(a)$を以下で定める。
  \eq*{
    f \in \End_{\bm{C}}(a) \defiff f \in \Hom_{\bm{C}}(a, a)
  }
}

\dfn{関手}{
  アリティ\num{1}の関数記号$F$が、圏$\bm{C}$から圏$\bm{D}$への関手であるとは、以下の\num{3}つを満たすことである。
  \eqg*{
    \forall f \in \bm{C} \qty(F(f) \in \bm{D}) \\*
    \forall f \in \bm{C} \qty\Big(\dom(F(f)) = F(\dom(f)) \land \cod(F(f)) = F(\cod(f))) \\*
    \forall f, g \in \bm{C} \qty\Big(\Comp(f, g) \rightarrow F(g \circ f) = F(g) \circ F(f))
  }
}

\cor*{
  圏$\bm{C}, \bm{D}$について、$\bm{C}$から$\bm{D}$への関手$F$を考える。
  このとき、以下が成り立つ。
  \eq*{
    \forall a \in \Obj_{\bm{C}} \qty(a = \dom(a) \rightarrow F(a) = \dom(F(a)))
  }
}

\cor*{
  圏$\bm{C}, \bm{D}$について、$\bm{C}$から$\bm{D}$への関手$F$を考える。
  このとき、以下が成り立つ。
  \eq*{
    \forall f \in \bm{C} \qty(\text{$f$が同型射} \rightarrow \text{$F(f)$が同型射})
  }
}

\dfn{関手の相等}{
  圏$\bm{C}$から圏$\bm{D}$への関手$F, F^\prime$について、以下を満たすとき、$F = F^\prime$と書く。
  \eq*{
    \forall f \in \bm{C} \qty(F(f) = F^\prime(f))
  }
}

\dfn{関手の合成}{
  圏$\bm{C}, \bm{D}, \bm{E}$を考える。

  $\bm{C}$から$\bm{D}$への関手$F$と、$\bm{D}$から$\bm{E}$への関手$G$について、以下で示すアリティ\num{1}の関数記号$H$は、$\bm{C}$から$\bm{E}$への関手である。
  \eq*{
    H(f) \coloneqq G(F(f))
  }

  この$H$を、$F$と$G$の合成と呼び、$G \circ F$で表す。
}

\dfn{恒等関手}{
  圏$\bm{C}$について、関数記号$\id$は、$\bm{C}$から$\bm{C}$への関手である。

  圏論において、$\id$を恒等関手と呼ぶ。
  また、$\bm{C}$から$\bm{C}$への恒等関手であることを、明示的に$\id_{\bm{C}}$で表す。
}

\dfn{圏同型}{
  圏$\bm{C}, \bm{D}$を考える。

  $\bm{C}$から$\bm{D}$への関手$F$と、$\bm{D}$から$\bm{C}$への関手$G$が存在して、以下が成り立つとき、$\bm{C}$と$\bm{D}$は圏同型であると言う。
  \eq*{
    G \circ F = \id_{\bm{C}} \land F \circ G = \id_{\bm{D}}
  }
}


\lsubsection{簡約と逆}

\dfn{左簡約可能}{
  圏$\bm{C}$と、$\bm{C}$の射$f$について、以下を満たすとき、$f$は$\bm{C}$において左簡約可能であるという。
  \eq*{
    \forall g, h \in \bm{C} \qty(\Comp(g, f) \land \Comp(h, f) \land f \circ g = f \circ h \rightarrow g = h)
  }

  左簡約可能な射を単射と呼ぶ。
}

\dfn{右簡約可能}{
  圏$\bm{C}$と、$\bm{C}$の射$f$について、$f^\opp$が$\bm{C}^\opp$で左簡約可能であるとき、$f$は$\bm{C}$において右簡約可能であるという。

  右簡約可能な射を全射と呼ぶ。
}

\dfn{左可逆}{
  圏$\bm{C}$と、$\bm{C}$の射$f$について、以下を満たすとき、$f$は$\bm{C}$において左可逆であるという。
  \eq*{
    \exists g \in \bm{C} \qty(\Comp(g, f) \land g \circ f = \dom(f))
  }
}

\cor*{
  圏$\bm{C}$について、$\bm{C}$において左可逆な射は、$\bm{C}$において左簡約可能である。
}

\dfn{右可逆}{
  圏$\bm{C}$と、$\bm{C}$の射$f$について、$f^\opp$が$\bm{C}^\opp$で左可逆であるとき、$f$は$\bm{C}$において右可逆であるという。
}

\cor*{
  圏$\bm{C}$について、$\bm{C}$において右可逆な射は、$\bm{C}$において右簡約可能である。
}

\cor*{
  圏$\bm{C}$と、$\bm{C}$の射$f, g$について、$\Comp(f, g)$であるとする。
  このとき、以下が成り立つ。
  \begin{enumerate}
    \item $f, g$がともに$\bm{C}$において左簡約可能ならば、$g \circ f$は$\bm{C}$において左簡約可能である。
    \item $f, g$がともに$\bm{C}$において右簡約可能ならば、$g \circ f$は$\bm{C}$において右簡約可能である。
    \item $f, g$がともに$\bm{C}$において左可逆ならば、$g \circ f$は$\bm{C}$において左可逆である。
    \item $f, g$がともに$\bm{C}$において右可逆ならば、$g \circ f$は$\bm{C}$において右可逆である。
  \end{enumerate}
}

\cor*{
  圏$\bm{C}$と、$\bm{C}$の射$f, g$について、$\Comp(f, g)$であるとする。
  このとき、以下が成り立つ。
  \begin{enumerate}
    \item $g \circ f$が$\bm{C}$において左簡約可能ならば、$f$は$\bm{C}$において左簡約可能である。
    \item $g \circ f$が$\bm{C}$において右簡約可能ならば、$g$は$\bm{C}$において右簡約可能である。
  \end{enumerate}
}

\dfn{逆射}{
  射$f$について、以下を満たす射$g$を$f$の逆射と呼ぶ。
  \eq*{
    \Comp(f, g) \land \Comp(g, f) \land g \circ f = \dom(f) \land f \circ g = \cod(f)
  }
}

\lem{逆射の一意性}{
  射$g, h$が$f$の逆射であるとき、$g = h$である。
}{
  以下より成り立つ。

  $g = g \circ \dom(g) = g \circ \cod(f) = g \circ f \circ h = \dom(f) \circ h = \cod(h) \circ h = h$より、成り立つ。
}

\rem{逆射}{
  $f$の逆射は、\lemref{逆射の一意性}より一意に定まるので、これを$f^{-1}$と表す。
}

\cor*{
  逆射を持つ射$f$について、以下が成り立つ。
  \eq*{
    \qty(f^{-1})^{-1} = f
  }
}

\lem{同型射}{
  圏$\bm{C}$と、$\bm{C}$の射$f$について、以下の\num{2}つは同値である。
  \begin{enumerate}
    \item $f$は、$\bm{C}$に逆射を持つ。
    \item $f$は、$\bm{C}$において、左可逆かつ右可逆である。
  \end{enumerate}
}{
  $1. \to 2.$は明らか。

  \vskip\baselineskip

  $2. \to 1.$を示す。

  左可逆より、射$g$が存在して$g \circ f = \dom(f)$であり、右可逆より、射$h$が存在して$f \circ h = \cod(f)$である。

  $g = g \circ (f \circ h) = \qty(g \circ f) \circ h = h$であるので、$g = h$

  ゆえに、$g$は$f \circ g = \cod(f)$を満たす。
}

\dfn{同型射}{
  圏$\bm{C}$について、$\bm{C}$において左可逆かつ右可逆な射を、$\bm{C}$の同型射と呼ぶ。
}

\cor{恒等射は同型射}{
  圏$\bm{C}$について、$\bm{C}$の恒等射$f$の逆射は自身であり、ゆえに$f$は$\bm{C}$の同型射である。
}


\lsubsection{自然変換}

\dfn{自然変換}{
  圏$\bm{C}, \bm{D}$と、$\bm{C}$から$\bm{D}$への関手$F, F^\prime$を考える。

  アリティ\num{1}の関数記号$\eta$が、$F$から$F^\prime$への自然変換であるとは、以下の\num{3}つ全てを満たすことである。
  \eqg*{
    \forall a \in \Obj_{\bm{C}} \qty(\eta(a) \in \bm{D}) \\*
    \forall f \in \bm{C} \qty\Big(\Comp(\eta(\dom(f)), F^\prime(f)) \land \Comp(F(f), \eta(\cod(f)))) \\*
    \forall f \in \bm{C} \qty\Big(F^\prime(f) \circ \eta(\dom(f)) = \eta(\cod(f)) \circ F(f))
  }
}

\begin{center}
  \begin{minipage}{0.15\linewidth}
    $\bm{C}$ \\
    \begin{tikzcd}
      a \ar[r, "f"] & b \\
    \end{tikzcd}
  \end{minipage}
  \begin{minipage}{0.25\linewidth}
    $\bm{D}$ \\
    \begin{tikzcd}
      F(a) \ar[r, "F(f)"] \ar[d, "\eta(a)"'] & F(b) \ar[d, "\eta(b)"] \\
      F^\prime(a) \ar[r, "F^\prime(f)"'] & F^\prime(b) \\
    \end{tikzcd}
  \end{minipage}
\end{center}

\dfn{自然変換の相等}{
  圏$\bm{C}$から圏$\bm{D}$への関手$F, F^\prime$を考える。

  $F$から$F^\prime$への自然変換$\eta, \theta$について、以下を満たすとき、$\eta = \theta$と書く。
  \eq*{
    \forall a \in \Obj_{\bm{C}} \qty(\eta(a) = \theta(a))
  }
}

\dfn{恒等な自然変換}{
  圏$\bm{C}, \bm{D}$と、$\bm{C}$から$\bm{D}$への関手$F$について、関数記号$F$は、$F$から$F$への自然変換である。

  $F$が自然変換であることを明示的に、関手$F$に対する恒等な自然変換$F$と呼ぶ。
}

\dfn{自然同型}{
  圏$\bm{C}, \bm{D}$と、$\bm{C}$から$\bm{D}$への関手$F, F^\prime$を考える。

  $F$から$F^\prime$への自然変換$\eta$が自然同型であるとは、以下を満たすことである。
  \eq*{
    \forall a \in \Obj_{\bm{C}} \qty(\text{$\eta(a)$は同型射})
  }
}

\cor{恒等な自然変換は自然同型}{
  関手$F$に対する恒等な自然変換$F$は、$F$から$F$への自然同型である。
}

\dfn{自然変換の垂直合成}{
  圏$\bm{C}, \bm{D}$と、$\bm{C}$から$\bm{D}$への関手$F, F^\prime, F^{\prime \prime}$を考える。

  $F$から$F^\prime$への自然変換$\eta$と$F^\prime$から$F^{\prime \prime}$への自然変換$\theta$について、
  以下で定義するアリティ\num{1}の関数記号$\theta \circ \eta$は$F$から$F^{\prime \prime}$への自然変換である。
  \eq*{
    \qty(\theta \circ \eta)(a) \coloneqq \theta(a) \circ \eta(a)
  }

  $\theta \circ \eta$を、$\eta$と$\theta$の垂直合成と呼ぶ。
}

\begin{center}
  \begin{minipage}{0.15\linewidth}
    $\bm{C}$ \\
    \begin{tikzcd}
      a \ar[r, "f"] & b \\
    \end{tikzcd}
  \end{minipage}
  \begin{minipage}{0.25\linewidth}
    $\bm{D}$ \\
    \begin{tikzcd}
      F(a) \ar[r, "F(f)"] \ar[d, "\eta(a)"'] \ar[dd, bend right=60, red, "\qty(\theta \circ \eta)(a)"'] & F(b) \ar[d, "\eta(b)"] \ar[dd, bend left=60, red, "\qty(\theta \circ \eta)(b)"] \\
      F^\prime(a) \ar[r, "F^\prime(f)"'] \ar[d, "\theta(a)"'] & F^\prime(b) \ar[d, "\theta(b)"] \\
      F^{\prime \prime}(a) \ar[r, "F^{\prime \prime}(f)"'] & F^{\prime \prime}(b) \\
    \end{tikzcd}
  \end{minipage}
\end{center}

\dfn{自然変換の水平合成}{
  圏$\bm{C}, \bm{D}, \bm{E}$と、$\bm{C}$から$\bm{D}$への関手$F, F^\prime$、$\bm{D}$から$\bm{E}$への関手$G, G^\prime$を考える。

  $F$から$F^\prime$への自然変換$\eta$と、$G$から$G^\prime$への自然変換$\theta$について、以下で定義するアリティ\num{1}の関数記号$\theta \ast \eta$は$G \circ F$から$G^\prime \circ F^\prime$への自然変換である。
  \eq*{
    \qty(\theta \ast \eta)(a) \coloneqq \theta(F^\prime(a)) \circ G(\eta(a))
  }

  $\theta \ast \eta$を、$\eta$と$\theta$の水平合成と呼ぶ。
}

\begin{center}
  \begin{minipage}{0.15\linewidth}
    $\bm{C}$ \\
    \begin{tikzcd}
      a \ar[r, "f"] & b \\
    \end{tikzcd}
  \end{minipage}
  \begin{minipage}{0.25\linewidth}
    $\bm{D}$ \\
    \begin{tikzcd}
      F(a) \ar[r, "F(f)"] \ar[d, "\eta(a)"'] & F(b) \ar[d, "\eta(b)"] \\
      F^\prime(a) \ar[r, "F^\prime(f)"'] & F^\prime(b) \\
    \end{tikzcd}
  \end{minipage}
  \begin{minipage}{0.5\linewidth}
    $\bm{E}$ \\
    \begin{tikzcd}
      & G^\prime(F(a)) \ar[rr, "G^\prime(F(f))"] \ar[dd] & & G^\prime(F(b)) \ar[dd, "G^\prime(\eta(b))"] \\
      G(F(a)) \ar[rr, crossing over, "G(F(f))", near end] \ar[dd, "G(\eta(a))"'] \ar[ur, "\theta(F(a))"] \ar[rd, "\qty(\theta \ast \eta)(a)", red] & & G(F(b)) \ar[ur, "\theta(F(b))"] \ar[rd, "\qty(\theta \ast \eta)(b)", red] \\
      & G^\prime(F^\prime(a)) \ar[rr] & & G^\prime(F^\prime(b)) \\
      G(F^\prime(a)) \ar[rr, "G(F^\prime(f))"'] \ar[ur, "\theta(F^\prime(a))", near end] & & G(F^\prime(b)) \ar[from=uu, crossing over, "G(\eta(b))", near end] \ar[ur, "\theta(F^\prime(b))"'] \\
    \end{tikzcd}
  \end{minipage}
\end{center}

\lem*{
  圏$\bm{C}, \bm{D}, \bm{E}$と、$\bm{C}$から$\bm{D}$への関手$F, F^\prime$、$\bm{D}$から$\bm{E}$への関手$G, G^\prime$を考える。

  $F$から$F^\prime$への自然変換$\eta$と、$G$から$G^\prime$への自然変換$\theta$について、以下が成り立つ。
  \eq*{
    \forall a \in \bm{C} \qty\Big(\theta(F^\prime(a)) \circ G(\eta(a)) = G^\prime(\eta(a)) \circ \theta(F(a)))
  }
}{
  $\theta$は$G$から$G^\prime$への自然変換であるので、明らか。
}

\thm{相互交換法則}{
  圏$\bm{C}, \bm{D}, \bm{E}$と、$\bm{C}$から$\bm{D}$への関手$F, F^\prime, F^{\prime \prime}$、$\bm{D}$から$\bm{E}$への関手$G, G^\prime, G^{\prime \prime}$を考える。

  $F$から$F^\prime$への自然変換$\sigma$、$F^\prime$から$F^{\prime \prime}$への自然変換$\tau$、$G$から$G^\prime$への自然変換$\eta$、$G^\prime$から$G^{\prime \prime}$への自然変換$\theta$について、以下が成り立つ。
  \eq*{
    \qty(\theta \circ \eta) \ast \qty(\tau \circ \sigma) = \qty(\theta \ast \tau) \circ \qty(\eta \ast \sigma)
  }
}{
  $\bm{C}$の任意の対象$a$について、以下が成り立つ。
  \eqa*{
    \qty(\qty(\theta \circ \eta) \ast \qty(\tau \circ \sigma))(a)
    &= \qty(\theta \circ \eta)\qty(F^{\prime \prime}(a)) \circ G \qty(\qty(\tau \circ \sigma)(a)) \\*
    &= \theta\qty(F^{\prime \prime}(a)) \circ \eta\qty(F^{\prime \prime}(a)) \circ G\qty(\tau(a)) \circ G \qty(\sigma(a)) \\*
    &= \theta\qty(F^{\prime \prime}(a)) \circ G^\prime\qty(\tau(a)) \circ \eta\qty(F^\prime(a)) \circ G \qty(\sigma(a)) \\*
    &= \qty(\theta \ast \tau)(a) \circ \qty(\eta \ast \sigma)(a) \\*
    &= \qty(\qty(\theta \ast \tau) \circ \qty(\eta \ast \sigma))(a)
  }

  ただし第二行から第三行へは、$\eta$が$G$から$G^\prime$への自然変換であることを用いた。
}

\begin{center}
  \begin{minipage}{0.8\linewidth}
    $\bm{E}$ \\
    \begin{tikzcd}
      G(F(a)) \ar[rr, "\eta(F(a))"] \ar[dd, "G(\sigma(a))"'] \ar[dddd, bend right=90, red, "G\qty((\tau \circ \sigma)(a))"'] \ar[rrdd, red, "(\eta \ast \sigma) (a)"] & & G^\prime(F(a)) \ar[rr, "\theta(F(a))"] \ar[dd, "G^\prime(\sigma(a))"] & & G^{\prime \prime}(F(a)) \ar[dd, "G^{\prime \prime}(\sigma(a))"] \\
      & & & & \\
      G(F^\prime(a)) \ar[rr, "\eta(F^\prime(a))"] \ar[dd, "G(\tau(a))"'] & & G^\prime(F^\prime(a)) \ar[rr, "\theta(F^\prime(a))"] \ar[dd, "G^\prime(\tau(a))"] \ar[rrdd, red, "(\theta \ast \tau) (a)"] & & G^{\prime \prime}(F^\prime(a)) \ar[dd, "G^{\prime \prime}(\tau(a))"] \\
      & & & & \\
      G(F^{\prime \prime}(a)) \ar[rr, "\eta(F^{\prime \prime}(a))"'] \ar[rrrr, bend right, red, "\qty(\theta \circ \eta) \qty(F^{\prime \prime} (a))"'] & & G^\prime(F^{\prime \prime}(a)) \ar[rr, "\theta(F^{\prime \prime}(a))"'] & & G^{\prime \prime}(F^{\prime \prime}(a)) \\
    \end{tikzcd}
  \end{minipage}
\end{center}

\dfn{圏同値}{
  圏$\bm{C}, \bm{D}$を考える。

  $\bm{C}$から$\bm{D}$への関手$F$と、$\bm{D}$から$\bm{C}$への関手$G$が存在して、
  $\id_{\bm{C}}$から$G \circ F$への自然同型$\eta$と、$\id_{\bm{D}}$から$F \circ G$への自然同型$\theta$が存在するとき、
  $\bm{C}$と$\bm{D}$は圏同値であると言う。
}

\begin{center}
  \begin{minipage}{0.4\linewidth}
    $\bm{C}$ \\
    \begin{tikzcd}
      a \ar[r, "f"] \ar[d, bend left, "\eta(a)"] & b \ar[d, bend left, "\eta(b)"] \\
      (G \circ F) (a) \ar[r, "(G \circ F) (f)"] \ar[u, bend left, "\eta(a)^{-1}"] & (G \circ F) (b) \ar[u, bend left, "\eta(b)^{-1}"] \\
    \end{tikzcd}
  \end{minipage}
  \begin{minipage}{0.4\linewidth}
    $\bm{D}$ \\
    \begin{tikzcd}
      c \ar[r, "g"] \ar[d, bend left, "\theta(c)"] & d \ar[d, bend left, "\theta(d)"] \\
      (F \circ G) (c) \ar[r, "(F \circ G) (g)"] \ar[u, bend left, "\theta(c)^{-1}"] & (F \circ G) (d) \ar[u, bend left, "\theta(d)^{-1}"] \\
    \end{tikzcd}
  \end{minipage}
\end{center}

\cor{圏同型ならば圏同値}{
  圏$\bm{C}, \bm{D}$について、$\bm{C}$と$\bm{D}$が圏同型ならば、$\bm{C}$と$\bm{D}$は圏同値である。
}


\lsubsection{さまざまな圏}

\dfnf{圏$\bm{1}$}{圏1}{
  \num{1}つの対象からなる離散圏を、$\bm{1}$で表す。
}

\dfnf{圏$\bm{2}$}{圏2}{
  \num{2}つの対象からなる離散圏を、$\bm{2}$で表す。
}

\dfnf{圏$\bm{\rightrightarrows}$}{並行射からなる圏}{
  \num{2}つの対象$a, b$と、\num{2}つの射$f \colon a \to b, g \colon a \to b$からなる圏を、$\bm{\rightrightarrows}$で表す。
}

\dfnf{圏$\bm{\leftarrow \rightarrow}$}{スパンの圏}{
  \num{3}つの対象$a, b, c$と、\num{2}つの射$f \colon c \to a, g \colon c \to b$からなる圏を、$\bm{\leftarrow \rightarrow}$で表す。
}

\dfnf{圏$\bm{\rightarrow \leftarrow}$}{コスパンの圏}{
  圏$\qty(\bm{\leftarrow \rightarrow})^\opp$を、$\bm{\rightarrow \leftarrow}$で表す。
}

\rem{自然変換は射}{
  圏$\bm{D}$から圏$\bm{C}$への関手$F, G$について、$F$から$G$への自然変換$\eta$を考える。

  $\dom(\eta)$を$F$に対する恒等な自然変換$F$、$\cod(\eta)$を$G$に対する恒等な自然変換$G$として、合成を垂直合成で定義すると、
  自然変換は射の公理をみたす。
}

\dfn{関手圏}{
  圏$\bm{C}, \bm{D}$について、自然変換を射とみなすと、
  $\bm{D}$から$\bm{C}$への関手間の自然変換であることは、圏となる。

  この圏を、$\bm{D}$から$\bm{C}$への関手圏と呼び、$\Func(\bm{D}, \bm{C})$または$\bm{C}^{\bm{D}}$と表す。
}

\cor{関手圏の双対}{
  $\Func(\bm{D}, \bm{C})^\opp$と$\Func(\bm{D}^\opp, \bm{C}^\opp)$は圏同型である。
}

\dfn{対角関手}{
  圏$\bm{C}, \bm{D}$について、$\bm{C}$から$\Func(\bm{D}, \bm{C})$への以下で定める関手$\Delta$を考える。
  \eq*{
    \Delta(c)(d) \coloneqq c
  }

  このように定めた関手$\Delta$を対角関手と呼ぶ。
}


\lsubsection{普遍}

\dfn{普遍射}{
  圏$\bm{C}, \bm{D}$と、$\bm{D}$から$\bm{C}$への関手$F$、$\bm{C}$の対象$a$を考える。

  $F$から$a$への普遍射とは、$\bm{D}$の対象$x$と圏$\bm{C}$の射$u$の組であって、以下の\num{2}つを満たすものである。
  \eqg*{
    u \in \Hom_{\bm{C}}\qty(F(x), a) \\*
    \forall y \in \Obj_{\bm{D}} \forall f \in \Hom_{\bm{C}}\qty(F(y), a) \exists! g \in \Hom_{\bm{D}}(y, x) \qty(f = u \circ F(g))
  }
}

\begin{center}
  \begin{minipage}{0.3\linewidth}
    $\bm{D}$ \\
    \begin{tikzcd}
      y \ar[r, "g", dotted, red] & x \\
      & \\
    \end{tikzcd}
  \end{minipage}
  \begin{minipage}{0.3\linewidth}
    $\bm{C}$ \\
    \begin{tikzcd}
      F(y) \ar[r, "F(g)", dotted] \ar[dr, "f"'] & F(x) \ar[d, "u"] \\
      & a \\
    \end{tikzcd}
  \end{minipage}
\end{center}

\lem{普遍射の一意性}{
  $\bm{D}$から$\bm{C}$への関手$F$、$\bm{C}$の対象$a$について、$x, u$と$x^\prime, u^\prime$が$F$から$a$への普遍射であるとする。

  このとき、以下を満たす。
  \eq*{
    \exists! h \in \Hom_{\bm{D}}(x^\prime, x) \qty(u^\prime = u \circ F(h))
  }

  さらに上で与える$h$は同型射であり、その逆射$h^{-1}$は$u = u^\prime \circ F(h^{-1})$を満たす。
}{
  $x, u$は普遍射であるので、$x^\prime, u^\prime$について定義より、
  $\bm{D}$の射$g \colon x^\prime \to x$であって$u^\prime = u \circ F(g)$なる射が一意に存在する。

  同様に$x^\prime, u^\prime$は普遍射であるので、$x, u$について定義より、
  $\bm{D}$の射$g^\prime \colon x \to x^\prime$であって$u = u^\prime \circ F(g^\prime)$なる射が一意に存在する。

  ゆえに、$u = u^\prime \circ F(g^\prime) = u \circ F(g) \circ F(g^\prime) = u \circ F(g \circ g^\prime)$が成り立つ。

  $x, u$は普遍射であるので、$x, u$について定義より、
  $\bm{D}$の射$i \colon x \to x$であって$u = u \circ F(i)$なる射が一意に存在する。

  $x, g \circ g^\prime$は$i$の条件を満たすので、$i = x = g \circ g^\prime$

  同様に、$x^\prime = g^\prime \circ g$である。

  したがって、$g^\prime$は$g$の逆射である。
}

\dfn{極限}{
  圏$\bm{C}, \bm{D}$と、$\bm{C}$から$\Func(\bm{D}, \bm{C})$への対角関手$\Delta$を考える。

  $\bm{D}$から$\bm{C}$への関手$X$について、$\Delta$から$X$への普遍射を$X$の極限と呼ぶ。
}

\dfn{積}{
  離散圏$\bm{J}$、圏$\bm{C}$と、$\bm{J}$から$\bm{C}$への関手$X$を考える。

  このとき、$X$の極限を$X$の積と呼ぶ。
  Xの積$x, u$について、この$x$を$\prod X$と表す。
}

\begin{center}
  \begin{minipage}{0.3\linewidth}
    $\bm{C}$ \\
    \begin{tikzcd}
      Y \ar[r, "q", dotted, red] \ar[dr, "p_j"'] & \prod X \ar[d, "u_j"] \\
      & X_j \\
    \end{tikzcd}
  \end{minipage}
  \begin{minipage}{0.3\linewidth}
    $\Func(\bm{J}, \bm{C})$ \\
    \begin{tikzcd}
      \Delta(Y) \ar[r, "\Delta(q)", dotted] \ar[dr, "p"'] & \Delta \qty(\prod X) \ar[d, "u"] \\
      & X \\
    \end{tikzcd}
  \end{minipage}
\end{center}

\dfn{終対象}{
  圏$\bm{C}$を考える。

  $\bm{C}$から圏$\bm{1}$への関手は一意に定まる。
  これを$U$とすると、$U$から$\ast$への普遍射$x, u$は、$u = \ast$となる。

  この$x$を、$\bm{C}$の終対象と呼ぶ。
}

\begin{center}
  \begin{minipage}{0.2\linewidth}
    $\bm{C}$ \\
    \begin{tikzcd}
      y \ar[r, "g", dotted, red] & x \\
    \end{tikzcd}
  \end{minipage}
  \begin{minipage}{0.2\linewidth}
    $\bm{1}$ \\
    \begin{tikzcd}
      \ast \\
    \end{tikzcd}
  \end{minipage}
\end{center}

\dfn{引き戻し}{
  圏$\bm{C}$と、$\bm{C}$の射$f^\prime, g^\prime$を考える。
  $\cod(f^\prime) = \cod(g^\prime)$とする。

  圏$\bm{\rightarrow \leftarrow}$(対象ではない射を$f, g$とする)から$\bm{C}$への関手$X$であって、以下を満たす$X$を考える。
  \eq*{
    X(f) = f^\prime \land X(g) = g^\prime
  }

  このとき、$X$の極限を、$f^\prime, g^\prime$の引き戻しと呼ぶ。
}

\begin{center}
  \begin{minipage}{0.2\linewidth}
    $\bm{\rightarrow \leftarrow}$ \\
    \begin{tikzcd}
      & b \ar[d, "g"] \\
      a \ar[r, "f"'] & c
    \end{tikzcd}
  \end{minipage}
  \begin{minipage}{0.4\linewidth}
    $\bm{C}$ \\
    \begin{tikzcd}
      & Y \ar[dd, "q", red, dotted] \ar[dddl, "p(a)"'] \ar[ddrr, "p(b)"] \ar[dddr, "p(c)"] & & \\
      & & & \\
      & x \ar[ld, "u(a)"'] \ar[rr, "u(b)"] \ar[rd, "u(c)"'] & & X(b) \ar[ld, "g^\prime = X(g)"] \\
      X(a) \ar[rr, "f^\prime = X(f)"'] & & X(c) & \\
    \end{tikzcd}
  \end{minipage}
  \begin{minipage}{0.2\linewidth}
    $\Func(\bm{\rightarrow \leftarrow}, \bm{C})$ \\
    \begin{tikzcd}
      \Delta(Y) \ar[r, "\Delta(q)", dotted] \ar[dr, "p"'] & \Delta(x) \ar[d, "u"] \\
      & X
    \end{tikzcd}
  \end{minipage}
\end{center}

\dfn{等化子}{
  圏$\bm{C}$と、$\bm{C}$の射$f^\prime, g^\prime$を考える。
  $\dom(f^\prime) = \dom(g^\prime) \land \cod(f^\prime) = \cod(g^\prime)$とする。

  このとき、$f^\prime, g^\prime$の引き戻し$x, u$を、$f^\prime, g^\prime$の等化子と呼び、この$x$を$\Eq(f^\prime, g^\prime)$で表す。
}

\begin{center}
  \begin{minipage}{0.2\linewidth}
    $\bm{\rightrightarrows}$ \\
    \begin{tikzcd}
      a \ar[r, "f"', bend right] \ar[r, "g", bend left] & b \\
    \end{tikzcd}
  \end{minipage}
  \begin{minipage}{0.45\linewidth}
    $\bm{C}$ \\
    \begin{tikzcd}
      & Y \ar[dd, "q", red, dotted] \ar[dddl, "p(a)"'] \ar[ddrr, "p(a)"] \ar[dddr, "p(b)"] & & \\
      & & & \\
      & \Eq(f^\prime, g^\prime) \ar[ld, "u(a)"'] \ar[rr, "u(a)"] \ar[rd, "u(b)"'] & & X(a) \ar[ld, "g^\prime = X(g)"] \\
      X(a) \ar[rr, "f^\prime = X(f)"'] & & X(b) & \\
    \end{tikzcd}
  \end{minipage}
  \begin{minipage}{0.25\linewidth}
    $\Func(\bm{\rightrightarrows}, \bm{C})$ \\
    \begin{tikzcd}
      \Delta(Y) \ar[r, "\Delta(q)", dotted] \ar[dr, "p"'] & \Delta\qty(\Eq(f^\prime, g^\prime)) \ar[d, "u"] \\
      & X \\
    \end{tikzcd}
  \end{minipage}
\end{center}

\lem{等化子から得る単射}{
  $f$と$g$の等化子$\Eq(f, g), u$について、$u\qty(\dom(f))$は単射である。
}{
  $\cod(h_1) = \cod(h_2) = \Eq(X(f), X(g))$なる射$h_1, h_2$を考える。

  $u\qty(\dom(f)) \circ h_1 = u\qty(\dom(f)) \circ h_2$について、
}

\dfn{余普遍射}{
  圏$\bm{C}, \bm{D}$と、$\bm{D}$から$\bm{C}$への関手$F$、$\bm{C}$の対象$a$を考える。

  $a$から$F$への余普遍射とは、$F^\opp$から$a$への普遍射$x, u$について、$x, u^\opp$である。
}

\begin{center}
  \begin{minipage}{0.15\linewidth}
    $\bm{D}$ \\
    \begin{tikzcd}
      y & x \ar[l, "g"', dotted, red] \\
      &
    \end{tikzcd}
  \end{minipage}
  \begin{minipage}{0.15\linewidth}
    $\bm{D}^\opp$ \\
    \begin{tikzcd}
      y \ar[r, "g^\opp", dotted, red] & x \\
      &
    \end{tikzcd}
  \end{minipage}
  \begin{minipage}{0.3\linewidth}
    $\bm{C}^\opp$ \\
    \begin{tikzcd}
      F^\opp(y) \ar[rr, "F^\opp(g^\opp)", dotted] \ar[drr, "f"'] & & F^\opp(x) \ar[d, "u"] \\
      & & a
    \end{tikzcd}
  \end{minipage}
  \begin{minipage}{0.3\linewidth}
    $\bm{C}$ \\
    \begin{tikzcd}
      F(y) & & F(x) \ar[ll, "F(g)"', dotted] \\
      & & a \ar[u, "u^\opp"'] \ar[ull, "f^\opp"]
    \end{tikzcd}
  \end{minipage}
\end{center}

\dfn{余極限}{
  圏$\bm{C}, \bm{D}$と、$\bm{C}$から$\Func(\bm{D}, \bm{C})$への対角関手$\Delta$を考える。

  $\bm{D}$から$\bm{C}$への関手$X$について、$X$から$\Delta$への余普遍射を$X$の余極限と呼ぶ。
}

\dfn{和}{
  離散圏$\bm{J}$、圏$\bm{C}$と、$\bm{J}$から$\bm{C}$への関手$X$を考える。

  このとき、$X$の余極限を$X$の和と呼ぶ。
}

\dfn{始対象}{
  圏$\bm{C}$と、圏$\bm{1}$を考える。

  $\bm{C}$から$\bm{1}$への関手は一意に定まる。
  これを$U$とすると、$\ast$から$U$への余普遍射は、\dfnref{余普遍射}の条件を満たす$x$と$\ast$の組となる。

  この$x$を、$\bm{C}$の始対象と呼ぶ。
}

\dfn{押し出し}{
  圏$\bm{C}$と、$\bm{C}$の射$f^\prime, g^\prime$を考える。
  $\dom(f^\prime) = \dom(g^\prime)$とする。

  圏$\bm{\leftarrow \rightarrow}$(対象ではない射を$f, g$とする)から$\bm{C}$への関手$X$であって、以下を満たす$X$を考える。
  \eq*{
    X(f) = f^\prime \land X(g) = g^\prime
  }

  このとき、$X$の余極限を、$f^\prime, g^\prime$の押し出しと呼ぶ。
}

\dfn{余等化子}{
  圏$\bm{C}$と、$\bm{C}$の射$f^\prime, g^\prime$を考える。
  $\dom(f^\prime) = \dom(g^\prime) \land \cod(f^\prime) = \cod(g^\prime)$とする。

  このとき、$f^\prime, g^\prime$の押し出し$x, u$を、$f^\prime, g^\prime$の余等化子と呼ぶ。
}

\lem{余等化子から得る全射}{
  $f$と$g$の余等化子$x, u$について、$u\qty(\cod(f))$は全射である。
}{
  \lemref{等化子から得る単射}より、双対性から明らか。
}


\lsubsection{随伴}

\dfn{左随伴関手}{
  圏$\bm{C}, \bm{D}$を考える。

  $\bm{D}$から$\bm{C}$への関手$F$が左随伴関手であるとは、$\bm{C}$の任意の対象$a$について、$F$から$a$への普遍射が存在することである。
}

\dfn{右随伴関手}{
  圏$\bm{C}, \bm{D}$を考える。

  $\bm{C}$から$\bm{D}$への関手$G$が右随伴関手であるとは、$G^\opp$が左随伴関手であることである。
}

\dfn{随伴}{
  圏$\bm{C}, \bm{D}$と、$\bm{D}$から$\bm{C}$への関手$F$、$\bm{C}$から$\bm{D}$への関手$G$を考える。

  $\id_{\bm{D}}$から$G \circ F$への自然変換$\eta$と、$F \circ G$から$\id_{\bm{C}}$への自然変換$\theta$が存在して、以下を満たすならば、$F, G$は随伴であると呼ぶ。
  \eqg*{
    F = \qty(\theta \ast F) \circ \qty(F \ast \eta) \\*
    G = \qty(G \ast \theta) \circ \qty(\eta \ast G)
  }

  このとき、$F$を$G$の左随伴、$G$を$F$の右随伴と呼ぶ。
}

\begin{center}
  \begin{minipage}{0.3\linewidth}
    $\Func(\bm{D}, \bm{C})$ \\
    \begin{tikzcd}
      F \ar[rr, "F \ast \eta"] \ar[drr, "F"'] & & F \circ G \circ F \ar[d, "\theta \ast F"] \\
      & & F
    \end{tikzcd}
  \end{minipage}
  \begin{minipage}{0.3\linewidth}
    $\Func(\bm{C}, \bm{D})$ \\
    \begin{tikzcd}
      G \ar[rr, "\eta \ast G"] \ar[drr, "G"'] & & G \circ F \circ G \ar[d, "G \ast \theta"] \\
      & & G
    \end{tikzcd}
  \end{minipage}
\end{center}

\cor*{
  \corref{関手圏の双対}より、$F, G$が随伴ならば、$G^\opp, F^\opp$は随伴である。
}

\lem{随伴は随伴関手}{
  圏$\bm{C}, \bm{D}$と、$\bm{D}$から$\bm{C}$への関手$F$、$\bm{C}$から$\bm{D}$への関手$G$を考える。

  $F, G$が随伴ならば、$F$は左随伴関手であり、$G$は右随伴関手である。
}{
  随伴であるので、\dfnref{随伴}を満たす自然変換$\eta, \theta$が存在する。

  双対性から、$F$が左随伴関手であることを示せば十分である。

  \vskip\baselineskip

  $\bm{C}$の任意の対象$a$について、$G(a), \theta(a)$が$F$から$a$への普遍射となることを示す。

  $\bm{D}$の対象$y$と、$\bm{C}$の射$f \colon F(y) \to a$を考える。

  このとき、$g \coloneqq G(f) \circ \eta(y)$を考えると、$\theta$の自然性と$F = \qty(\theta \ast F) \circ \qty(F \ast \eta)$より、以下が成り立つ。
  \eqa*{
    \theta(a) \circ F(g)
    &= \theta(a) \circ \qty(F \circ G)(f) \circ \qty(F \ast \eta)(y) \\*
    &= f \circ \qty(\theta \ast F)(y) \circ \qty(F \ast \eta)(y) \\*
    &= f \circ F(y) \\*
    &= f
  }

  \vskip\baselineskip

  次に、$\bm{D}$の射$g^\prime \colon y \to G(a)$が存在して、$f = \theta(a) \circ F(g^\prime)$であるとする。

  このとき、$G = \qty(G \ast \theta) \circ \qty(\eta \ast G)$と$\eta$の自然性より、以下が成り立つ。
  \eqa*{
    g^\prime
    &= G(a) \circ g^\prime \\*
    &= \qty(G \ast \theta)(a) \circ \qty(\eta \ast G)(a) \circ g^\prime \\*
    &= \qty(G \ast \theta)(a) \circ \qty(G \circ F)(g^\prime) \circ \eta(y) \\*
    &= G \qty(\theta(a) \circ F(g^\prime)) \circ \eta(y) \\*
    &= G(f) \circ \eta(y) \\*
    &= g
  }

  ゆえに一意である。
}

\begin{center}
  \begin{minipage}{0.45\linewidth}
    $\bm{D}$ \\
    \begin{tikzcd}
      & y \ar[dd, "\eta(y)"] \ar[rrdd, "g = G(f) \circ \eta(y)", red, dotted] & & \\
      & & & \\
      & \qty(G \circ F)(y) \ar[rr, "G(f)"] & & G(a) \\
      & & & \\
    \end{tikzcd}
  \end{minipage}
  \begin{minipage}{0.45\linewidth}
    $\bm{C}$ \\
    \begin{tikzcd}
      & F(y) \ar[dd, "\qty(F \ast \eta)(y)"] \ar[dddl, "F(y)"'] \ar[dddr, "f"] \ar[rrdd, "F(g)", dotted] & & \\
      & & & \\
      & \qty(F \circ G \circ F)(y) \ar[ld, "\qty(\theta \ast F)(y)"] \ar[rr, "\qty(F \circ G)(f)"] & & \qty(F \circ G)(a) \ar[ld, "\theta(a)"] \\
      F(y) \ar[rr, "f"'] & & a & \\
    \end{tikzcd}
  \end{minipage}
\end{center}

\lem{左随伴は余極限を保存する}{
  圏$\bm{J}, \bm{C}, \bm{D}$と、$\bm{J}$から$\bm{D}$への関手$X$、$\bm{D}$から$\bm{C}$への関手$F$、$\bm{C}$から$\bm{D}$への関手$G$を考える。
  $F, G$は随伴であるとする。

  $\qty(x, u)$が$X$の余極限であるならば、$\qty(F(x), F \ast u)$は$F \circ X$の余極限である。
}{
  随伴であるので、\dfnref{随伴}を満たす自然変換$\eta, \theta$が存在する。

  $\bm{D}$から$\Func(\bm{J}, \bm{D})$への対角関手を$\Delta_{\bm{D}}$、$\bm{C}$から$\Func(\bm{J}, \bm{C})$への対角関手を$\Delta_{\bm{C}}$とする。

  \vskip\baselineskip

  $\bm{C}$の対象$y$と、$F \circ X$から$\Delta_{\bm{C}}(y)$への自然変換$f$を考える。

  \lemref{随伴は随伴関手}より、$G(y), \theta(y)$は、$F$から$y$への普遍射である。

  したがって$\bm{J}$の任意の対象$j$について、$\theta(y) \circ F(w(j)) = f(j)$を満たす射$w(j)$が一意に存在する。

  ここで、$f$は$F \circ X$から$\Delta_{\bm{C}}(y)$への自然性であるので、$\bm{J}$の任意の対象$j, j^\prime$について以下が成り立つ。
  \eqa*{
    \theta(y) \circ F(w(j^\prime) \circ X(i))
    &= \theta(y) \circ F(w(j^\prime)) \circ \qty(F \circ X)(i) \\*
    &= f(j^\prime) \circ \qty(F \circ X)(i) \\*
    &= f(j) \\*
    &= \theta(y) \circ F(w(j))
  }

  一意性より、$w(j^\prime) \circ X(i) = w(j)$である。
  よって、$w$は$X$から$\Delta_{\bm{D}}$への自然変換である。

  $x, u$は$X$から$\Delta_{\bm{D}}$への余普遍射であるので、$w = \Delta_{\bm{D}}(h) \circ u$を満たす$\bm{D}$の射$h$が一意に存在する。

  ここで、$g \coloneqq \theta(y) \circ F(h)$とすると、$\bm{J}$の任意の対象$j$について以下が成り立つ。
  \eqa*{
    g \circ \qty(F \ast u)(j)
    &= \theta(y) \circ F(h) \circ \qty(F \ast u)(j) \\*
    &= \theta(y) \circ F(h \circ u(j)) \\*
    &= \theta(y) \circ \qty(F \ast w)(j) \\*
    &= f(j)
  }

  したがって、$\Delta_{\bm{C}}(g) \circ \qty(F \ast u) = f$である。

  \vskip\baselineskip

  次に、$\bm{C}$の射$g^\prime$が存在して、$\Delta_{\bm{C}}(g^\prime) \circ \qty(F \ast u) = f$であるとする。

  $G(y), \theta(y)$は$F$から$y$への普遍射であるため、$g^\prime = \theta(y) \circ F(h^\prime)$を満たす$\bm{D}$の射$h^\prime$が一意に存在する。

  $\bm{J}$の任意の対象$j$について、以下が成り立つ。
  \eqa*{
    \theta(y) \circ F\qty(h \circ u(j))
    &= f(j) \\*
    &= g^\prime(j) \circ \qty(F \ast u)(j) \\*
    &= \theta(y) \circ F(h^\prime) \circ \qty(F \ast u)(j) \\*
    &= \theta(y) \circ F\qty(h^\prime \circ u(j))
  }

  再び$G(y), \theta(y)$は$F$から$y$への普遍射であるため、一意性より$h \circ u(j) = h^\prime \circ u(j)$である。
  ゆえに、$\Delta_{\bm{D}}(h^\prime) \circ u = \Delta_{\bm{D}}(h) \circ u$である。

  $x, u$は、$X$の余極限であるため、一意性より$h = h^\prime$である。
  したがって、$g = g^\prime$である。
}

\begin{center}
  \begin{minipage}{0.25\linewidth}
    $\bm{J}$ \\
    \begin{tikzcd}
      & & \\
      j \ar[rr, "i"] & & j^\prime \\
    \end{tikzcd}
  \end{minipage}
  \begin{minipage}{0.3\linewidth}
    $\bm{D}$ \\
    \begin{tikzcd}
      G(y) & & \\
      & x \ar[lu, "h"'] & \\
      X(j) \ar[uu, "w(j)"] \ar[rr, "X(i)"'] \ar[ru, "u(j)"] & & X(j^\prime) \ar[lu, "u(j^\prime)"'] \ar[lluu, "w(j^\prime)"', bend right=50] \\
    \end{tikzcd}
  \end{minipage}
  \begin{minipage}{0.3\linewidth}
    $\bm{C}$ \\
    \begin{tikzcd}
      & \qty(F \circ G)(y) \ar[ld, "\theta(y)"'] & \\
      y & & F(x) \ar[lu, "F(h)"'] \ar[ll, "g"', red, dotted, near start] \\
      & \qty(F \circ X)(j) \ar[lu, "f(j)"] \ar[uu, "\qty(F \ast w)(j)", near start] \ar[ru, "\qty(F \ast u)(j)"'] & \\
    \end{tikzcd}
  \end{minipage}
\end{center}

\begin{center}
  \begin{minipage}{0.35\linewidth}
    $\Func(\bm{J}, \bm{D})$ \\
    \begin{tikzcd}
      \Delta_{\bm{D}}\qty(G(y)) & \\
      & \Delta_{\bm{D}}(x) \ar[lu, "\Delta_{\bm{D}}(h)"'] \\
      X \ar[ru, "u"'] \ar[uu, "w"] & \\
    \end{tikzcd}
  \end{minipage}
  \begin{minipage}{0.35\linewidth}
    $\Func(\bm{J}, \bm{C})$ \\
    \begin{tikzcd}
      & \Delta_{\bm{C}}\qty(\qty(F \circ G)(y)) \ar[ld, "\Delta_{\bm{C}}\qty(\theta(y))"'] & \\
      \Delta_{\bm{C}}(y) & & \Delta_{\bm{C}}\qty(F(x)) \ar[lu, "\Delta_{\bm{C}}\qty(F(h))"'] \ar[ll, "\Delta_{\bm{C}}(g)"', dotted, near start] \\
      & F \circ X \ar[lu, "f"] \ar[uu, "F \ast w", near start] \ar[ru, "F \ast u"'] & \\
    \end{tikzcd}
  \end{minipage}
\end{center}

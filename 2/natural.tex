\lsection{自然数}

\lsubsection{自然数の構成}

\axi{無限の公理}{
  \eq*{
    \exists A \qty(\varnothing \in A \land \forall x \in A \qty(x \cup \qty{x} \in A))
  }
}

\dfn{無限系譜}{
  集合$X$が無限系譜であるとは、以下を満たすことである。
  \eq*{
    \varnothing \in X \land \forall x \qty(x \in X \rightarrow x \cup \qty{x} \in X)
  }

  \axiref{無限の公理}より無限系譜は存在する。
}

\cor*{
  \eq*{
    \forall X \forall Y \qty(M(X) \land M(Y) \rightarrow \text{$X \cap Y$は無限系譜である})
  }
}

\dfn{後者関数}{
  無限系譜$X$について、以下のように定める写像$s \colon X \to X$を後者関数(successor)と呼ぶ。
  \eq*{
    s(x) \coloneqq x \cup \qty{x}
  }
}

\lem*{
  以下が成り立つ。
  \eq*{
    \forall A \qty(A \neq \varnothing \land \forall X \in A \qty(\text{$X$は無限系譜である}) \rightarrow \text{$\bigcap A$は無限系譜である})
  }
}{
  空でない$A$を考える。

  $\forall X \in A \qty(\varnothing \in X)$であるので、$\varnothing \in \bigcap A$である。

  $\forall x \in \bigcap A$について、$\forall X \in A \qty(x \in X)$であり、無限系譜であることから$s(x) \in X$である。
  ゆえに、$s(x) \in \bigcap A$

  よって、$\bigcap A$は無限系譜である。
}

\lem*{
  無限系譜$X$について、以下で定める集合は無限系譜である。
  \eq*{
    \bigcap \qty{Y \subset X \mid \text{$Y$は無限系譜である}}
  }
}{
  $A \coloneqq \qty{Y \subset X \mid \text{$Y$は無限系譜である}}$とすると、
  $X \in A$より、$A \neq \varnothing$である。

  \mlemref{-1}より成り立つ。
}

\lem*{
  \mlemref{-1}で定める集合は、無限系譜$X$の取り方によらずに一意に定まる。
}{
  $\omega(X) \coloneqq \bigcap \qty{Y \subset X \mid \text{$Y$は無限系譜である}}$とする。

  $X_1, X_2$の二つの無限系譜を考える。

  $X_1$の部分で無限系譜である$Y$を考える。

  \mlemref{-2}より$X_2 \cap Y$も無限系譜であり、定義より$\omega(X_2) \subset X_2 \cap Y \subset Y$である。

  任意の$Y$について成り立つので、$\omega(X_2) \subset \omega(X_1)$である。

  同様に$\omega(X_1) \subset \omega(X_2)$であるため、示される。
}

\dfn{自然数}{
  \mlemref{0}より、無限系譜$X$の取り方によらず$\omega(X)$が一意に定まる。
  これを自然数$\N$(または$\omega$)と呼ぶ。
  誤解のない範囲で、自然数$\N$の要素も自然数と呼ぶ。
}

\dfn{数字}{
  数字を定義する。
  \eq*{
    0 \coloneqq \qty{}, 1 \coloneqq \qty{0}, 2 \coloneqq \qty{0, 1}, 3 \coloneqq \qty{0, 1, 2}, \cdots
  }
}

\thm{Peanoの公理}{
  以下の全てを満たす。
  \eqg*{
    0 \in \N \\*
    \forall n \in \N \qty(s(n) \in N) \\*
    \forall n \in \N \qty(s(n) \neq 0) \\*
    \forall n, m \in \N \qty(s(n) = s(m) \rightarrow n = m) \\*
    \forall E \qty(E \subset \N \land 0 \in E \land \forall n \qty(n \in E \rightarrow s(n) \in E) \rightarrow E = \N)
  }
}{
  第一、第二の公理は、無限系譜であることより明らか。

  \vskip\baselineskip

  第三の公理は$n \in s(n)$と、空集合は要素を持たないことから示される。

  \vskip\baselineskip

  第四の公理を示す。

  $n \in s(n) \rightarrow n \in s(m)$であり、$n \in s(m) \rightarrow n = m \lor n \in m$

  よって、$\qty(n = m \lor n \in m) \land \qty(m = n \lor m \in n)$

  \axiref{正則性の公理}より示される。

  \vskip\baselineskip

  第五の公理を示す。

  前件より、$E \subset \N$である。

  $0 \in E \land \forall n \qty(n \in E \rightarrow s(n) \in E)$より、$E$は無限系譜である。

  定義より$\N = \omega(E) \subset E$である。
}

\thm{数学的帰納法}{
  アリティ1の述語記号$P$について、
  \eq*{
    \forall n \qty(n \in \N \rightarrow P(n)) \leftrightarrow P(0) \land \forall n \qty\Big(n \in \N \land \qty\big(P(n) \rightarrow P(s(n))))
  }
}{
  右は明らか。

  \vskip\baselineskip

  左は、$E = \qty{n \in \N \mid P(n)}$として\thmref{Peanoの公理}第五式から示される。
}

\dfn{前者関数}{
  後者関数$s$は、\thmref{Peanoの公理}の第四式より単射であるため、左逆写像を持つ。

  これを前者関数$p \in \N \setminus \qty{0} \to \N$と呼ぶ。
}

\lem*{
  \eq*{
    \forall n \in \N \qty(n \neq 0 \rightarrow s(p(n)) = n)
  }
}{
  $n = 0$のとき前件否定より成り立つ。

  ある$n$で成り立つとき、$p$が$s$の左逆写像より$s(p(s(n))) = s(n)$が成り立つ。

  \thmref{数学的帰納法}より、任意の$n$について成り立つ。
}

\lem*{
  \eq*{
    \forall n \in \N \qty(m \in n \rightarrow m \in \N)
  }
}{
  $n = 0 = \varnothing$のとき自明。

  ある$n$で成り立つとき、$s(n) = n \cup \qty{n}$より、$s(n)$で成り立つ。

  \thmref{数学的帰納法}より、任意の$n$について成り立つ。
}


\lsubsection{自然数の順序}

\dfn{自然数の順序}{
  自然数$\N$上で二項関係$\leq$を以下のように定める。{包含の半順序性}より、$\leq$は半順序である。
  \eq*{
    n \leq m \defiff n \subset m
  }
}

\cor*{
  \eq*{
    \forall n \in \N \qty(0 \leq n)
  }
}

\cor*{
  \eq*{
    \forall n \in \N \qty(n < s(n))
  }
}

\lem{自然数の順序の要素のよる特徴づけ}{
  \eq*{
    \forall n, m \in \N \qty(n \subsetneq m \leftrightarrow n \in m)
  }
}{
  $0$は空集合より、$\lnot \qty(n \in 0) \land \lnot \qty(n \subsetneq 0)$
  ゆえに、$n \subsetneq 0 \leftrightarrow n \in 0$

  ある$m$で成り立つとする。

  \vskip\baselineskip

  左について、仮定より、
  \eqg*{
    n \in s(m) \\*
    n \in m \cup \qty{m} \\*
    n \in m \lor n = m \\*
    n \subsetneq m \lor n = m \\*
    n \subset m
  }

  $m \notin m$より、
  \eq*{
    n \subsetneq m \cup \qty{m} = s(m)
  }

  \thmref{数学的帰納法}より、左が成り立つ。

  \vskip\baselineskip

  右について、既に示した左を用いて、
  \eqg*{
    m \in n \land n \subsetneq s(m) \\*
    m \subsetneq n \land n \subsetneq s(m) \\*
    \bot \\*
    n \subsetneq s(m) \rightarrow m \notin n
  }

  ゆえに、
  \eqg*{
    n \subsetneq s(m) = m \cup \qty{m} \\*
    n \subset m \\*
    n \subsetneq m \lor n = m
  }

  選言の一つ目について、仮定より、
  \eqg*{
    n \subsetneq m \\*
    n \in m \\*
    n \in s(m)
  }

  選言の二つ目について、
  \eq*{
    n = m \rightarrow n \in s(m)
  }

  \thmref{数学的帰納法}より左が成り立つ。
}

\lem{後者と順序}{
  \eq*{
    \forall n, m \in \N \qty(n < m \rightarrow s(n) \leq m)
  }
}{
  $n \leq m \land n \neq m$より、$n \subsetneq m$

  \lemref{自然数の順序の要素のよる特徴づけ}より、$n \in m$。ゆえに、$s(n) = n \cup \qty{n} \subset m$
}

\thm{自然数の全順序}{
  $\qty(\N, \leq)$は全順序集合、すなわち以下を満たす。
  \eq*{
    \forall n, m \in \N \qty(n \leq m \lor m \leq n)
  }
}{
  $n = 0$について、$\forall m \qty(n \leq m)$である。

  ある$n \in \N$で成り立つとすると、

  $m \leq n$のとき、$m \leq n \leq s(n)$

  $n \leq m \land \lnot \qty(m \leq n)$すなわち$n \leq m \land n \neq m$のとき、\lemref{後者と順序}より、$s(n) \leq m$

  \thmref{数学的帰納法}より、任意の$n$について示される。
}

\lem{自然数の順序の後者による保存}{
  \eq*{
    \forall n, m \in \N \qty(n < m \rightarrow s(n) < s(m))
  }
}{
  $n < m \land s(m) \leq s(n)$を仮定する。

  $m < s(m)$であるので、$m < s(n)$

  ゆえに、$m \in s(n) = n \cup \qty{n}$

  したがって、$m \in n \lor m = n$

  \lemref{自然数の順序の要素のよる特徴づけ}より、$m \leq n$。矛盾する。

  背理法より示される。
}

\thm{最小値原理}{
  自然数$\N$の空でない部分集合$A$は最小元を持つ。
}{
  持たないと仮定する。

  このとき、$\forall n \in \N \forall m \in \N \qty(m \leq n \rightarrow m \notin A)$を示す。

  $n = 0$のとき、$0 \in A$ならば仮定に反するので明らか。

  ある$n$で成り立つとする。$s(n) \in A$ならば$s(n)$は$A$の最小元となるので、$s(n) \notin A$。

  \thmref{数学的帰納法}を用いて上の命題が示されるので、$\forall n \in \N \qty(n \notin A)$が直ちに言える。

  これは$A$が空でないことに反する。背理法より、示される。
}

\thm{無限降下法}{
  アリティ1の述語記号$P$について、
  \eq*{
    \forall n \in \N \exists m \in \N \qty(m < n \land P(n) \rightarrow P(m)) \rightarrow \forall n \in \N \qty(\lnot P(n))
  }
}{
  $\exists n \in \N \qty(P(n))$とすると、$\qty{n \in \N \mid P(n)}$は空でない。\thmref{最小値原理}より、最小値$n_0$を持つ。

  これは仮定に反するので、背理法より示される。
}


\lsubsection{自然数の加法}

\lem*{
  以下で定める部分写像$+$は$\N \times \N$について左全域的である。
  \eqg*{
    \forall n \in \N \qty(+(\qty(n, 0)) \coloneqq n) \\*
    \forall n, m \in \N \qty(+(\qty(n, s(m))) \coloneqq s(n + m))
  }
}{
  $\forall \qty(n, m) \in \N \times \N$に対して、$+(\qty(n, m))$が存在することを示す。

  $m = 0$のとき定義より成り立つ。

  ある$m$で成り立つとすると、定義より$\qty(n, s(m))$でも成り立つ。

  \thmref{数学的帰納法}より、任意の$m$について示される。
}

\dfn{自然数の加法}{
  \mlemref{0}より定まる写像を加法と呼び、$+$で表す。

  書き直して再掲する。
  \eqg*{
    \forall n \in \N \qty(n + 0 \coloneqq n) \\*
    \forall n, m \in \N \qty(n + s(m) \coloneqq s(n + m))
  }
}

\cor{1の和}{
  \eq*{
    \forall n \in \N \qty(s(n) = n + 1)
  }
}

\lem{自然数の加法の結合法則}{
  \eq*{
    \forall n, m, l \in \N \qty(\qty(n + m) + l = n + \qty(m + l))
  }
  つまり上の等式の両辺は、$n + m + l$と表記してもよい。
}{
  $l = 0$のとき、以下より成り立つ。
  \eq*{
    \qty(n + m) + 0 = \qty(n + m) = n + m = n + \qty(m + 0)
  }

  ある$l$で成り立つとき、
  \eqa*{
    \qty(n + m) + s(l) &= s(\qty(n + m) + l) \\*
    &= s(n + \qty(m + l)) \\*
    &= n + s(\qty(m + l)) \\*
    &= n + \qty(m + s(l))
  }

  \thmref{数学的帰納法}より、任意の$l$について示される。
}

\lem{自然数の加法の単位元}{
  \eq*{
    \forall n \in \N \qty(n + 0 = 0 + n = n)
  }
}{
  定義より、
  \eq*{
    n + 0 = n
  }

  $\forall n \in \N \qty(0 + n = n)$を示す。

  $n = 0$のとき、$0 + 0 = 0$より満たす。

  ある$n$で成り立つとき、
  \eqa*{
    0 + s(n) = s(0 + n) = s(n)
  }

  \thmref{数学的帰納法}より、任意の$n$について示される。
}

\lem*{
  \eq*{
    \forall n, m \in \N \qty(s(n) + m = s(n + m))
  }
}{
  $m = 0$のとき、以下より成り立つ。
  \eq*{
    s(n) + 0 = s(n) = s(n + 0)
  }

  ある$m$について成り立つとき、
  \eq*{
    s(n) + s(m) = s(s(n) + m) = s(s(n + m)) = s(n + s(m))
  }

  \thmref{数学的帰納法}より、任意の$m$について示される。
}

\lem{自然数の加法の交換法則}{
  \eq*{
    \forall n, m \in \N \qty(n + m = m + n)
  }
}{
  $m = 0$のとき、\lemref{自然数の加法の単位元}より示される。

  ある$m$で成り立つとき、\mlemref{-1}より、
  \eq*{
    n + s(m) = s(n + m) = s(m + n) = s(m) + n
  }

  \thmref{数学的帰納法}より、任意の$m$について示される。
}

\lem{自然数の順序の加法による保存}{
  \eq*{
    \forall n, m, l \in \N \qty(n < m \rightarrow n + l < m + l)
  }
}{
  $l = 0$のとき、明らか。

  ある$l$で成り立つとき、\lemref{自然数の順序の後者による保存}より、
  \eq*{
    n + s(l) = s(n + l) < s(m + l) = m + s(l)
  }

  \thmref{数学的帰納法}より、任意の$n$について示される。
}

\lem{自然数の順序の加法による特徴づけ}{
  \eq*{
    \forall n, m \in \N \qty(m \leq n \leftrightarrow \exists k \in \N \qty(m + k = n))
  }

  右について、この$k$は一意に定まる。
}{
  左は、全順序性、\lemref{自然数の順序の加法による保存}、$\forall k \in \N \qty(0 \leq k)$を用いて、背理法より示される。

  右について考える。

  $n = 0$のとき、$m = 0$ならば$k = 0$が存在。$m \neq 0$ならば前件否定。

  ある$n$で成り立つときを考える。$n < m \lor m \leq n$である。

  $n < m$のとき、\lemref{後者と順序}より$s(n) \leq m$である。
  $s(n) < m$のとき前件否定より自明。$s(n) = m$のとき、$k = 0$で存在。

  $m \leq n$のとき、仮定より$\exists k \in \N \qty(m + k = n)$

  ゆえに、$m \leq s(n)$かつ$s(n) = s(m + k) = m + s(k)$となる。

  \thmref{数学的帰納法}より、任意の$n$について示される。\\*

  $k < k'$とすると、\lemref{自然数の順序の加法による保存}より矛盾。$k > k'$でも同様より、示される。
}

\dfn{自然数の減法}{
  自然数$n, m$について、$m \leq n$であるとき、\lemref{自然数の順序の加法による特徴づけ}より一意に定まる$k$を$n - m$と表す。
}


\lsubsection{自然数の乗法}

\lem*{
  以下で定める部分写像$\times$は$\N \times \N$について左全域的である。
  \eqg*{
    \forall n \in \N \qty(\times(\qty(n, 0)) \coloneqq 0) \\*
    \forall n, m \in \N \qty(\times(\qty(n, s(m))) \coloneqq n + \times(\qty(n, m)))
  }
}{
  $\forall \qty(n, m) \in \N \times \N$に対して、$\times(\qty(n, m))$が存在することを示す。

  $m = 0$のとき定義より成り立つ。

  ある$m$で成り立つとすると、定義と加法の定義より$\qty(n, s(m))$でも成り立つ。

  \thmref{数学的帰納法}より、任意の$m$について示される。
}

\dfn{自然数の乗法}{
  \mlemref{0}より定まる二項演算を乗法と呼び、$\times$で表す。

  書き直して再掲する。
  \eqg*{
    \forall n \in \N \qty(n \times 0 \coloneqq 0) \\*
    \forall n, m \in \N \qty(n \times s(m) \coloneqq n + n \times m)
  }

  乗法$\times$は、加法$+$よりも先に演算される。

  また、$m \times n$を誤解のない範囲で$m n$と略記する。
}

\lem{左零元}{
  \eq*{
    \forall n \in \N \qty(0 \times n = 0)
  }
}{
  $n = 0$のとき、$0 \times 0 = 0$

  ある$n$で成り立つとき、
  \eq*{
    0 \times s(n) = 0 + 0 \times n = 0 + 0 = 0
  }

  \thmref{数学的帰納法}より、任意の$n$について示される。
}

\lem{自然数の乗法の単位元}{
  \eq*{
    \forall n \in \N \qty(n = n \times 1 = 1 \times n)
  }
}{
  左について$1 = s(0)$より、
  \eq*{
    n \times 1 = n \times s(0) = n + n \times 0 = n + 0 = n
  }

  右について考える。

  $n = 0$のとき、$1 \times 0 = 0$

  ある$n$で成り立つとき、
  \eq*{
    1 \times s(n) = 1 + 1 \times n = 1 + n = s(n)
  }

  \thmref{数学的帰納法}より、任意の$n$について示される。
}

\lem{自然数の右分配法則}{
  \eq*{
    \forall n, m, l \in \N \qty(\qty(n + m) \times l = n \times l + m \times l)
  }
}{
  $l = 0$のとき、以下より成り立つ。
  \eq*{
    \qty(n + m) \times 0 = 0 = 0 + 0 = n \times 0 + m \times 0
  }

  ある$l$で成り立つとき、
  \eqa*{
    \qty(n + m) \times s(l) &= n + m + \qty(n + m) \times l \\*
    &= n + m + n \times l + m \times l \\*
    &= n \times s(l) + m \times s(l)
  }

  \thmref{数学的帰納法}より、任意の$l$について示される。
}

\lem{自然数の乗法の交換法則}{
  \eq*{
    \forall n, m \in \N \qty(n \times m = m \times n)
  }
}{
  \lemref{左零元}より、$m = 0$について、$n \times 0 = 0 \times n$

  ある$m$で成り立つとき、
  \eqa*{
    n \times s(m) &= n + n \times m \\*
    &= n + m \times n \\*
    &= m \times n + n \\*
    &= m \times n + 1 \times n \\*
    &= \qty(m + 1) \times n \\*
    &= s(m) \times n
  }

  \thmref{数学的帰納法}より、任意の$n$について示される。
}

\lem{自然数の乗法の結合法則}{
  \eq*{
    \forall n, m, l \in \N \qty(\qty(n \times m) \times l = n \times \qty(m \times l))
  }
  つまり上の等式の両辺は、$n \times m \times l$と表記してもよい。
}{
  $l = 0$のとき、以下より成り立つ。
  \eq*{
    \qty(n \times m) \times 0 = 0 = n \times 0 = n \times \qty(m \times 0)
  }

  ある$l$で成り立つとき、
  \eqa*{
    \qty(n \times m) \times s(l) &= n \times m + \qty(n \times m) \times l \\*
    &= n \times m + n \times \qty(m \times l) \\*
    &= m \times n + \qty(m \times l) \times n \\*
    &= \qty(m + m \times l) \times n \\*
    &= n \times \qty(m + m \times l) \\*
    &= n \times \qty(1 \times m + l \times m) \\*
    &= n \times \qty(s(l) \times m) \\*
    &= n \times \qty(m \times s(l))
  }

  \thmref{数学的帰納法}より、任意の$l$について示される。
}

\lem{自然数の分配法則}{
  \eqg*{
    \forall n, m, l \in \N \qty(\qty(n + m) \times l = n \times l + m \times l) \\*
    \forall n, m, l \in \N \qty(n \times \qty(m + l) = n \times m + n \times l)
  }
}{
  右分配法則と交換法則より示される。
}

\lem{自然数の順序の乗法による保存}{
  \eq*{
    \forall n, m, l \in \N \qty(\qty(n \neq 0 \land m < l) \rightarrow n \times m < n \times l)
  }
}{
  $n = 0$のとき、前件否定より明らか。

  $n = 1$のとき、$1$が乗法の単位元であることから明らか。

  ある$n \geq 1$で成り立つとき、
  \eq*{
    s(n) \times m = n \times m + m < n \times m + l < n \times l + l = s(n) \times l
  }

  \thmref{数学的帰納法}より、任意の$n$について示される。
}

\lem{自然数の乗法の簡約則}{
  \eq*{
    \forall n, m, l \in \N \qty(n \times m = n \times l \land n \neq 0 \rightarrow m = l)
  }
}{
  $m < l$のとき、\lemref{自然数の順序の乗法による保存}より$n \times m < n \times l$。ゆえに、$n \times m \neq n \times l$。

  $l < m$のとき、同様に$n \times m \neq n \times l$。

  背理法より示される。
}

\lem*{
  \eq*{
    \forall n, m \in \N \qty(n \times m = 0 \rightarrow n = 0 \lor m = 0)
  }
}{
  $n = 0$のとき自明。

  $n \neq 0$のとき、$n \times m = n \times 0$より、自然数の乗法の簡約則から$m = 0$。
}


\lsubsection{自然数の除法}

\lem*{
  \eq*{
    \forall a, b \in \N \qty(b \neq 0 \rightarrow \exists q, r \in \N \qty(a = b q + r \land r < b))
  }
}{
  $a = 0$のとき、$q = r = 0$で成り立つ。

  ある$a$で成り立つとき、すなわち$n_a, r_a$が存在するときを考える。

  $r_a < b$より、$s(r_a) \leq b$である。

  $s(r_a) < b$のとき、$q_{s(a)} = n_a, r_{s(a)} = s(r_a)$は仮定より満たす。

  $s(r_a) = b$のとき、$q_{s(a)} = s(n_a), r_{s(a)} = 0$は仮定より満たす。

  \thmref{数学的帰納法}より、任意の$a$について成り立つ。
}

\lem*{
  \eq*{
    \forall a, b \in \N \qty(b \neq 0 \rightarrow \forall q, r, q', r' \in \N \qty(a = q b + r \land r < b \land a = q' b + r' \land r' < b \rightarrow q = q' \land r = r'))
  }
}{
  $q < q'$とすると\lemref{自然数の順序の加法による特徴づけ}より、$\exists q_d \in \N \qty(q + q_d = q' \land q_d \geq 1)$

  このとき$a = b q + r = b q' + r'$より、自然数の加法の簡約則から$r = b q_d + r'$

  これは$r < b \leq b q_d \leq b q_d + r$より矛盾。よって$q \geq q'$

  同様に$q \leq q'$。したがって、$q = q'$

  自然数の加法の簡約則から、$r = r'$

  よって示される。
}

\dfn{自然数の除法}{
  \mlemref{-1}、\mlemref{0}より、以下の写像$\divisionsymbol \colon \N \times \qty(\N \setminus \qty{0}) \to \N \times \N$を定義できる。
  \eq*{
    a \divisionsymbol b = q \cdots r \defiff a = b q + r
  }

  $\divisionsymbol (\qty(a, b))$を$a \divisionsymbol b$と、値域の元$\qty(q, r)$を$q \cdots r$と表記するものとする。

  特に、$q$を商、$r$を余りと呼ぶ。
}

\dfn{倍数}{
  自然数$n, m$について、$n \divisionsymbol m$の余りが$0$のとき、$n$は$m$の倍数、または$m$は$n$の約数と言う。
}

\cor*{
  任意の自然数$n$について、$0$は$n$の倍数であり、$1$は$n$の約数である。
  また、$n$は$n$の倍数でも約数でもある。
}

\dfn{偶奇}{
	自然数$n$について、$2$で割った余りが$0$であるとき、$n$を偶数と呼ぶ。

	そうでないとき、$n$を奇数と呼ぶ。
}

\dfn{素数}{
	約数をちょうど2つ持つ自然数を素数と呼ぶ。
}

\cor*{
  素数$p$の約数は、$1$と$p$である。
}

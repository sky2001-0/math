\lsection{位相環}

\lsubsection{位相環}

\dfn{位相環}{
  環$\qty(\qty(R, +), \times)$と、$X$の開集合系$\mathcal{O}$が以下が成り立つとき、順序対$\qty(\qty(\qty(R, +), \times), \mathcal{O})$を位相環と呼ぶ。
  \begin{enumerate}
    \item $\qty(\qty(R, +), \mathcal{O})$は位相群
    \item $\times \in R^{R \times R}$が連続
  \end{enumerate}
}

\lem{位相環の条件}{
  位相群$\qty(R, +)$について、$R$が位相環であることは、以下の3つを満たすことと同値である。
  \begin{itemize}
    \item $\times \in R^{R \times R}$は、$\qty(0_R, 0_R)$で連続。
    \item $\forall x_0 \in R$について、$l \in R^R, l(x) = x_0 x$は、$0_R$で連続。
    \item $\forall x_0 \in R$について、$r \in R^R, r(x) = x x_0$は、$0_R$で連続。
  \end{itemize}
}{
  必要性は明らか。\\*

  十分性を示す。

  $\forall \qty(x_0, y_0) \in R \times R$について、$x y = x_0 y_0 + \qty(x - x_0) \qty(y - y_0) + \qty(x - x_0) y_0 + x_0 \qty(x - y)$より、$\qty(x_0, y_0)$で連続。
}

\lem{位相環の収束ネットの積}{
  位相環$R$上のネット$\qty(x_\lambda)_{\lambda \in \Lambda}, \qty(y_\lambda)_{\lambda \in \Lambda}$が、$x, y \in R$に収束するならば、$\qty(x_\lambda y_\lambda)_{\lambda \in \Lambda}$は$x y$に収束するネットである。
}{
  \dfnref{位相環}と\thmref{点連続と収束}より成り立つ。
}

\lem{位相環と連続写像}{
  位相空間$X$と位相環$R$について、連続写像$f \in R^X, g \in R^X$を考える。

  このとき、写像$f \times g \in G^X, \qty(f \times g)(x) = f(x) \times g(x)$は連続である。
}{
  \dfnref{位相環}と\thmref{連続写像の合成}より成り立つ。
}

\lem{位相環上の多項式}{
  位相環$R$上の多項式$\varphi \in R[X]$について、式の値を与える$\varphi$は連続である。
}{
  定義より明らか。
}

\dfn{位相体}{
  体$\qty(\qty(F, +), \times)$と、$X$の開集合系$\mathcal{O}$が以下が成り立つとき、順序対$\qty(\qty(\qty(F, +), \times), \mathcal{O})$を位相体と呼ぶ。
  \begin{enumerate}
    \item $\qty(\qty(\qty(F, +), \times), \mathcal{O})$は位相環
    \item $-1 \in \qty(F^\times)^{F^\times}$が連続
  \end{enumerate}
}

\lem{位相体の収束ネットの逆}{
  Hausdorffな位相体$F$上のネット$\qty(x_\lambda)_{\lambda \in \Lambda}$が$x \in F^\times$に収束するならば、以下で定めるネット$\qty(y_\lambda)_{\lambda \in \Lambda}$は$1_F / x$に収束するネットである。
  \eq*{
    y_\lambda =
    \begin{cases}
      1_F / x_\lambda & \qty(x_\lambda \neq 0_F) \\*
      0_F & \qty(x_\lambda = 0_F)
    \end{cases}
  }
}{
  Hausdorffと収束の定義より$\exists B_0 \in \mathcal{B}(x) \exists \lambda_0 \in \Lambda \qty(0_F \notin B_0 \land \qty(x_\lambda)_{\lambda \in \Lambda_{\succcurlyeq \lambda_0}} \subset B_0)$

  \dfnref{位相体}と\thmref{点連続と収束}より成り立つ。
}


\lsubsection{順序環}

\dfn{順序環}{
  順序群$\qty(\qty(R, +), \leq)$について、順序対$\qty(\qty(R, +), \times)$が可換環でありかつ以下を満たすとき、順序対$\qty(\qty(\qty(R, +), \times), \leq)$を順序環と呼ぶ。
  \eq*{
    \forall a, b \in R \qty(0_R < a \land 0_R < b \rightarrow 0_R < a \times b)
  }
}

\rem{可換全順序環}{
  一般の順序環の定義は、半順序(全順序とは限らない)と両立する(可換とは限らない)環構造である。

  可換全順序環(環構造は可換環であり、半順序構造は全順序をなしている)を順序環と呼ぶことは、必ずしも一般的でないことに注意されたい。

}

\lem{順序環の性質}{
  順序環$R$について、以下が成り立つ。
  \eqg*{
    \forall a \in R \qty(a \neq 0_R \rightarrow 0_R < a a) \\*
    0_R < 1_R \\*
    \forall a \in R^\times \qty(0_R < a \rightarrow 0_R < a^{-1})
  }
}{
  $0_R < a$のとき順序環の定義より明らか。$a < 0_R$のとき$0_R < -a$であり、\lemref{環の性質}より$a a = (-a)(-a)$であるので示される。\\*

  零環でないので、$0_R \neq 1_R$である。すでに示した第一式より$0_R < 1_R 1_R = 1_R$\\*

  $a^{-1} \leq 0_R$とすると、$0_R \leq a (-a^{-1}) = -1_R$ゆえに$1_R \leq 0_R$となりすでに示した第二式に矛盾。
}

\lem{順序環の三角性}{
  順序環$R$について、以下が成り立つ。
  \eq*{
    \forall r \in R^+ \exists r' \in R^+ \qty(r' \times r' \leq r)
  }
}{
  $1_R \leq r$のとき、$r' = 1_R$として成り立つ。

  $0_R < r < 1_R$のとき、$r r < r$より、$r' = r$として成り立つ。
}

\thm{順序環は整域}{
  順序環は整域である。
}{
  整域でないとすると、$\exists a, b \in R \qty(a b = 0_R \land a \neq 0_R \land b \neq 0_R)$

  $r \coloneqq a, b \in R$について、$0_R < r \lor r < 0_R$で場合分けすることにより矛盾を得る。背理法より示される。
}

\dfn{順序体}{
  順序環が体であるとき、順序体と呼ぶ。
}

\thm{順序体の稠密性}{
  順序体$F$について、
  \eq*{
    \forall a, b \in F \qty(a < b \rightarrow \exists c \in F \qty(a < c \land c < b))
  }
}{
  $a, b \in F$として、$0_F < 1_F + 1_F$より、
  \eq*{
    c \coloneqq \qty(a + b) \times \qty(1_F + 1_F)^{-1}
  }
  は条件を満たす。
}


\lsubsection{位相環としての順序環}

\thm{順序環のノルム}{
  順序環$R$について、以下が成り立つ。
  \eqg*{
    \forall a \in G \qty(\abs{a} = 0_R \leftrightarrow a = 0_R) \\*
    \forall a, b \in G \qty(\abs{a + b} \leq \abs{a} + \abs{b}) \\*
    \forall a, b \in G \qty(\abs{a} \abs{b} = \abs{a b})
  }
}{
  \lemref{順序群のノルム}より、第一式、第二式は成り立つ。

  第三式について考える。$r \coloneqq a, b \in R$について、$0_R \leq r \lor r < 0_R$で場合分けすることにより得る。
}

\thm*{
  順序環$R$は位相環である。すなわち、$R$上の以下の演算は連続である。
  \begin{itemize}
    \item $\times \in R^{R \times R}$
  \end{itemize}
}{
  $\qty(0_R, 0_R)$について考える。

  $\forall B \in \mathcal{B}(0_R) \exists a, b \in R \qty(0_R \in \sqty{a, b} = B)$である。

  \lemref{順序環の三角性}より、$\exists r \in R^+ \qty(r \times r \leq \min \qty{-a, b})$である。

  $B' \coloneqq \sqty{-r, r} \times \sqty{-r, r} \mathcal{B} \times \mathcal{B}$であり、$\times(B') \subset B$\\*

  $\forall x_0 \in R$について、$f \in R^R, f(x) = x \times x_0$を考える。

  $x_0 = 0_R$のとき、定値写像より連続。

  $x_0 \neq 0_R$のとき、点$0_R$について考える。$\forall B \in \mathcal{B}(0_R) \exists a, b \in R \qty(0_R \in \sqty{a, b} = B)$である。

  $r \coloneqq \min \qty{-a, b} / \abs{x_0}$について、$B' \coloneqq \sqty{-r, r}$であり、$f(B') \subset B$\\*

  可換性と\lemref{位相環の条件}より位相環である。
}

\thm*{
  順序体$F$は位相体である。すなわち、$F$上の以下の演算は連続である。
  \begin{itemize}
    \item $-1 \in \qty(F^\times)^{F^\times}$
  \end{itemize}
}{
  開基$B = \sqty{a, b} \cap F^\times$を考える。

  $B = \varnothing$のとき、${-1}^{-1}(B) = \varnothing \in \mathcal{O}$である。

  $0 < a < b$のとき、${-1}^{-1}(B) = \sqty{1 / b, 1 / a} \in \mathcal{O}$である。

  $0 = a < b$のとき、${-1}^{-1}(B) = \qty{x \in F \mid 1 / b < x} \in \mathcal{O}$である。

  その他の場合も同様。ゆえに位相体である。
}

\lem{順序体上のCauchy列の積}{
  順序体$F$上のCauchy列$\qty(x_n)_{n \in \N}, \qty(y_n)_{n \in \N}$について、$\qty(x_n y_n)_{n \in \N}$はCauchy列となる。
}{
  \corref{Cauchy列は全有界}と\thmref{順序群では全有界ならば有界}より、2つのCauchy列は有界、すなわち$\exists K \in F^+ \forall n \in \N \qty(\abs{x_n} \leq K \land \abs{y_n} \leq K)$である。

  $2_F \coloneqq 1_F + 1_F$として、$\forall \epsilon \in F^+$について、
  \eqg*{
    \forall n, m \in \N \qty(n, m > N_1(\epsilon \times 2_F^{-1} \times K^{-1}) \rightarrow \abs{x_n - x_m} < \epsilon \times 2_F^{-1} \times K^{-1}) \\*
    \forall n, m \in \N \qty(n, m > N_2(\epsilon \times 2_F^{-1} \times K^{-1}) \rightarrow \abs{y_n - y_m} < \epsilon \times 2_F \times K^{-1}) \\*
  }

  ゆえに、
  \eq*{
    \forall n, m \in \N \qty(n, m > \max \qty{N_1(\epsilon \times 2_F^{-1} \times K^{-1}), N_2(\epsilon \times 2_F^{-1} \times K^{-1})} \rightarrow \abs{x_n y_n - x_m y_m} \leq \abs{x_n} \abs{y_n - y_m} + \abs{y_m} \abs{x_n - x_m} < \epsilon)
  }
}

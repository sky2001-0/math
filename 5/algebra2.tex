\lsection{代数の補足}

\lsubsectionf{$\Z, \Q$から見る代数}{Z, Qから見る代数}

\dfn{単生群}{
	群$G$について以下が成り立つとき、$G$を単生群と呼ぶ。
	\eq*{
		\exists g \in G \qty(G = \qty{g^n \mid n \in \Z})
	}
}

\thmf{$\Z$-加群としての可換群}{Z-加群としての可換群}{
  可換群$\qty(G, +)$を考える。
  以下で定める$\rho \in \End(G)^\Z$について、順序対$\qty(\qty(G, +), \rho)$は$\Z$-加群である。
  \eq*{
    \rho(n)(g) \coloneqq
    \begin{cases}
      \sum_{m \in n} g & \qty(0 \leq n) \\*
      \sum_{m \in (-n)} (-g) & \qty(n < 0)
    \end{cases}
  }
}{
  正負の場合の場合分けを考えて、\lemref{モノイド上の指数法則}より、成り立つ。
}

\lemf{分数体$\Q$}{分数体Q}{
	有理数$\Q$について、以下が成り立つ。
	\eq*{
		\forall q \in \Q \exists a, b \in \Z \qty(q = a / b)
	}
}{
	$\Q$の定義(\secref{有理数})より明らか。
}


\lsubsection{素体}


% \lem{標数}{
%   体$F$について、???
% }

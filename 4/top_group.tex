\lsection{位相群}

\lsubsection{位相群}

\dfn{位相群}{
  群$\qty(G, \cdot)$と、$G$の開集合系$\mathcal{O}$を考える。
  \dfnref{直積空間}の意味で、位相空間$G \times G$が与えられる。
  
  写像$\varphi \in G^{G \times G}, \varphi(x, y) = x \cdot y^{-1}$が連続であるとき、順序対$\qty(\qty(G, \cdot), \mathcal{O})$を位相群と呼ぶ。
}

\thm{位相群の自明な連続写像}{
  位相群$\qty(G, \cdot)$について、以下を満たす。
  \begin{enumerate}
    \item $\varphi_1 \in G^G, \varphi_1(x) = x^{-1}$は同相写像である。
    \item $\varphi_2 \in G^{G \times G}, \varphi_2(x, y) = x \cdot y$は連続写像である。
    \item $\forall a, b \in G$について、$\varphi_{a, b} \in G^G, \varphi_{a, b}(x) = a \cdot x \cdot y$は同相写像である。
  \end{enumerate}
}{
  $x^{-1} = e_G \cdot x^{-1}$であるので、\lemref{直積と連続}より連続である。

  $\varphi_1$の逆写像は$\varphi_1$であるため、同相写像である。\\*

  $x \cdot y = x \cdot \qty(y^{-1})^{-1}$であるため、連続である。\\*

  第一式から、\lemref{直積と連続}より連続である。
}

\thm{位相群の左一様構造}{
  位相群$G$と基本近傍系$\mathcal{B}(x)$について、以下で定める集合系$\mathcal{V}$は基本近縁系である。
  \eq*{
    \mathcal{V} \coloneqq \qty{\tilde{B} \coloneqq \qty{\qty(x, y) \in G \times G \mid x^{-1} y \in B} \mid B \in \mathcal{B}(e)}
  }

  さらに、基本近縁系$\mathcal{V}$の定める近傍系$\mathcal{N'}(x)$は、基本近傍系$\mathcal{B}(x)$の定める近傍系$\mathcal{N}(x)$に一致する。
}{
  \dfnref{基本近傍系}第一式から$\mathcal{V} \neq \varnothing$\\*

  $\forall V \in \mathcal{V}$について\dfnref{基本近傍系}第二式から$\forall x \in G \forall B \in \mathcal{B}(e) \qty(x^{-1} x = e \in B)$である。ゆえに$\qty(x, x) \in V$\\*

  $\forall V_1, V_2 \in \mathcal{V}$について、$\exists B_1, B_2 \in \mathcal{B}(e) \qty(V_1 = \tilde{B_1} \land V_2 = \tilde{B_2})$である。

  \dfnref{基本近傍系}第三式より$\exists B \in \mathcal{B}(e) \qty(B \subset B_1 \cap B_2)$であり、定義より$\tilde{B} \subset V_1 \cap V_2 \land \tilde{B} \in \mathcal{V}$\\*

  $\forall V \in \mathcal{V} \exists B \in \mathcal{B}(e) \qty(V = \tilde{B})$である。

  \thmref{位相群の自明な連続写像}第二と\thmref{連続と点連続}より$\exists B_1, B_2 \in \mathcal{B}(e) \qty(\cdot(B_1 \times B_2) \subset B)$

  \dfnref{基本近傍系}第三式より$\exists B' \in \mathcal{B}(e) \qty(\cdot(B' \times B') \subset \cdot(B_1 \times B_2) \subset B)$

  定義より$\tilde{B'} \circ \tilde{B'} \subset \tilde{\cdot \qty(B' \times B')} \subset \tilde{B} = V$\\*

  $\forall V \in \mathcal{V} \exists B \in \mathcal{B}(e) \qty(V = \tilde{B})$である。

  \thmref{位相群の自明な連続写像}第一と\thmref{連続と点連続}より$\exists B' \in \mathcal{B}(e) \qty(-1(B') \subset B)$となるので、$\tilde{B'} \subset V^{-1}$\\*

  $\mathcal{V}$の与える基本近傍系は$\mathcal{B'}(x) = \qty{\tilde{B}[x] \mid B \in \mathcal{B}(e)}$である。\\*

  \thmref{位相群の自明な連続写像}第二と\lemref{直積と連続}より$\forall D \in \mathcal{B}(x) \exists B \in \mathcal{B}(e) \qty(\qty{x y \mid y \in B} \subset D)$

  $\tilde{B}[x] = \qty{y \in G \mid x^{-1} y \in B} = \qty{x z \mid z \in B} \subset D$\\*

  \thmref{位相群の自明な連続写像}第二と\lemref{直積と連続}より$\forall B \in \mathcal{B}(e) \exists D \in \mathcal{B}(x) \qty(\qty{x^{-1} y \mid y \in D} \subset B)$

  $\tilde{B}[x] = \qty{y \in G \mid x^{-1} y \in B} \supset \qty{z \in G \mid x^{-1} z \in \qty{x^{-1} y \mid y \in D}} = D$

  ゆえに\lemref{近傍系の一意性}より成り立つ。
}

\cor*{
  第一可算な位相群は、可算一様空間である。
}

\thm{位相群の自明な一様連続写像}{
  位相群$\qty(G, \cdot)$について、以下を満たす。
  \begin{enumerate}
    \item $\varphi_1 \in G^G, \varphi_1(x) = x^{-1}$は一様同型写像である。
    \item $\forall a, b \in G$について、$\varphi_{a, b} \in G^G, \varphi_{a, b}(x) = a \cdot x \cdot y$は一様同型写像である。
  \end{enumerate}
}{
  $x^{-1} = e_G \cdot x^{-1}$であるので、\lemref{直積と連続}より連続である。

  $\varphi_1$の逆写像は$\varphi_1$であるため、同相写像である。\\*

  $x \cdot y = x \cdot \qty(y^{-1})^{-1}$であるため、連続である。\\*

  第一式から、\lemref{直積と連続}より連続である。
}

\lemf{位相群は$T_3$}{位相群はT_3}{
  位相群は$T_3$空間である。
}{
  \thmref{位相群の左一様構造}、\thmref{一様空間はT_3}より成り立つ。
}

\thm{位相群と分離}{
  位相群$G$について、以下の3つは同値である。
  このとき、$G$はHausdorffであると呼ぶ。
  \begin{enumerate}
    \item $G$は$T_2$空間
    \item $G$は$T_1$空間
    \item $\qty{e}$は閉集合
  \end{enumerate}
}{
  $1. \rightarrow 2.$は、\lemref{T_2ならばT_1}より明らか。\\*

  $2. \rightarrow 3.$は、\thmref{Frechet性}より明らか。\\*

  $3. \rightarrow 1.$を示す。

  $\qty{\qty(x, x) \mid x \in G} = \cdot^{-1}(\qty{e}) \in \mathcal{O}$より\thmref{Hausdorff性}から示される。
}


\lsubsection{可換位相群}

\dfn{可換位相群}{
  位相群$G$の群構造が可換群であるとき、$G$を可換位相群と呼ぶ。
}

\lem{可換位相群の一様的な特徴}{
  可換位相群$G$について、以下を満たす。
  \begin{enumerate}
    \item $\varphi_1 \in G^G, \varphi_1(x) = x^{-1}$は一様同型写像である。
    \item $G$上のCauchyネット$\qty(x_\lambda)_{\lambda \in \Lambda}, \qty(y_\lambda)_{\lambda \in \Lambda}$について、$\qty(x_\lambda y_\lambda)_{\lambda \in \Lambda}$はCauchyネットである。
  \end{enumerate}
}{
  \thmref{位相群の左一様構造}の$\mathcal{V}$を考える。\\*

  $\forall V \in \mathcal{V}$について、\lemref{基本近縁系の三角性}より$\exists W \in \mathcal{V} \qty(W \cup W^{-1} \subset V)$である。
  
  $\qty(x, y) \in W$について、可換性より$\qty(y^{-1}, x^{-1}) \in W$であるため、$\qty(x^{-1}, y^{-1}) \in V$となる。
  
  よって一様連続である。
  $\varphi_1$の逆写像は$\varphi_1$であるため、一様同型写像である。\\*

  $\forall V \in \mathcal{V}$について、$\exists W \in \mathcal{V} \qty(W \circ W \subset V)$

  Cauchyより、$\exists \lambda_x \in \Lambda \forall \lambda_1, \lambda_2 \in \Lambda_{\succcurlyeq \lambda_x} \qty(\qty(x_{\lambda_1}, x_{\lambda_2}) \in W)$かつ$\exists \lambda_y \in \Lambda \forall \lambda_1, \lambda_2 \in \Lambda_{\succcurlyeq \lambda_y} \qty(\qty(y_{\lambda_1}, y_{\lambda_2}) \in W)$

  有向性より、$\exists \lambda_0 \in \Lambda \forall \lambda_1, \lambda_2 \in \Lambda_{\succcurlyeq \lambda_0} \qty(\qty(x_{\lambda_1}, x_{\lambda_2}), \qty(y_{\lambda_1}, y_{\lambda_2}) \in W)$

  $\qty(x_{\lambda_1} y_{\lambda_1}, x_{\lambda_2} y_{\lambda_1}), \qty(x_{\lambda_2} y_{\lambda_1}, x_{\lambda_2} y_{\lambda_2}) \in W$より、$\qty(x_{\lambda_1} y_{\lambda_1}, x_{\lambda_2} y_{\lambda_2}) \in V$
}

\dfn{総和可能}{
  Hausdorffな可換位相群$G$について、$G$上の点列$\qty(x_n)_{n \in \N}$を考える。

  点列$\qty(\sum_{m \in n} x_m)_{n \in \N}$が$G$上の点$s$に収束するとき、点列$\qty(x_n)_{n \in \N}$は総和可能と呼ぶ。

  このとき、$\sum_{m \in \N} x_m = s$と表記する。
}


\lsubsection{順序群}

\dfn{順序群}{
  可換群$\qty(G, +)$上の全順序$\leq$が以下を満たすとき、順序対$\qty(\qty(G, +), \leq)$を順序群と呼ぶ。
	または単に$G$と書いて順序群を表す。
  \eq*{
    \forall a, b, c \in G \qty(a \leq b \rightarrow a + c \leq b + c)
  }
}

\rem{可換全順序群}{
  一般の順序群の定義は、半順序(全順序とは限らない)と両立する(可換とは限らない)群構造である。

  可換全順序群(群構造は可換群であり、半順序構造は全順序をなしている)を順序群と呼ぶことは、必ずしも一般的でないことに注意されたい。
}

\cor*{
  順序群$G$について、以下が成り立つ。
  \eq*{
    \forall a, b, c \in G \qty(a < b \rightarrow a + c < b + c)
  }
}

\dfn{正錐}{
  順序群$G$について、以下で定める集合$G^+$を$G$の正錐と呼ぶ。
  \eq*{
    G^+ \coloneqq \qty{g \in G \mid 0_G < g}
  }
}

\cor*{
  順序群$G$が自明であることと、正錐が空であることは必要十分である。
}

\lem{非自明な順序群は最大を持たない}{
  非自明な順序群$G$は、最大元と最小元を持たない。
}

\lem{順序群の三角性}{
  非自明な順序群$G$について、$G^+$が最小元を持たないとき以下が成り立つ。
  \eq*{
    \forall g \in G^+ \exists g' \in G^+ \qty(g' + g' \leq g)
  }
}{
  最小元を持たないので、$\exists g_1 \in G^+ \qty(g_1 < g)$である。

  $g_1 + g_1 \leq g$のとき、$g' = g_1$として明らか。

  $g_1 + g_1 > g$のとき、$g' \coloneqq g - g_1$について、$g' + g' = g + g - g_1 - g_1 < g$
}

\thm{順序群は位相群}{
  順序群$\qty(\qty(G, +), \leq)$は位相群である。
}{
  順序群$G$が自明であるならば、明らか。
  非自明な場合を考える。

  写像$\varphi \in G^{G \times G}, \varphi(x, y) = x - y$を考える。

  $\forall \qty(x, y) \in G \times G$における点連続性を考える。
  
  $\forall a, b \in G \qty(\varphi(x, y) \in \sqty{a, b})$を考える。\\*

  $G^+$が最小元$m$を持つとき、$\qty{x} = \sqty{x - m, x + m} \subset \sqty{a, b}$である。
  同様に$\qty{y} \subset \sqty{a, b}$でもある。

  $\forall B \in \mathcal{B}(\varphi(x, y)) \qty(\varphi(\qty{x} \times \qty{y}) = \qty{\varphi(x, y)} \subset B)$である。\\*

  $G^+$が最小元を持たないとき、\lemref{順序群の三角性}より$\exists g \in G^+ \qty(g + g \leq \min \qty{\varphi(x, y) - a, b - \varphi(x, y)})$である。

  ゆえに、$\varphi \qty(\sqty{x - g, x + g} \times \sqty{y - g, y + g}) \subset \sqty{a, b}$である。\\*

  \dfnref{順序空間}の定める開基$\mathcal{B}$について、\lemref{最大がない場合の順序の開基}、\lemref{非自明な順序群は最大を持たない}より、$\forall B \in \mathcal{B} \exists a, b \in G \qty(\sqty{a, b} = B)$である。

  \thmref{連続と点連続}より$\varphi$は連続である。
}

\dfn{絶対値}{
  順序群$G$について、以下で定める写像$\abs{} \in G^G$を絶対値と呼ぶ。
  \eq*{
    \abs{a} \coloneqq \max \qty{a, -a}
  }
}

\cor*{
  順序群$G$について以下が成り立つ。
  \eq*{
    \forall a \in G \qty(0_G \leq \abs{a})
  }
}

\lem{順序群のノルム}{
  順序群$G$について、以下が成り立つ。
  \eqg*{
    \forall a \in G \qty(\abs{a} = 0_G \leftrightarrow a = 0_G) \\*
    \forall a, b \in G \qty(\abs{a + b} \leq \abs{a} + \abs{b}) \\*
    \forall a \in G \qty(\abs{a} = \abs{-a})
  }
}{
  $g \coloneqq a, b \in G$について、$0_G \leq g \lor g < 0_G$で場合分けすることにより得る。
}

\thm{順序群の一様構造}{
  順序群$G$について、以下で定める集合系$\mathcal{V}$は基本近縁系である。
  \eq*{
    \mathcal{V} \coloneqq \qty{V_g \coloneqq \qty{\qty(x, y) \in G \times G \mid \abs{x - y} < g} \mid g \in G^+}
  }

  さらに、基本近縁系$\mathcal{V}$の定める近縁系$\mathcal{U'}$は、\thmref{位相群の左一様構造}の定める近縁系$\mathcal{U}$に一致する。
}{
	$\forall g \in G^+ \exists B \in \mathcal{B}(0_G) \qty(B = \sqty{-g, g} \land \tilde{B} = V_g)$である。

	$\forall B \in \mathcal{B}(0_G)$について、$B = \sqty{a, b} \land a < 0_G < b$である。$g = \min \qty{-a, b} \in G^+$について、$V_g \subset \tilde{B}$である。

	\lemref{近縁系の一意性}より成り立つ。
}

\lem{絶対値は一様連続}{
  順序群$\qty(\qty(G, +), \leq)$上の絶対値$\abs{}$は一様連続である。
}{
  $\forall g \in G^+ \qty(\qty(\abs{x}, \abs{y}) \in V_g)$であるので、成り立つ。
}

\thm{順序群では全有界ならば有界}{
  順序群$G$の部分集合$A$について、$A$が全有界ならば有界である。
}{
  非自明性より$\exists g \in G^+$である。

  全有界より$\exists Y \in \P(A) \qty(\abs{Y} < \infty \land A \subset \qty{B(g)[y] \mid y \in Y})$である。

  \thmref{有限全順序集合の最大元}より$\exists u, d \in G \qty(u = \max Y \land d = \min Y)$である。

  全有界性より$\forall a \in A \qty(d - g < a \land a < u + g)$
}

\thmf{$\epsilon$-$\delta$論法}{epsilon-delta論法}{
  順序群$G$上のネット$\qty(a_\lambda)_{\lambda \in \Lambda}$を考える。

  $\qty(a_\lambda)_{\lambda \in \Lambda}$が$r \in G$に収束することは、以下を満たすことと必要十分である。
  \eq*{
    \forall \epsilon \in G^+ \exists \lambda_0 \in \Lambda \forall \lambda \in \Lambda_{\geq \lambda_0} \qty(\abs{a_\lambda - r} < \epsilon)
  }

  $\qty(a_\lambda)_{\lambda \in \Lambda}$がCauchyであることは、以下を満たすことと必要十分である。
  \eq*{
    \forall \epsilon \in G^+ \exists \lambda_0 \in \Lambda \forall \lambda_1, \lambda_2 \in \Lambda_{\geq \lambda_0} \qty(\abs{a_{\lambda_1} - a_{\lambda_2}} < \epsilon)
  }
}{
  \thmref{順序群の一様構造}より明らか。
}

\dfn{Archimedes}{
  順序群$G$が以下を満たすとき、$G$はArchimedes的であると呼ぶ。
  \eq*{
    \forall x, y \in G^+ \exists n \in \N \qty(x < \sum_{m \in n} y)
  }
}

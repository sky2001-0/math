\lsection{位相群}

\lsubsection{位相群}

\dfn{位相群}{
  群$\qty(G, \cdot)$と、$X$の開集合系$\mathcal{O}$が以下が成り立つとき、順序対$\qty(\qty(G, \cdot), \mathcal{O})$を位相群と呼ぶ。
  \begin{enumerate}
    \item $\cdot \in G^{G \times G}$が連続
    \item 逆元を与える写像$-1 \in G^G$が連続
  \end{enumerate}
}

\dfn{部分位相群}{
  位相群$\qty(\qty(G, \cdot), \mathcal{O})$の部分群$\qty(H, \cdot)$について、$\qty(\qty(H, \cdot), \qty{H \cap O \mid O \in \mathcal{O}})$は位相群である。
  これを位相群$G$の部分位相群と呼ぶ。
}

\lem{位相群の収束ネットの積と逆}{
  位相群$G$上のネット$\qty(x_\lambda)_{\lambda \in \Lambda}, \qty(y_\lambda)_{\lambda \in \Lambda}$が、$x, y \in G$に収束するならば、$\qty(x_\lambda y_\lambda)_{\lambda \in \Lambda}, \qty(x_\lambda^{-1})_{\lambda \in \Lambda}$は$x y, x^{-1}$に収束するネットである。
}{
  \dfnref{位相群}と\thmref{点連続と収束}より成り立つ。
}

\lem{位相群と連続写像}{
  位相空間$X$と位相群$G$について、連続写像$f \in G^X, g \in G^X$を考える。

  このとき、写像$f \cdot g \in G^X, \qty(f \cdot g) (x) = f(x) \cdot g(x)$と、写像$\hat{f} \in G^X, \hat{f}(x) = f(x)^{-1}$は連続である。
}{
  \dfnref{位相群}と\thmref{連続写像の合成}より成り立つ。
}

\thm{位相群と分離}{
  位相群$G$について、以下の3つは同値である。このとき、$G$はHausdorffであると呼ぶ。
  \begin{enumerate}
    \item $G$は$T_2$空間
    \item $G$は$T_1$空間
    \item $\qty{e}$は閉集合
  \end{enumerate}
}{
  $1. \rightarrow 2.$は、\lemref{T_2ならばT_1}より明らか。\\*

  $2. \rightarrow 3.$は、\thmref{Frechet性}より明らか。\\*

  $3. \rightarrow 1.$を示す。

  $\qty{\qty(x, x) \mid x \in G} = \cdot^{-1}(\qty{e}) \in \mathcal{O}$より\thmref{Hausdorff性}から示される。
}

\thm{位相群の左一様構造}{
  位相群$G$と基本近傍系$\mathcal{B}(x)$について、以下で定める集合系$\mathcal{V}$は基本近縁系である。
  \eq*{
    \mathcal{V} \coloneqq \qty{\tilde{B} \coloneqq \qty{\qty(x, y) \in G \times G \mid x^{-1} y \in B} \mid B \in \mathcal{B}(e)}
  }

  さらに、基本近縁系$\mathcal{V}$の定める近傍系$\mathcal{N'}(x)$は、基本近傍系$\mathcal{B}(x)$の定める近傍系$\mathcal{N}(x)$に一致する。
}{
  \dfnref{基本近傍系}第一式から$\mathcal{V} \neq \varnothing$\\*

  $\forall V \in \mathcal{V}$について\dfnref{基本近傍系}第二式から$\forall x \in G \forall B \in \mathcal{B}(e) \qty(x^{-1} x = e \in B)$である。ゆえに$\qty(x, x) \in V$\\*

  $\forall V_1, V_2 \in \mathcal{V}$について、$\exists B_1, B_2 \in \mathcal{B}(e) \qty(V_1 = \tilde{B_1} \land V_2 = \tilde{B_2})$である。

  \dfnref{基本近傍系}第三式より$\exists B \in \mathcal{B}(e) \qty(B \subset B_1 \cap B_2)$であり、定義より$\tilde{B} \subset V_1 \cap V_2 \land \tilde{B} \in \mathcal{V}$\\*

  $\forall V \in \mathcal{V} \exists B \in \mathcal{B}(e) \qty(V = \tilde{B})$である。

  \dfnref{位相群}第一式と\thmref{連続と点連続}より$\exists B_1, B_2 \in \mathcal{B}(e) \qty(\cdot(B_1 \times B_2) \subset B)$

  \dfnref{基本近傍系}第三式より$\exists B' \in \mathcal{B}(e) \qty(\cdot(B' \times B') \subset \cdot(B_1 \times B_2) \subset B)$

  定義より$\tilde{B'} \circ \tilde{B'} \subset \tilde{\cdot \qty(B' \times B')} \subset \tilde{B} = V$\\*

  $\forall V \in \mathcal{V} \exists B \in \mathcal{B}(e) \qty(V = \tilde{B})$である。

  \dfnref{位相群}第二式と\thmref{連続と点連続}より$\exists B' \in \mathcal{B}(e) \qty(-1(B') \subset B)$となるので、$\tilde{B'} \subset V^{-1}$\\*

  $\mathcal{V}$の与える基本近傍系は$\mathcal{B'}(x) = \qty{\tilde{B}[x] \mid B \in \mathcal{B}(e)}$である。\\*

  \dfnref{位相群}第一式と\lemref{直積と連続}より$\forall D \in \mathcal{B}(x) \exists B \in \mathcal{B}(e) \qty(\qty{x y \mid y \in B} \subset D)$

  $\tilde{B}[x] = \qty{y \in G \mid x^{-1} y \in B} = \qty{x z \mid z \in B} \subset D$\\*

  \dfnref{位相群}第一式と\lemref{直積と連続}より$\forall B \in \mathcal{B}(e) \exists D \in \mathcal{B}(x) \qty(\qty{x^{-1} y \mid y \in D} \subset B)$

  $\tilde{B}[x] = \qty{y \in G \mid x^{-1} y \in B} \supset \qty{z \in G \mid x^{-1} z \in \qty{x^{-1} y \mid y \in D}} = D$

  ゆえに\lemref{近傍系の一意性}より成り立つ。
}

\lem{位相群のCauchyネットの積}{
  位相群$G$上のCauchyネット$\qty(x_\lambda)_{\lambda \in \Lambda}, \qty(y_\lambda)_{\lambda \in \Lambda}$について、$\qty(x_\lambda y_\lambda)_{\lambda \in \Lambda}$はCauchyネットである。
}{
  $\forall V \in \mathcal{V}$について、$\exists W \in \mathcal{V} \qty(W \circ W \subset V)$

  Cauchyより、$\exists \lambda_x \in \Lambda \forall \lambda_1, \lambda_2 \in \Lambda_{\succcurlyeq \lambda_x} \qty(\qty(x_{\lambda_1}, x_{\lambda_2}) \in W)$かつ$\exists \lambda_y \in \Lambda \forall \lambda_1, \lambda_2 \in \Lambda_{\succcurlyeq \lambda_y} \qty(\qty(y_{\lambda_1}, y_{\lambda_2}) \in W)$

  有向性より、$\exists \lambda_0 \in \Lambda \forall \lambda_1, \lambda_2 \in \Lambda_{\succcurlyeq \lambda_0} \qty(\qty(x_{\lambda_1}, x_{\lambda_2}), \qty(y_{\lambda_1}, y_{\lambda_2}) \in W)$

  \thmref{位相群の左一様構造}の定義より$\qty(x_{\lambda_1} y_{\lambda_1}, x_{\lambda_2} y_{\lambda_1}), \qty(x_{\lambda_2} y_{\lambda_1}, x_{\lambda_2} y_{\lambda_2}) \in W$より、$\qty(x_{\lambda_1} y_{\lambda_1}, x_{\lambda_2} y_{\lambda_2}) \in V$
}

\lem{可換位相群のCauchyネットの逆}{
  可換位相群$G$上のCauchyネット$\qty(x_\lambda)_{\lambda \in \Lambda}$について、$\qty(x_\lambda^{-1})_{\lambda \in \Lambda}$はCauchyネットである。
}{
  $\forall V \in \mathcal{V}$について、$\exists W \in \mathcal{V} \qty(W \subset V^{-1})$

  Cauchyより、$\exists \lambda_0 \in \Lambda \forall \lambda_1, \lambda_2 \in \Lambda_{\succcurlyeq \lambda_x} \qty(\qty(x_{\lambda_1}, x_{\lambda_2}) \in W)$である。

  $\qty(x_{\lambda_2}, x_{\lambda_1}) \in V$である。可換性と\thmref{位相群の左一様構造}の定義より$\qty(x_{\lambda_1}^{-1}, x_{\lambda_2}^{-1}) \in V$
}

\dfn{総和可能}{
  Hausdorffな可換位相群$G$について、$G$上の点列$\qty(x_n)_{n \in \N}$を考える。

  点列$\qty(\sum_{m = 0}^n x_m)_{n \in \N}$が、$G$上の点$s$に収束するとき、点列$\qty(x_n)_{n \in \N}$は総和可能と呼ぶ。

  このとき、$\sum_{m = 0}^\infty x_m \coloneqq s$と表記する。
}


\lsubsection{順序群}

\dfn{順序群}{
  可換群$\qty(G, +)$上の全順序$\leq$が以下を満たすとき、順序対$\qty(\qty(G, +), \leq)$を順序群と呼ぶ。
	または単に$G$と書いて順序群を表す。
  \eq*{
    \forall a, b, c \in G \qty(a \leq b \rightarrow a + c \leq b + c)
  }
}

\rem{順序群の定義}{
  一般の順序群の定義は、半順序(全順序とは限らない)と両立する(可換とは限らない)群構造である。
  より強い意味で使っていることに注意されたい。
}

\cor*{
  順序群$G$について、以下が成り立つ。
  \eq*{
    \forall a, b, c \in G \qty(a < b \rightarrow a + c < b + c)
  }
}

\dfn{正錐}{
  順序群$G$について、以下で定める集合$G^+$を$G$の正錐と呼ぶ。
  \eq*{
    G^+ \coloneqq \qty{g \in G \mid 0_G < g}
  }
}

\cor*{
  順序群$G$が自明であることと、正錐が空であることは必要十分である。
}

\cor*{
  非自明な順序群$G$は、最大元と最小元を持たない。
}

\dfn{Archimedes}{
  順序群$G$が以下を満たすとき、$G$はArchimedes的であると呼ぶ。
  \eq*{
    \forall x, y \in G^+ \exists n \in \N \qty(x < \sum_{m \in n} y)
  }
}

\lem{順序群の三角性}{
  非自明な順序群$G$について、$G^+$が最小元を持たないとき以下が成り立つ。
  \eq*{
    \forall g \in G^+ \exists g' \in G^+ \qty(g' + g' \leq g)
  }
}{
  最小元を持たないので、$\exists g_1 \in G^+ \qty(g_1 < g)$である。

  $g_1 + g_1 \leq g$のとき、$g' = g_1$として明らか。

  $g_1 + g_1 > g$のとき、$g' \coloneqq g - g_1$について、$g' + g' = g + g - g_1 - g_1 < g$
}

\thm{順序群は位相群}{
  順序群$\qty(\qty(G, +), \leq)$上の以下の2つの演算は連続である。
  \begin{enumerate}
    \item $+ \in G^{G \times G}$
    \item $- \in G^G$
  \end{enumerate}
}{
  \lemref{順序の開基}の定める開基$\mathcal{B}$を考える。\\*

  $\forall \qty(x, y) \in G$について考える。

  $G^+$が最小元$m$を持つならば$\qty{x} = \sqty{x - m, x + m} \in \mathcal{B}(x)$である。同様に$\qty{y} \in \mathcal{B}(y)$

  $\forall B \in \mathcal{B}(x + y) \qty(+(\qty{x} \times \qty{y}) = \qty{x + y} \subset B)$

  $G^+$が最小元を持たないならば$\forall B \in \mathcal{B}(x + y) \exists a, b \in G \qty(x + y \in \sqty{a, b} = B)$である。

  \lemref{順序群の三角性}より$\exists g \in G^+ \qty(g + g \leq \min \qty{x + y - a, b - (x + y)})$である。

  $B' \coloneqq \sqty{x - g, x + g} \times \sqty{y - g, y + g} \in \mathcal{B} \times \mathcal{B}$であり、$+(B') \subset B$

  \thmref{連続と点連続}より成り立つ。\\*

  $\forall B \coloneqq \sqty{a, b} \in \mathcal{B}$について考える。$f = -$とする。

  $f^{-1}(B) = \sqty{-b, -a} \in \mathcal{B}$であるので連続。
}

\dfn{絶対値}{
  順序群$G$について、以下で定める写像$\abs{} \in G^G$を絶対値と呼ぶ。
  \eq*{
    \abs{a} \coloneqq \max \qty{a, -a}
  }
}

\cor*{
  順序群$G$について以下が成り立つ。
  \eq*{
    \forall a \in G \qty(0_G \leq \abs{a})
  }
}

\lem{順序群のノルム}{
  順序群$G$について、以下が成り立つ。
  \eqg*{
    \forall a \in G \qty(\abs{a} = 0_G \leftrightarrow a = 0_G) \\*
    \forall a, b \in G \qty(\abs{a + b} \leq \abs{a} + \abs{b}) \\*
    \forall a \in G \qty(\abs{a} = \abs{-a})
  }
}{
  $g \coloneqq a, b \in G$について、$0_G \leq g \lor g < 0_G$で場合分けすることにより得る。
}

\lem{絶対値は連続}{
  順序群$\qty(\qty(G, +), \leq)$上の演算$\abs{}$は連続である。
}{
  \lemref{順序の開基}の定める開基$\mathcal{B}$を考える。

  $\forall B \coloneqq \qty{x \in G \mid a < x \land x < b} \in \mathcal{B}$について考える。

  $b \leq a \lor b < 0_G$ならば、$\abs{}^{-1}(B) = \abs{}^{-1}(\varnothing) = \varnothing \in \mathcal{O}$である。

  $a < 0_G \leq b$であるとき、$\abs{}^{-1}(B) = \sqty{-b, b} \in \mathcal{B} \subset \mathcal{O}$である。

  $0_G \leq a$であるとき、$\abs{}^{-1}(B) = \sqty{-b, -a} \cup \sqty{a, b} \in \mathcal{O}$である。

  \lemref{開基と連続}より連続である。
}

\thm{順序群の一様構造}{
  順序群$G$について、以下で定める集合系$\mathcal{V}$は基本近縁系である。
  \eq*{
    \mathcal{V} \coloneqq \qty{V_g \coloneqq \qty{\qty(x, y) \in G \times G \mid \abs{x - y} < g} \mid g \in G^+}
  }

  さらに、基本近縁系$\mathcal{V}$の定める近縁系$\mathcal{U'}$は、\thmref{位相群の左一様構造}の定める近縁系$\mathcal{U}$に一致する。
}{
	$\forall g \in G^+ \exists B \in \mathcal{B}(0_G) \qty(B = \sqty{-g, g} \land \tilde{B} \subset V_g)$である。

	$\forall B \in \mathcal{B}(0_G)$について、$B = \sqty{a, b} \land a < 0_G < b$である。$g = \min \qty{-a, b} \in G^+$について、$V_g \subset \tilde{B}$である。

	\lemref{近縁系の一意性}より成り立つ。
}

\thm{順序群では全有界ならば有界}{
  順序群$G$の部分集合$A$について、$A$が全有界ならば有界である。
}{
  非自明性より$\exists g \in G^+$である。

  全有界より$\exists Y \in \P(A) \qty(\abs{Y} < \infty \land A \subset \qty{B(g)[y] \mid y \in Y})$である。

  \thmref{有限全順序集合の最大元}より$\exists u, d \in G \qty(u = \max Y \land d = \min Y)$である。

  全有界性より$\forall a \in A \qty(d - g < a \land a < u + g)$
}

\thmf{$\epsilon$-$\delta$論法}{epsilon-delta論法}{
  順序群$G$上の点列$\qty(a_n)_{n \in \N}$を考える。

  $\qty(a_n)_{n \in \N}$が$r \in G$に収束することは、以下を満たすことと必要十分である。
  \eq*{
    \forall \epsilon \in G^+ \exists N \in \N \forall n \in \N \qty(N \leq n \rightarrow \abs{a_n - r} < \epsilon)
  }

  $\qty(a_n)_{n \in \N}$がCauchyであることは、以下を満たすことと必要十分である。
  \eq*{
    \forall \epsilon \in G^+ \exists N \in \N \forall n, m \in \N \qty(N \leq n \land N \leq m \rightarrow \abs{a_n - a_m} < \epsilon)
  }
}{
  \thmref{順序群の一様構造}より明らか。
}

\lem{順序群上のCauchy列の絶対値}{
  順序群$G$上のCauchy列$\qty(x_n)_{n \in \N}$について、$\qty(\abs{x_n})_{n \in \N}$はCauchy列となる。

  さらに、$\qty(x_n)_{n \in \N}$が$a$に収束するとき、$\qty(\abs{x_n})_{n \in \N}$は$\abs{a}$に収束する。
}{
  \lemref{順序群のノルム}より$\abs{\abs{x_n} - \abs{x_m}} \leq \abs{x_n - x_m}$であるのでCauchy列である。

  \lemref{順序群のノルム}より$\abs{\abs{x_n} - \abs{a}} \leq \abs{\abs{x_n - a}} = \abs{x_n - a}$であるので、$\abs{a}$に収束する。
}

\lsection{有理数}

\lsubsection{有理数の構成}

\lem{有理数の準備:前順序}{
  直積集合$\Z \times \qty(\Z \setminus \qty{0})$上で以下の自己関係$\preccurlyeq_Q$は前順序である。
  \eq*{
    \forall \qty(a_1, a_2), \qty(b_1, b_2) \in \Z \times \qty(\Z \setminus \qty{0}) \qty(\qty(a_1, a_2) \preccurlyeq_Q \qty(b_1, b_2) \defiff a_1 a_2 b_2 b_2 \leq b_1 a_2 a_2 b_2)
  }
}{
  反射的である。\\*

  推移的であることを示す。
  \eqg*{
    a_1 a_2 b_2 b_2 \leq a_2 a_2 b_1 b_2 \land b_1 b_2 c_2 c_2 \leq b_2 b_2 c_1 c_2 \\*
    a_1 a_2 b_2 b_2 c_2 c_2 \leq a_2 a_2 b_1 b_2 c_2 c_2 \land a_2 a_2 b_1 b_2 c_2 c_2 \leq a_2 a_2 b_2 b_2 c_1 c_2 \\*
    a_1 a_2 b_2 b_2 c_2 c_2 \leq a_2 a_2 b_2 b_2 c_1 c_2 \\*
    a_1 a_2 c_2 c_2 \leq a_2 a_2 c_1 c_2
  }

  したがって、前順序である。
}

\lem{有理数の準備:前順序の全域性}{
  直積集合$\Z \times \qty(\Z \setminus \qty{0})$上で\lemref{有理数の準備:前順序}の前順序$\preccurlyeq_Q$は以下を満たす。
  \eq*{
    \forall a, b \in \Z \times \qty(\Z \setminus \qty{0}) \qty(a \preccurlyeq_Q b \lor b \preccurlyeq_Q a)
  }
}{
  整数の全順序性より明らか。
}

\lem{有理数の準備:加法可換群}{
  直積集合$\Z \times \qty(\Z \setminus \qty{0})$上の以下の演算$+$は可換群をなす。
  \eq*{
    \forall \qty(a_1, a_2), \qty(b_1, b_2) \in \Z \times \qty(\Z \setminus \qty{0}) \qty(\qty(a_1, a_2) + \qty(b_1, b_2) \defiff \qty(a_1 b_2 + b_1 a_2, a_2 b_2))
  }
}{
  $\Z$は整域よりマグマである。\\*

  結合法則を示す。
  \eqa*{
    a + \qty(b + c)
    &= \qty(a_1, a_2) + \qty(b_1 c_2 + c_1 b_2, b_2 c_2) \\*
    &= \qty(\qty(b_1 c_2 + c_1 b_2) a_2 + a_1 \qty(b_2 c_2), a_2 b_2 c_2) \\*
    &= \qty(a_2 b_2 c_1 + \qty(b_1 a_2 + a_1 b_2) c_2, a_2 b_2 c_2) \\*
    &= \qty(a_1 b_2 + b_1 a_2, a_2 b_2) + \qty(c_1, c_2) \\*
    &= \qty(\qty(a_1, a_2) + \qty(b_1, b_2)) + \qty(c_1, c_2)
  }

  $\qty(0, 1)$は単位元である。

  $\qty(a_1, a_2)$に対して、$\qty(-a_1, a_2)$は逆元である。

  交換法則は、整数の可換環としての性質より明らか。
}

\lem{有理数の準備:乗法可換モノイド}{
  直積集合$\Z \times \qty(\Z \setminus \qty{0})$上の以下の演算$\times$について、可換モノイドをなす。
  \eq*{
    \forall \qty(a_1, a_2), \qty(b_1, b_2) \in \Z \times \qty(\Z \setminus \qty{0}) \qty(\qty(a_1, a_2) \times \qty(b_1, b_2) \defiff \qty(a_1 b_1, a_2 b_2))
  }
}{
  $\Z$は整域より$\times$はマグマであり、$\Z$が環であることから半群である。\\*

  $\qty(1, 1)$は単位元である。\\*

  $\Z$が可換環であることから可換である。
}

\lem{有理数の準備:同値類}{
  直積集合$\Z \times \qty(\Z \setminus \qty{0})$上で以下の自己関係$\sim_Q$は同値類である。
  \eq*{
    \forall \qty(a_1, a_2), \qty(b_1, b_2) \in \Z \times \qty(\Z \setminus \qty{0}) \qty(\qty(a_1, a_2) \sim_Q \qty(b_1, b_2) \defiff \qty(a_1, a_2) \preccurlyeq \qty(b_1, b_2) \land \qty(b_1, b_2) \preccurlyeq \qty(a_1, a_2))
  }
}{
  \lemref{有理数の準備:前順序}より前順序。明らかに対称的であるので、同値類である。
}

\lem{約分}{
  \eq*{
    \forall \qty(a_1, a_2), \qty(b_1, b_2) \in \Z \times \qty(\Z \setminus \qty{0}) \forall n \in \Z \setminus \qty{0} \qty(\qty(a_1, a_2) \sim_Q \qty(b_1, b_2) \leftrightarrow a_1 b_2 = a_2 b_1)
  }
}{
  \lemref{整数の乗法の簡約則}より明らか。
}

\lem{有理数の準備:両立}{
  直積集合$\Z \times \qty(\Z \setminus \qty{0})$上で\lemref{有理数の準備:同値類}の同値類$\sim_Q$は、\lemref{有理数の準備:前順序}の前順序、\lemref{有理数の準備:加法可換群}の加法、\lemref{有理数の準備:乗法可換モノイド}の乗法のそれぞれと両立する。
}{
  前順序と両立することは、$\sim_Q$の定義と前順序の推移性より明らか。\\*

  $\qty(a_1, a_2) \sim_Q \qty(a'_1, a'_2) \land \qty(b_1, b_2) \sim_Q \qty(b'_1, b'_2)$とする。
  \eqg*{
    a_1 a'_2 b_2 b'_2 = a_2 a'_1 b_2 b'_2 \land a_2 a'_2 b_1 b'_2 = a_2 a'_2 b_2 b'_1 \\*
    a_1 a'_2 b_2 b'_2 + a_2 a'_2 b_1 b'_2 = a_2 a'_1 b_2 b'_2 + a_2 a'_2 b_2 b'_1 \\*
    \qty(a_1 b_2 + a_2 b_1) a'_2 b'_2 = \qty(a'_1 b'_2 + a'_2 b'_1) a_2 b_1 \\*
    \qty(a_1 b_2 + a_2 b_1, a_2 b_2) \sim_Q \qty(a'_1 b'_2 + a'_2 b'_1, a'_2 b'_2)
  }

  $\qty(a_1, a_2) \sim_Q \qty(a'_1, a'_2) \land \qty(b_1, b_2) \sim_Q \qty(b'_1, b'_2)$とする。\lemref{整数の乗法の簡約則}より、
  \eqg*{
    a_1 a'_2 b_1 b'_2 = a'_1 a_2 b'_1 b_2 \\*
    \qty(a_1 b_1) \qty(a'_2 b'_2) = \qty(a'_1 b'_1) \qty(a_2 b_2)
  }
}

\dfn{有理数}{
  \lemref{有理数の準備:両立}より定まる商マグマ$\qty(\Z \times \qty(\Z \setminus \qty{0})) / \sim_Q$を有理数と呼び、$\Q$で表す。
  また、その元も有理数と呼ぶ。

  定義より$\Q$は可換群である。
}


\lsubsection{有理数の性質}

\lem{体としての有理数}{
  可換群$\qty(\Q, +)$上の\lemref{有理数の準備:両立}より定まる演算$\times$は、体をなす。
}{
  右分配法則を示す。\lemref{約分}より、
  \eqa*{
    \qty(\qty[\qty(a_1, a_2)] + \qty[\qty(b_1, b_2)]) \times \qty[\qty(c_1, c_2)]
    &= \qty[\qty(a_1 b_2 + b_1 a_2, a_2 b_2)] \times \qty[\qty(c_1, c_2)] \\*
    &= \qty[\qty(\qty(a_1 b_2 + b_1 a_2) c_1, a_2 b_2 c_2)] \\*
    &= \qty[\qty(a_1 c_1 b_2 + b_1 c_1 a_2, a_2 b_2 c_2)] \\*
    &= \qty[\qty(a_1, a_2)] \times \qty[\qty(c_1, c_2)] + \qty[\qty(b_1, b_2)] \times \qty[\qty(c_1, c_2)]
  }

  左分配法則は、右分配法則と交換法則より示される。\\*

  $\qty(a_1, a_2)$について、$a_1 \neq 0$ならば逆元$\qty(a_2, a_1)$が存在する。

  $a_1 = 0$のとき、$\qty[\qty(0, a_2)] = \qty[\qty(0, 0)] = 0_\Q$

  ゆえに、$\Q^\times = \Q \setminus \qty{0_\Q}$
}

\lem{有理数の準備:全順序}{
  \lemref{有理数の準備:両立}より与えられる$\Q$上の前順序$\leq$は全順序である。
}{
  $\sim_Q$の定義より反対称的である。

  \lemref{有理数の準備:前順序の全域性}より全順序である。
}

\thm{順序体としての有理数}{
  $\qty(\Q, \leq)$は順序体である。
}{
  \lemref{有理数の準備:全順序}より全順序。\\*

  第一式について、$a = \qty[\qty(a_1, a_2)], b = \qty[\qty(b_1, b_2)], c = \qty[\qty(c_1, c_2)]$として、$\Z$が順序環であることより、
  \eqa*{
    a_1 a_2 b_2 b_2 &< a_2 a_2 b_1 b_2 \\*
    a_1 a_2 b_2 b_2 c_2 c_2 &< a_2 a_2 b_1 b_2 c_2 c_2 \\*
    a_1 a_2 b_2 b_2 c_2 c_2 + a_2 a_2 b_2 b_2 c_1 c_2 &< a_2 a_2 b_1 b_2 c_2 c_2 + a_2 a_2 b_2 b_2 c_1 c_2 \\*
    \qty(a_1 c_2 + a_2 c_1) a_2 b_2 b_2 c_2 &< \qty(b_1 c_2 + b_2 c_1) a_2 a_2 b_2 c_2 \\*
    \qty(a_1 c_2 + a_2 c_1) a_2 b_2 b_2 c_2 c_2 c_2 &< \qty(b_1 c_2 + b_2 c_1) a_2 a_2 b_2 c_2 c_2 c_2 \\*
    \qty(a_1 c_2 + a_2 c_1) \qty(a_2 c_2) \qty(b_2 c_2) \qty(b_2 c_2) &< \qty(b_1 c_2 + b_2 c_1) \qty(a_2 c_2) \qty(a_2 c_2) \qty(b_2 c_2) \\*
    \qty[\qty(a_1 c_2 + a_2 c_1, a_2 c_2)] &< \qty[\qty(b_1 c_2 + b_2 c_1, b_2 c_2)]
  }

  第二式について、$a = \qty[\qty(a_1, a_2)], b = \qty[\qty(b_1, b_2)]$とすると、$0 < a_1 a_2 \land 0 < b_1 b_2$である。

  ゆえに、$0_\Z < \qty(a_1 b_1) \qty(a_2 b_2)$であるので、$0_\Q < a b$
}

\lem*{
  写像$\varphi \in \Q^\Z, \varphi(a) = \qty[\qty(a, 1)]$は以下を満たす。
  \begin{enumerate}
    \item 単射
    \item 加法、乗法について環準同型
    \item 全順序と両立
  \end{enumerate}
}{
  定義より明らか。
}

\dfn{整数の有理数への埋め込み}{
  \mlemref{0}の写像$\varphi$について、像$\varphi(\Z)$を誤解のない範囲で整数$\Z$と呼ぶ。
}

\cor*{
  \eq*{
    \Z \subset \Q
  }
}

\thmf{$\Q$上のArchimedesの原理}{Q上のArchimedesの原理}{
  順序体$\Q$はArchimedes的である。
}{
  $\forall a, b \in \Q^+$を考える。
  $a = \qty[\qty(a_1, a_2)], b = \qty[\qty(b_1, b_2)]$とする。

  $a_2 < 0$のとき、\lemref{約分}より$a_2 > 0$とする$a_1, a_2$をとることができて、背理法から$a_1 > 0$。
  同様に$b_1, b_2 > 0$である。

  $n = s(a_1 b_2)$について、$a < n b$である。
}

\thmf{$\Q$は第二可算}{Qは第二可算}{
  $\Q$は第二可算である。
}{
  $\mathcal{B} \coloneqq \qty{\sqty{a, b} \mid a, b \in \Q}$は開基である。

  \thmref{自然数の直積は可算}より、$\mathcal{B}$は可算。
}

\thmf{$\Q$は完全不連結}{Qは完全不連結}{
  $\Q$の部分$S$が2点以上を含むならば、$S$は連結でない。
}{
  $a, b \in S \qty(a < b)$とする。

  $U \coloneqq \qty{x \in \Q \mid \qty(1 + \frac{x - a}{b - a})^2 < 2}$とする。$a \in U \land b \notin U$である。

  $U$は開かつ閉集合より、成り立つ。
}


\lsection{体と拡大}

\lsubsection{素体}

\dfn{部分体}{
  体$F$の部分環$R$が体をなすとき、体$R$を体$F$の部分体と呼ぶ。
}

\dfn{素体}{
  体$F$の部分体$F$がのみのとき、体$F$を素体と呼ぶ。
}

\lem{部分素体の存在}{
  体$F$について、$F$のある部分体$K$が存在して、$K$は素体である。

  さらに、体$F$の素体である部分体は一意に定まる。
}{
  以下の集合$K$は、体をなす。
  \eq*{
    K \coloneqq \bigcap \qty{F' \mid \text{$F'$は$F$の部分体}}
  }

  $K$の部分体$L$は、$F$の部分体であるので、定義より$K \subset L$である。
  ゆえに、$K$は素体である。

  体$F$の素体である部分体$K'$について、$K \subset K'$であり、$K'$が素体であることより$K = K'$
}
